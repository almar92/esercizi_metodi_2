\section{Spunti di teoria generale (Alessandro Marcelli)}

\subsection{Concetti generali}

Un'\textbf{equazione differenziale ordinaria} (EDO) di ordine $n$ viene genricamente definita come
\begin{align}
	\Phi(x, u, u', u'', \dots, u^{(n)}) = 0
\end{align}
In questo corso ci si occupa solo di equazioni che possono essere scritte in \textbf{forma normale}, ovvero dove si può esplicitare in funzione della derivata massima
\begin{align}
	u^{(n)}=f(x, u, u', u'', \dots, u^{(n-1)})
\end{align}
con la proprietà di poter essere riscritte come sistemi di $n$ EDO del 1o ordine
nel seguente modo
\begin{align}
	\left\{\begin{array}{c}
		v_1 \quad = u' \hfill \\
		v_2 \quad = v_1' = u'' \hfill \\
		\dots \hfill \\
		v_{n-1} = v_{n-2}' \hfill \\
		v_n \quad = f(x, u, v_1, v_2, \dots, v_{n-1}) \hfill 
	\end{array}
	\right.
\end{align}
Questo sistema definisce un insieme infinito di curve detto \textbf{integrale generale} dell'equazione.

Per individuare una curva specifica bisogna specificare almeno un punto  per cui deve passare, e i rispettivi valori che le varie derivate vi devono assumere fino alla $u^{(n-1)}$. Questo prende il nome di \textbf{problema di Cauchy}.

Nota: visto che come abbiamo visto i problemi di ordine superiore al primo possono essere visti come combinazione di EDO al primo ordine, nella teoria ci appoggeremo a problemi del 1o ordine del tipo
\begin{align}
	\left\{\begin{array}{c}
		u'(x) \; = F(x,u(x)) \hfill \\
		u(x_0) \;  = u_0 \hfill 
	\end{array}
	\right.
\end{align}

\subsection{Esistenza ed unicità delle soluzioni}

L'esistenza delle soluzioni di una EDO è dettata dai seguenti due teoremi:
\begin{enumerate}
	\item \textbf{Teorema di Cauchy:} \textit{sia assegnato un problema di Cauchy in un dominio
		\begin{align}
			\mathcal{D} = (a,b)x(c,d) \taleche (x_0,u_0) \in \mathcal{D}
		\end{align}
		Se rispetto ad $u$ si verifica la condizione di Lipschitzianità per $F(x,u)$
		\begin{align}
			|F(x,u_1) - F(x,u_2)| \leq L|u_1 - u_2|
		\end{align}
		Allora la soluzione del problema di Cauchy esiste ed è unica.	
	}
	\item \textbf{Teorema di Peano:} \textit{sia assegnato un problema di Cauchy in un dominio
		\begin{align}
			\mathcal{D} = (a,b)x(c,d) \taleche (x_0,u_0) \in \mathcal{D}
		\end{align}
		Se la $F(x,u)$ è continua, allora esiste un opportuno intorno $(x_0, x_0+\delta) \in \mathcal{D}$ tale per cui la soluzione del problema esiste.
	}
\end{enumerate}

Qualora ci si trovi nel secondo caso, non è più sufficiente un unico punto per determinare la soluzione, e bisogna ijnvece specificare, in numero pari all'ordine della EDO, delle \textbf{condizioni di bordo} (in inglese \textbf{Boundary Conditions}, BC), arriviamo così a definire due tipi di problemi:
\begin{enumerate}
	\item \textbf{Problemi di Dirichlet:} condizioni al bordo su $u(x)$. 
	
	Nel caso di equazioni lineari, questi si riconducono allo studio degli autovalori dei corrispondenti operatori lineari, e prendono il nome di \textbf{Problemi di Sturm-Liouville}.
	\item \textbf{Problemi di Neumann:} condizioni al bordo su $u'(x)$
\end{enumerate}

\subsection{Distribuzioni}

Le \textbf{distribuzioni} sono una famiglia di operatori lineari integrali definite come
\begin{align}
	(T,\phi) &= \int_{-\infty}^{+\infty} dx \; T(x) \phi(x)\\
	(T',\phi) &= \int_{-\infty}^{+\infty} dx \; T'(x) \phi(x) = -\int_{-\infty}^{+\infty} dx \; T(x) \phi'(x) = -(T,\phi')
\end{align}

Per quanto riguarda questo corso ci limiteremo allo spazio $S^{\infty}$ delle \textbf{funzioni di prova}, con derivate continue di ogni ordine che all'infinito tendono a zero più rapidamente di ogni potenza.

Esempi noti di distribuzioni sono la \textbf{delta} e la \textbf{theta}.

La continuità delle distribuzioni come funzionali consente di ottenere una loro definizione come limite di altre distribuzioni.

\subsection{Equazioni Differenziali Lineari}
Si parla di EDO lineari di ordine $n$ quando possono essere poste nella forma
\begin{align}
	&\mathcal{L}^{(n)}_x u(x) = f(x) \label{edolin}\\
	&\mathcal{L}^{(n)}_x = \sum_{k=0}^{n} c_k(x) \frac{d^k}{dx^k}
\end{align}
Lo studio passa quindi da $F$ ai vari $c_k$. Notiamo come possiamo avere singolarità della soluzione solo intorno a punti singolari dei coefficienti, che definiamo come \textbf{punti singolari fissi}.

Spesso le soluzioni sono note solo attraverso sviluppi in serie di Laurent intorno a tali punti.

Nel caso lineare, la continuità dei coefficienti ci garantisce esistenza ed unicità delle soluzioni.

Definiamo la \textbf{omogenea associata} dell'equazione \ref{edolin} come
\begin{align}
	&\mathcal{L}^{(n)}_x u(x) = \sum_{k=0}^{n} c_k(x) \frac{d^k}{dx^k}u(x) = 0
\end{align}
L'integrale generale di una EDO lineare può essere scritto come combinazione della soluzione dell'omogenea $u_{om}(x)$ e di una soluzione particolare $u_{p}(x)$
\begin{align}
	u(x) = u_{om}(x) + u_{p}(x)
\end{align}

