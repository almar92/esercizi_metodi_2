\section{Equazioni Differenziali Ordinarie Complesse (Alessandro Marcelli)}

\subsection{Spunti di teoria}

\subsubsection{Concetti introduttivi}

Per iniziare il discorso, consideriamo il seguente problema di Cauchy del 1o ordine
\begin{align}
	\double{\dot{u}(x) = p(x)u(x)}{u(x) = u_0 \hfill}
\end{align}
Sappiamo che la soluzione sarà data da
\begin{align}
	u(x) = u_0 \cdot exp \left( - \int_{0}^{x} d\xi \; p(\xi) \right) \label{comp1}
\end{align}
Se prolunghiamo sui complessi e sviluppiamo in serie la $p(\xi)$ otteniamo
\begin{align}
	&p(\xi) = \sum_{k} c_k \xi^k \firstpassage
	&\int d\xi \; p(\xi) = \sum_{k \neq 1} c_k \frac{\xi ^{k+1}}{k+1} + c_{-1} \ln(\xi)
\end{align}
Avremo due casi
\begin{enumerate}
	\item $p(\xi)$ non possiede singolarità, e allora l'unica soluzione possibile è la \ref{comp1}
	\item $p(\xi)$ presenta dei poli, e allora abbiamo due scenari possibili
	\begin{enumerate}
		\item $c_{-1} \neq 0 \implies$ la soluzione è proporzionale a \ref{comp1} e ha una \textbf{singolarità algebrica} dell'origine
		\item $c_{-2} \neq 0 \implies$ la soluzione ha una \textbf{singolarità essenziale} nell'origine
	\end{enumerate}
	La presenza di poli rovina l'unicità della soluzione
\end{enumerate}

Ci concentriamo in questo corso nello studio di equazioni "standard" nella forma
\begin{align}
	\frac{d^2}{dz^2}u(z) + p(z) \frac{d}{dz}u(z) + q(z) u(z) = 0
\end{align}
Preso un punto $z_0 \in \C$, esso viene classificato come
\begin{enumerate}
	\item \textbf{punto regolare} dell'equazione se sia $p(z)$ che $q(z)$ sono analitiche in esso
	\item \textbf{punto singolare} dell'equazione se non lo sono
\end{enumerate}

\newpage

\subsubsection{Punti regolari e soluzione per serie}

Sia il seguente problema di Cauchy
\begin{align}
	\triple{\ddot{u}(z) + p(z) \dot{u}(z) + q(z) u(z) = 0}{\dot{u}(z_0) = c_1 \hfill }{u(z_0) = c_0 \hfill}
\end{align}
e sia $z_0$ un punto regolare dell'equazione. 

Vogliamo dimostrare come in un intorno $|z-z_0| < r_0$ la soluzione sia unica e come sia possibile determinarla con un procedimento iterativo.

Sappiamo come sia possibile trattare i problemi al 2o ordine come combinazione di problemi al 1o ordine. riscriviamo quindi
\begin{align}
	\left\{
	\begin{array}{c}
		\dot{v}(z) = -p(z) v(z) - q(z) u(z) \\
		\dot{u}(z) = v(z) \hfill\\
		v(z_0) = c_1 = \dot{u}(z_0) \hfill\\
		u(z_0) = c_0 \hfill
	\end{array}
	\right.
\end{align}
Questo è un caso particolare dove la dimostrazione passerebbe per la condizione di lipschitzianità invece che per quella di analiticità.

Integriamo formalmente le prime due equazioni, e otteniamo
\begin{align}
	v(z) &= v(z_0) + \int_{z_0}^{z} d\zeta\; [-p(\zeta) v(\zeta) - q(\zeta)u(\zeta)]\\
	u(z) &= u(z_0) + \int_{z_0}^{z} d\zeta \; v(\zeta) 
\end{align}
Andiamo ora a sostituire funzioni costanti pari ai valori di $v(z_0)$ e $u(z_0)$ negli integrali
\begin{align}
	v_1(z) &= v(z_0) + \int_{z_0}^{z} d\zeta\; [-p(\zeta) v_0 - q(\zeta)u_0]\\
	u_1(z) &= u(z_0) + \int_{z_0}^{z} d\zeta \; v_0 
\end{align}
Al passo successivo, poniamo $u_0 = u_1$ e $v_0=v_1$ e ripetiamo
\begin{align}
	v_2(z) &= v_1(z_0) + \int_{z_0}^{z} d\zeta\; [-p(\zeta) v_1(\zeta) - q(\zeta)u_1(\zeta)]\\
	u_2(z) &= u_1(z_0) + \int_{z_0}^{z} d\zeta \; v_1(\zeta) 
\end{align}
Siccome in $z=z_0$ i termini integrali spariscono, avremo in tutte le iterazioni che
\begin{align}
	v_n(z_0) &= v_0\\
	u_n(z_0) &= u_0
\end{align}
e possiamo quindi riscrivere
\begin{align}
	v_2(z) &= v_0 + \int_{z_0}^{z} d\zeta\; [-p(\zeta) v_1(\zeta) - q(\zeta)u_1(\zeta)]\\
	u_2(z) &= u_0 + \int_{z_0}^{z} d\zeta \; v_1(\zeta) \firstpassage
	&\dots \nextpassage
	v_n(z) &= v_0 + \int_{z_0}^{z} d\zeta\; [-p(\zeta) v_{n-1}(\zeta) - q(\zeta)u_{n-1}(\zeta)]\\ 
	u_n(z) &= u_0 + \int_{z_0}^{z} d\zeta \; v_{n-1}(\zeta)
\end{align}
Osserviamo come
\begin{align}
	u_n(z) = u_0 + (u_1 - u_0) + (u_2 - u_1) + \dots + (u_n - u_{n-1})
\end{align}
I termini n-simi possono essere quindi interpretati come somme parziali di ordine $n$
\begin{align}
	v_n(z) &= v_0 + \sum_{k=1}^{n} [v_{k}(z) - v_{k-1}(z)]\\ 
	u_n(z) &= u_0 + \sum_{k=1}^{n} [u_{k}(z) - u_{k-1}(z)]
\end{align}
Vogliamo quindi dimostrare che
\begin{align}
	&\limit{n}{\infty} u_n(z) = u(z)\\
	&\limit{n}{\infty} v_n(z) = v(z)\\
	&u(z), v(z) \quad \text{analitiche}
\end{align}
Definite le due grandezze
\begin{align}
	M &= max(|p|,|q|)\\
	m &> max(|u_0|,|v_0|)
\end{align}
Confrontandole con le espressioni di $u_1(z)$ e $v_1(z)$ otteniamo
\begin{align}
	|u_1(z) - u_0| &\leq m |z-z_0|\\
	|v_1(z) - v_0| &\leq 2Mm |z-z_0|	
\end{align}
Definendo
\begin{align}
	M_1 = max(m, 2Mm)
\end{align}
Possiamo maggiorare come
\begin{align}
	|u_1(z) - u_0| &\leq M_1 |z-z_0|\\
	|v_1(z) - v_0| &\leq M_1 |z-z_0|	
\end{align}
Passando all'iterazione successiva avremo
\begin{align}
	u_2(z) - u_1(z) = \left(\cancel{u_0} + \int_{z_0}^{z} d\zeta \; v_1(\zeta) \right) - \left(\cancel{u_0} + \int_{z_0}^{z} d\zeta \; v_0(\zeta) \right) = \int_{z_0}^{z} d\zeta \; [v_1(\zeta) - v_0(\zeta)]
\end{align}
E analogamente avremo
\begin{align}
	v_2(z) - v_1(z) = \int_{z_0}^{z} d\zeta\; [-p(\zeta) (v_1(\zeta)-v_0) - q(\zeta)(u_1(\zeta)-u_0)]
\end{align}
Anche qui possiamo quindi maggiorare come
\begin{align}
	u_2(z) - u_1(z) &= M_1\int_{z_0}^{z} d\zeta\; [\zeta - z_0)]\\
	v_2(z) - v_1(z) &= 2M_1m\int_{z_0}^{z} d\zeta\; [\zeta - z_0)]
\end{align}
Anche qui possiamo maggiorare con
\begin{align}
	&M_2 = max(M_1, 2M_1 m) \firstpassage
	&M_1\int_{z_0}^{z} d\zeta\; [\zeta - z_0)] \leq M_2 \frac{|z-z_0|^2}{2} = M_2 \frac{r^2}{2}
\end{align}
Da cui possiamo scrivere
\begin{align}
	|u_2(z) - u_1(z)| &\leq M_n \frac{r^n}{n!}\\
	|v_2(z) - v_1(z)| &\leq M_n \frac{r^n}{n!}	
\end{align}
Se adesso riprendiamo la
\begin{align}
	u_n(z) &= u_0 + \sum_{k=1}^{n} [u_{k}(z) - u_{k-1}(z)]
\end{align}
Possiamo riscrivere
\begin{align}
	|u_n(z)| &\leq \sum_{k=0}^{n} \frac{(Mr)^n}{n!}
\end{align}
Stesso discorso facciamo per $v(z)$.

Esiste quindi per forza un $r_0$ tale per cui la serie converge in modo assoluto e uniforme, e che quindi definisce una funzione analitica.

Possiamo quindi ridefinire la $u(z)$ tramite la sua espansione di taylor come
\begin{align}
	\triple{u(z) = \sum_{n=0}^{\infty} c_n(z-z_0)^n}{\dot{u}(z_0) = v_0 = c_1 \hfill}{u(z_0) = u_0 = c_0 \hfill}
\end{align}
Da qui non ci resta che ricavare i coefficienti. Esplicitiamo i termini e sostituiamo
\begin{align}
       u(z) &= c_0 + c_1 (z-z_0) + \sum_{n=2}^{\infty} c_n(z-z_0)^n = \nonumber \\
            &= u_0 + v_0 (z-z_0) + \sum_{n=2}^{\infty} c_n(z-z_0)^n \firstpassage
 \dot{u}(z) &= v_0 + \sum_{n=2}^{\infty} c_n n(z-z_0)^{n-1} \firstpassage
\ddot{u}(z) &= \sum_{n=2}^{\infty} c_n n(n-1)(z-z_0)^{n-2}
\end{align}
Sostituiamo nell'equazione generale e otteniamo
\begin{align}
	\ddot{u}(z) =& - p(z) \dot{u}(z) - q(z) u(z) \firstpassage
	\sum_{n=2}^{\infty} c_n n(n-1)(z-z_0)^{n-2} =& -p(z) \left(v_0 + \sum_{n=2}^{\infty} c_n n(z-z_0)^{n-1}\right) - q(z) \left( u_0 + v_0 (z-z_0) + \sum_{n=2}^{\infty} c_n(z-z_0)^n \right) \nextpassage
	\sum_{n=2}^{\infty} c_n n(n-1)(z-z_0)^{n-2} =& -p(z)v_0 - q(z)[u_0 + v_0 (z-z_0) ] + \nonumber\\ 
	&- \sum_{n=2}^{\infty} c_n [np(z)(z-z_0)^{n-1} + q(z)(z-z_0)^n] \label{serieode}
\end{align}
\newpage
Per comodità ripetiamo qui la \ref{serieode}
\begin{align}
	\sum_{n=2}^{\infty} c_n n(n-1)(z-z_0)^{n-2} =& -p(z)v_0 - q(z)[u_0 + v_0 (z-z_0) ] + \nonumber\\ 
	&- \sum_{n=2}^{\infty} c_n [np(z)(z-z_0)^{n-1} + q(z)(z-z_0)^n]
\end{align}
Si procede quindi per confronto, trovando
\begin{align}
	n = 2 \to c_2 \cdot 2 \cdot (2-1) (z-z_0)^{2-2} = 2c_2
\end{align}
A dx della \ref{serieode} non abbiamo alcun termine corrispondente, ergo
\begin{align}
	c_2 = 0
\end{align}
Procediamo e troviamo
\begin{align}
	n = 3 \to c_3 \cdot 3 \cdot (3-1) (z-z_0)^{3-2} = 6c_3 (z-z_0)
\end{align}
A dx troviamo che il termine corrispondente è
\begin{align}
	- q(z)v_0 (z-z_0)
\end{align}
E dunque per confronto
\begin{align}
	&6c_3 \cancel{(z-z_0)} = - q(z)v_0 \cancel{(z-z_0)}\firstpassage
	&c_3 = -\frac{v_0}{6}q(z) 
\end{align}
Proseguiamo imperterriti a mazzetta fino a trovare relazioni ricorrenti tra i coefficienti, e otteniamo così il  nostro sviluppo in serie.


\subsubsection{Punti singolari}

Per costruzione, le singolarità di una EDO della forma
\begin{align}
	\ddot{u}(z) + p(z) \dot{u}(z) + q(z)u(z) = 0
\end{align}
devono coincidere con quelle delle funzioni $p(z)$ e $q(z)$, esono divise in due categorie
\begin{enumerate}
	\item punti \textbf{singolari regolari} (anche detti \textbf{inessenziali} o \textbf{fuchsiani}).
	
	Sono particolarmente importanti quelli in cui le soluzioni non presentano singolarità essenziali.
	\item punti \textbf{singolari irregolari}
\end{enumerate}
Concentriamoci un attimo sul caso di un punto singolare regolare $z_0$. Le soluzioni che possono essere trovate intorno ad esso di base non saranno in generale monodrome, e qualora si applichi un prolungamento ci troveremo ad avere
\begin{align}
	(z-z_0) &\to (z-z_0)e^{2\pi i}\\
	u_1(z)  &\to U_1(z)\\
	u_2(z)  &\to U_2(z)
\end{align}
\newpage
Sappiamo però che $u_1(z)$ e $u_2(z)$ sono linearmente indipendenti, e formano quindi una base nello spazio delle soluzioni.

Di conseguenza possiamo scrivere
\begin{align}
	&\double{U_1(z) = a_{11} u_1(z) + a_{12} u_2(z)}{U_2(z) = a_{21} u_1(z) + a_{22} u_2(z)}\firstpassage
	&\left(
	 \begin{array}{c}
		 U_1(z)\\
		 U_2(z)
	 \end{array}\right) = 
	 \begin{pmatrix}
		 a_{11} & a_{12}\\
		 a_{21} & a_{22}
	 \end{pmatrix} 	\left(
     \begin{array}{c}
		 u_1(z)\\
		 u_2(z)
	\end{array}\right)
\end{align}
Si può dimostrare come
\begin{align}
	W(U_1,U_2) = \det(A) \cdot W(u_1, u_1)
\end{align}

La matrice $A$ è non singolare e dipendente dalla polidromia di $u_1(z)$ e $u_2(z)$. Come di consueto gli autovalori si ricavano dall'equazione caratteristica
\begin{align}
	\det(A -\lambda \mathbb{1}) = 0
\end{align}
e possono essere distinti o coincidenti.

A noi interessa trovare una base dove le polidromie in gioco siano le più semplici possibili. 

Questo si traduce nello scegliere una base che renda o diagonale o di Jordan (diagonale a blocchi) la matrice A. Andiamo a studiare i due possibili casi.

\subsubsubsection{Soluzioni per autovalori di $A$ non coincidenti}
Sia il caso $\lambda_1 \neq \lambda_2$, in questo caso possiamo porre
\begin{align}
	A = \begin{pmatrix}
		\lambda_1 & 0 \\
		0 & \lambda_1 
		\end{pmatrix} \to \double{U_1 = \lambda_1 u_1}{U_2 = \lambda_2 u_2}
\end{align}	
Definita la polidromia delle potenze
\begin{align}
	(z-z_0)^{\rho_i} \to (z-z_0)^{\rho_i} e^{2\pi i \rho_i} 
\end{align}
Possiamo estenderla alle soluzioni imponendo
\begin{align}
	\rho_i = \frac{\ln(\lambda_i)}{2\pi i}
\end{align}
questo rende infatti monodrome le funzioni
\begin{align}
	f_i(z) = \frac{u_i(z)}{(z-z_0)^{\rho_i}}
\end{align}
E quindi, ricordando che $z_0$ è una singolarità essenziale, possiamo espandere in esso le $f_i(z)$ in serie di Laurent, ottenendo
\begin{align}
	&f_i(z) = \sum_{k\in \Z} c_k (z-z_0)^k = \frac{u_i(z)}{(z-z_0)^{\rho_i}}\firstpassage
	&\double{u_1(z) = (z-z_0)^{\rho_1}\sum_{k\in \Z} c_k (z-z_0)^k}{u_2(z) = (z-z_0)^{\rho_2}\sum_{k\in \Z} c_k (z-z_0)^k}
\end{align}

\newpage

\subsubsubsection{Soluzioni per autovalori di $A$ coincidenti}
Se invece ci troviamo nel caso $\lambda_1 = \lambda_2$ non è detto che possiamo riscrivere la matrice in forma diagonale. Possiamo però ricondurci ad una matrice inferiore di Hessemberg del tipo
\begin{align}
	A = \begin{pmatrix}
		\lambda_1 & 0        \\
		b_{21}    & \lambda_1
	\end{pmatrix} \to \double{U_1(z) = \lambda_1 u_1(z) \hfill}{U_2(z) = b_{21} u_1(z) + \lambda_1 u_2(z)}
\end{align}
Riprendendo le funzioni definite prima, in questo caso vale ancora la monodromia per
\begin{align}
	f_1(z) =  \frac{u_1(z)}{(z-z_0)^{\rho_1}}
\end{align}
Invece per l'altra possiamo procedere nel seguente modo
\begin{align}
	\frac{U_2(z)}{U_1(z)} = \frac{b_{21} u_1(z) + \lambda_1 u_2(z)}{\lambda_1 u_1(z)} = \frac{b_{21}}{\lambda_1} + \frac{u_2(z)}{u_1(z)} 
\end{align}
Possiamo quindi affermare che il rapporto dei due vettori della base esibisce le stesse proprietà di polidromia del logaritmo (perché? cercare su appunti o scocciare paolo), e quindi possiamo definire la funzione monodroma
\begin{align}
	g(z) = \frac{u_2(z)}{u_1(z)} - \frac{b_{21}}{\lambda_1} \cdot \frac{\ln (z-z_0)}{2\pi i}
\end{align}
che possiamo espandere in serie di Laurent intorno a $z_0$
\begin{align}
	&g(z) = \sum_{k\in \Z}d_k (z-z_0)^k = \frac{u_2(z)}{u_1(z)} -A\ln (z-z_0) \spacer A = \frac{b_{21}}{\lambda_1} \cdot \frac{1}{2\pi i} \firstpassage
	&\frac{u_2(z)}{u_1(z)} = A\ln (z-z_0) + \sum_{k\in \Z}d_k (z-z_0)^k\nextpassage
	&u_2(z) = u_1(z) \cdot \left[ A\ln (z-z_0) + \sum_{k\in \Z}d_k (z-z_0)^k \right] \nextpassage
	&u_2(z) = A\ln(z-z_0)u_1(z) + u_1(z) \sum_{k\in \Z}d_k (z-z_0)^k\nextpassage
	&u_2(z) = A\ln(z-z_0)u_1(z) + (z-z_0)^{\rho_1}\sum_{k\in \Z} c_k (z-z_0)^k \sum_{k\in \Z}d_k (z-z_0)^k\nextpassagecomm{applico il prodotto di Cauchy?}
	&u_2(z) = A\ln(z-z_0)u_1(z) + (z-z_0)^{\rho_1}\sum_{k\in \Z}\sum_{i=0}^{k} c_k d_i (z-z_0)^{k+i}\nextpassage
	&u_2(z) = A\ln(z-z_0)u_1(z) + (z-z_0)^{\rho_1} \sum_{k\in \Z}b_k (z-z_0)^k
\end{align}

\newpage

\subsubsubsection{Punti singolari regolari e teorema di Fuchs}
Qualora $z_0$ sia un punto singolare regolare della funzione, e quindi non essenziale, le serie di Laurent devono troncarsi ad una potenza finita negativa. Dovremo quindi riscrivere, nel caso di autovalori non coincidenti
\begin{align}
	&\double{u_1(z) = (z-z_0)^{\tilde{\rho}_1}\sum_{n = -N}^{\infty} c_n (z-z_0)^n}{u_2(z) = (z-z_0)^{\tilde{\rho}_2}\sum_{n=-M}^{\infty} d_n (z-z_0)^n}
\end{align}
Sfruttando le proprietà di polidromia, sappiamo che possiamo porre
\begin{align}
	\tilde{\rho}_i = \rho_i + n_i \spacer \double{n_1 = N}{n_2 = M} \in \N
\end{align} 
in modo che
\begin{enumerate}
	\item essa sia contenuta tutta nei prefattori esterni alla somma
	\item la coda della serie di Laurent venga eliminata
\end{enumerate}
e possiamo quindi riscrivere
\begin{align}
	&\double{u_1(z) = (z-z_0)^{\rho_1}\sum_{k = 0}^{\infty} c_k (z-z_0)^k}{u_2(z) = (z-z_0)^{\rho_2}\sum_{k=0}^{\infty} d_k (z-z_0)^k} \firstpassage
	&\double{u_1(z) = \sum_{k = 0}^{\infty} c_k (z-z_0)^{k+\rho_1}}{u_2(z) = \sum_{k=0}^{\infty} d_k (z-z_0)^{k + \rho_2}}
\end{align}
Allargando il discorso al caso di autovalori coincidenti, possiamo scrivere
\begin{align}
	&\double{u_1(z) = \sum_{k = 0}^{\infty} c_k (z-z_0)^{k+\rho_1} \hfill }{u_2(z) = A \ln(z-z_0) \sum_{k = 0}^{\infty} c_k (z-z_0)^{k+\rho_1} + \sum_{k=0}^{\infty}b_k (z-z_0)^{k+\rho_1}} \firstpassage
	&\double{u_1(z) = \sum_{k = 0}^{\infty} c_k (z-z_0)^{k+\rho_1} \hfill }{u_2(z) = A \ln(z-z_0) \sum_{k = 0}^{\infty} c_k (z-z_0)^{k+\rho_1} + \sum_{k=0}^{\infty}b_k (z-z_0)^{k+\rho_1}}
\end{align}
Se andiamo a riprendere la forma generale della EDO
\begin{align}
	\ddot{u}(z) + p(z)\dot{u}(z) + q(z)u(z) = 0
\end{align}
a partire dalla forma delle soluzioni possiamo ricavare la forma dei coefficienti:
\begin{enumerate}
	\item Possiamo ricavare $q(z)$ direttamente dalla EDO
	\begin{align}
		q(z) = - \frac{\ddot{u}_1(z)}{u_1(z)} - p(z)\frac{\dot{u}_1(z)}{u_1(z)}
	\end{align}
	\item Invece $p(z)$ dall'equazione di Liouville che abbiamo introdotto in precedenza
	\begin{align}
		&\dot{W} + p(z) W = 0 \firstpassage
		& p(z) = - \frac{\dot{W}}{W} \nextpassage
		& p(z) = - \frac{u_1(z) \ddot{u}_2(z) - u_2(z) \ddot{u}_1(z)}{u_1(z) \dot{u}_2(z) - u_2(z) \dot{u}_1(z)} \label{pidizeta}
	\end{align}
\end{enumerate}
Detto questo, prendiamo in esame il caso degli autovalori non coincidenti.

Riscrivendo per comodità le soluzioni come
\begin{align}
	&\double{u_1(z) = (z-z_0)^{\rho_1} R_1(z)}{u_2(z) = (z-z_0)^{\rho_2}R_2(z) } \spacer \double{R_1(z) = \sum_{k = 0}^{\infty} c_k (z-z_0)^{k} }{ R_2(z) = \sum_{k = 0}^{\infty} d_k (z-z_0)^{k}}
\end{align}
E andiamo a calcolarne le derivate prime e seconde
\begin{align}
	\dot{u}_i(z) &= \rho_i(z-z_0)^{\rho_i-1} R_i(z) + (z-z_0)^{\rho_i} \dot{R}_i(z)\firstpassage
	\dot{u}_i(z) &= (z-z_0)^{\rho_i-1}\cdot [\rho_i R_i(z) +(z-z_0)\dot{R}_i(z) ] = \nonumber\\
				 &= (z-z_0)^{\rho_i-1}\cdot \tilde{R}_i(z) \firstpassagecomm{ripetendo il procedimento}
	\ddot{u}_i(z)&= (z-z_0)^{\rho_i-2}\cdot \tilde{\tilde{R}}_i(z)
\end{align}
Siamo così andati a definire le funzioni $R_i(z)$, $\tilde{R}_i(z)$ $\tilde{\tilde{R}}_i(z)$ che in $z_0$ sono
\begin{enumerate}
	\item analitiche 
	\item mai nulle
\end{enumerate}

Se quindi ad esempio prendiamo la \ref{pidizeta} possiamo scrivere
\begin{align}
	p(z) &= \frac{(z-z_0)^{\rho_1 + \rho_2 -2} (R_1(z)\tilde{\tilde{R}}_2(z) - R_2(z)\tilde{\tilde{R}}_1(z))}{(z-z_0)^{\rho_1 + \rho_2 -1} (R_1(z)\tilde{R}_2(z) - R_2(z)\tilde{R}_1(z))} = \nonumber \\
	 	 &=(z-z_0)^{-1} \frac{R_1(z)\tilde{\tilde{R}}_2(z) - R_2(z)\tilde{\tilde{R}}_1(z)}{R_1(z)\tilde{R}_2(z) - R_2(z)\tilde{R}_1(z)} \nextpassage
	p(z) &= \frac{P(z)}{z-z_0} \spacer P(z) = \frac{R_1(z)\tilde{\tilde{R}}_2(z) - R_2(z)\tilde{\tilde{R}}_1(z)}{R_1(z)\tilde{R}_2(z) - R_2(z)\tilde{R}_1(z)}
\end{align}
E in modo analogo troviamo
\begin{align}
	q(z) &= -\frac{(z-z_0)^{\rho_1-2}\cdot \tilde{\tilde{R}}_1(z)}{(z-z_0)^{\rho_1} R_1(z)} - p(z)\frac{(z-z_0)^{\rho_1-1}\cdot \tilde{R}_1(z)}{(z-z_0)^{\rho_1} R_1(z)} = \nonumber \\
		 &= -\frac{1}{(z-z_0)^{2}} \cdot \frac{\tilde{\tilde{R}}_1(z)}{R_1(z)} - \frac{P(z)}{z-z_0} \cdot \frac{1}{z-z_0} \cdot \frac{\tilde{R}_1(z)}{R_1(z)} = \nonumber \\
		 &=	-\frac{1}{(z-z_0)^{2}} \cdot \left[ \frac{\tilde{\tilde{R}}_1(z)}{R_1(z)} - P(z)\cdot \frac{\tilde{R}_1(z)}{R_1(z)} \right] = \nextpassage
	q(z) &= \frac{Q(z)}{(z-z_0)^2} \spacer Q(z) = P(z)\cdot \frac{\tilde{R}_1(z)}{R_1(z)} - \frac{\tilde{\tilde{R}}_1(z)}{R_1(z)} 
\end{align}

\newpage

Per costruzione, sia $P(z)$ che $Q(z)$ sono analitiche, e quindi
\begin{enumerate}
	\item $p(z)$ ha al massimo un polo singolo
	\item $q(z)$ ha al massimo un polo doppio
\end{enumerate}

Possiamo quindi enunciare il \textbf{Teorema di Fuch:} \textit{presa una EDO nella forma
\begin{align}
	\ddot{u}(z) + p(z)\dot{u}(z) + q(z)u(z) = 0
\end{align}
un punto $z_0$ si dice \textbf{singolare regolare} o \textbf{fuchsiano} se e solo se valgono le seguenti condizioni
\begin{align}
	\double{\limit{z}{z_0} p(z)(z-z_0) = p_0}{\limit{z}{z_0} q(z)(z-z_0)^2 = q_0} \spacer p_0, q_0 \; \text{costanti}
\end{align}
}
Questo significa che le EDO oggetto di studio di questo corso, in presenza di punti fuchsiani assumono la forma
\begin{align}
	\ddot{u}(z) + \frac{P(z-z_0)}{z-z_0} \cdot \dot{u}(z) + \frac{Q(z-z_0)}{(z-z_0)^2} \cdot u(z) = 0
\end{align}
Dobbiamo dimostrare ancora due cose
\begin{enumerate}
	\item che i coefficienti siano determinati a partire dalla EDO 
	\item che le serie corrispondenti siano convergenti
\end{enumerate}

\newpage

\subsection{Esempi}
N'aggio fatto in tempo, tanto Bufalini le ha scritte bene su Latex.