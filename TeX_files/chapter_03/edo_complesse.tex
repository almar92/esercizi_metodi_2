\section{Equazioni Differenziali Ordinarie Complesse (Alessandro Marcelli)}

\subsection{Spunti di teoria}

\subsubsection{Concetti introduttivi}

Per iniziare il discorso, consideriamo il seguente problema di Cauchy del 1o ordine
\begin{align}
	\double{\dot{u}(x) = p(x)u(x)}{u(x) = u_0 \hfill}
\end{align}
Sappiamo che la soluzione sarà data da
\begin{align}
	u(x) = u_0 \cdot exp \left( - \int_{0}^{x} d\xi \; p(\xi) \right) \label{comp1}
\end{align}
Se prolunghiamo sui complessi e sviluppiamo in serie la $p(\xi)$ otteniamo
\begin{align}
	&p(\xi) = \sum_{k} c_k \xi^k \firstpassage
	&\int d\xi \; p(\xi) = \sum_{k \neq 1} c_k \frac{\xi ^{k+1}}{k+1} + c_{-1} \ln(\xi)
\end{align}
Avremo due casi
\begin{enumerate}
	\item $p(\xi)$ non possiede singolarità, e allora l'unica soluzione possibile è la \ref{comp1}
	\item $p(\xi)$ presenta dei poli, e allora abbiamo due scenari possibili
	\begin{enumerate}
		\item $c_{-1} \neq 0 \implies$ la soluzione è proporzionale a \ref{comp1} e ha una \textbf{singolarità algebrica} dell'origine
		\item $c_{-2} \neq 0 \implies$ la soluzione ha una \textbf{singolarità essenziale} nell'origine
	\end{enumerate}
	La presenza di poli rovina l'unicità della soluzione
\end{enumerate}

Ci concentriamo in questo corso nello studio di equazioni "standard" nella forma
\begin{align}
	\frac{d^2}{dz^2}u(z) + p(z) \frac{d}{dz}u(z) + q(z) u(z) = 0
\end{align}
Preso un punto $z_0 \in \C$, esso viene classificato come
\begin{enumerate}
	\item \textbf{punto regolare} dell'equazione se sia $p(z)$ che $q(z)$ sono analitiche in esso
	\item \textbf{punto singolare} dell'equazione se non lo sono
\end{enumerate}

\newpage

\subsubsection{Punti regolari e soluzione per serie}

Sia il seguente problema di Cauchy
\begin{align}
	\triple{\ddot{u}(z) + p(z) \dot{u}(z) + q(z) u(z) = 0}{\dot{u}(z_0) = c_1 \hfill }{u(z_0) = c_0 \hfill}
\end{align}
e sia $z_0$ un punto regolare dell'equazione. 

Vogliamo dimostrare come in un intorno $|z-z_0| < r_0$ la soluzione sia unica e come sia possibile determinarla con un procedimento iterativo.

Sappiamo come sia possibile trattare i problemi al 2o ordine come combinazione di problemi al 1o ordine. riscriviamo quindi
\begin{align}
	\left\{
	\begin{array}{c}
		\dot{v}(z) = -p(z) v(z) - q(z) u(z) \\
		\dot{u}(z) = v(z) \hfill\\
		v(z_0) = c_1 = \dot{u}(z_0) \hfill\\
		u(z_0) = c_0 \hfill
	\end{array}
	\right.
\end{align}
Questo è un caso particolare dove la dimostrazione passerebbe per la condizione di lipschitzianità invece che per quella di analiticità.

Integriamo formalmente le prime due equazioni, e otteniamo
\begin{align}
	v(z) &= v(z_0) + \int_{z_0}^{z} d\zeta\; [-p(\zeta) v(\zeta) - q(\zeta)u(\zeta)]\\
	u(z) &= u(z_0) + \int_{z_0}^{z} d\zeta \; v(\zeta) 
\end{align}
Andiamo ora a sostituire funzioni costanti pari ai valori di $v(z_0)$ e $u(z_0)$ negli integrali
\begin{align}
	v_1(z) &= v(z_0) + \int_{z_0}^{z} d\zeta\; [-p(\zeta) v_0 - q(\zeta)u_0]\\
	u_1(z) &= u(z_0) + \int_{z_0}^{z} d\zeta \; v_0 
\end{align}
Al passo successivo, poniamo $u_0 = u_1$ e $v_0=v_1$ e ripetiamo
\begin{align}
	v_2(z) &= v_1(z_0) + \int_{z_0}^{z} d\zeta\; [-p(\zeta) v_1(\zeta) - q(\zeta)u_1(\zeta)]\\
	u_2(z) &= u_1(z_0) + \int_{z_0}^{z} d\zeta \; v_1(\zeta) 
\end{align}
Siccome in $z=z_0$ i termini integrali spariscono, avremo in tutte le iterazioni che
\begin{align}
	v_n(z_0) &= v_0\\
	u_n(z_0) &= u_0
\end{align}
e possiamo quindi riscrivere
\begin{align}
	v_2(z) &= v_0 + \int_{z_0}^{z} d\zeta\; [-p(\zeta) v_1(\zeta) - q(\zeta)u_1(\zeta)]\\
	u_2(z) &= u_0 + \int_{z_0}^{z} d\zeta \; v_1(\zeta) \firstpassage
	&\dots \nextpassage
	v_n(z) &= v_0 + \int_{z_0}^{z} d\zeta\; [-p(\zeta) v_{n-1}(\zeta) - q(\zeta)u_{n-1}(\zeta)]\\ 
	u_n(z) &= u_0 + \int_{z_0}^{z} d\zeta \; v_{n-1}(\zeta)
\end{align}
Osserviamo come
\begin{align}
	u_n(z) = u_0 + (u_1 - u_0) + (u_2 - u_1) + \dots + (u_n - u_{n-1})
\end{align}
I termini n-simi possono essere quindi interpretati come somme parziali di ordine $n$
\begin{align}
	v_n(z) &= v_0 + \sum_{k=1}^{n} [v_{k}(z) - v_{k-1}(z)]\\ 
	u_n(z) &= u_0 + \sum_{k=1}^{n} [u_{k}(z) - u_{k-1}(z)]
\end{align}
Vogliamo quindi dimostrare che
\begin{align}
	&\limit{n}{\infty} u_n(z) = u(z)\\
	&\limit{n}{\infty} v_n(z) = v(z)\\
	&u(z), v(z) \quad \text{analitiche}
\end{align}
Definite le due grandezze
\begin{align}
	M &= max(|p|,|q|)\\
	m &> max(|u_0|,|v_0|)
\end{align}
Confrontandole con le espressioni di $u_1(z)$ e $v_1(z)$ otteniamo
\begin{align}
	|u_1(z) - u_0| &\leq m |z-z_0|\\
	|v_1(z) - v_0| &\leq 2Mm |z-z_0|	
\end{align}
Definendo
\begin{align}
	M_1 = max(m, 2Mm)
\end{align}
Possiamo maggiorare come
\begin{align}
	|u_1(z) - u_0| &\leq M_1 |z-z_0|\\
	|v_1(z) - v_0| &\leq M_1 |z-z_0|	
\end{align}
Passando all'iterazione successiva avremo
\begin{align}
	u_2(z) - u_1(z) = \left(\cancel{u_0} + \int_{z_0}^{z} d\zeta \; v_1(\zeta) \right) - \left(\cancel{u_0} + \int_{z_0}^{z} d\zeta \; v_0(\zeta) \right) = \int_{z_0}^{z} d\zeta \; [v_1(\zeta) - v_0(\zeta)]
\end{align}
E analogamente avremo
\begin{align}
	v_2(z) - v_1(z) = \int_{z_0}^{z} d\zeta\; [-p(\zeta) (v_1(\zeta)-v_0) - q(\zeta)(u_1(\zeta)-u_0)]
\end{align}
Anche qui possiamo quindi maggiorare come
\begin{align}
	u_2(z) - u_1(z) &= M_1\int_{z_0}^{z} d\zeta\; [\zeta - z_0)]\\
	v_2(z) - v_1(z) &= 2M_1m\int_{z_0}^{z} d\zeta\; [\zeta - z_0)]
\end{align}
Anche qui possiamo maggiorare con
\begin{align}
	&M_2 = max(M_1, 2M_1 m) \firstpassage
	&M_1\int_{z_0}^{z} d\zeta\; [\zeta - z_0)] \leq M_2 \frac{|z-z_0|^2}{2} = M_2 \frac{r^2}{2}
\end{align}
Da cui possiamo scrivere
\begin{align}
	|u_2(z) - u_1(z)| &\leq M_n \frac{r^n}{n!}\\
	|v_2(z) - v_1(z)| &\leq M_n \frac{r^n}{n!}	
\end{align}
Se adesso riprendiamo la
\begin{align}
	u_n(z) &= u_0 + \sum_{k=1}^{n} [u_{k}(z) - u_{k-1}(z)]
\end{align}
Possiamo riscrivere
\begin{align}
	|u_n(z)| &\leq \sum_{k=0}^{n} \frac{(Mr)^n}{n!}
\end{align}
Stesso discorso facciamo per $v(z)$.

Esiste quindi per forza un $r_0$ tale per cui la serie converge in modo assoluto e uniforme, e che quindi definisce una funzione analitica.

Possiamo quindi ridefinire la $u(z)$ tramite la sua espansione di taylor come
\begin{align}
	\triple{u(z) = \sum_{n=0}^{\infty} c_n(z-z_0)^n}{\dot{u}(z_0) = v_0 = c_1 \hfill}{u(z_0) = u_0 = c_0 \hfill}
\end{align}
Da qui non ci resta che ricavare i coefficienti. Esplicitiamo i termini e sostituiamo
\begin{align}
       u(z) &= c_0 + c_1 (z-z_0) + \sum_{n=2}^{\infty} c_n(z-z_0)^n = \nonumber \\
            &= u_0 + v_0 (z-z_0) + \sum_{n=2}^{\infty} c_n(z-z_0)^n \firstpassage
 \dot{u}(z) &= v_0 + \sum_{n=2}^{\infty} c_n n(z-z_0)^{n-1} \firstpassage
\ddot{u}(z) &= \sum_{n=2}^{\infty} c_n n(n-1)(z-z_0)^{n-2}
\end{align}
Sostituiamo nell'equazione generale e otteniamo
\begin{align}
	\ddot{u}(z) =& - p(z) \dot{u}(z) - q(z) u(z) \firstpassage
	\sum_{n=2}^{\infty} c_n n(n-1)(z-z_0)^{n-2} =& -p(z) \left(v_0 + \sum_{n=2}^{\infty} c_n n(z-z_0)^{n-1}\right) - q(z) \left( u_0 + v_0 (z-z_0) + \sum_{n=2}^{\infty} c_n(z-z_0)^n \right) \nextpassage
	\sum_{n=2}^{\infty} c_n n(n-1)(z-z_0)^{n-2} =& -p(z)v_0 - q(z)[u_0 + v_0 (z-z_0) ] + \nonumber\\ 
	&- \sum_{n=2}^{\infty} c_n [np(z)(z-z_0)^{n-1} + q(z)(z-z_0)^n] \label{serieode}
\end{align}
\newpage
Per comodità ripetiamo qui la \ref{serieode}
\begin{align}
	\sum_{n=2}^{\infty} c_n n(n-1)(z-z_0)^{n-2} =& -p(z)v_0 - q(z)[u_0 + v_0 (z-z_0) ] + \nonumber\\ 
	&- \sum_{n=2}^{\infty} c_n [np(z)(z-z_0)^{n-1} + q(z)(z-z_0)^n]
\end{align}
Si procede quindi per confronto, trovando
\begin{align}
	n = 2 \to c_2 \cdot 2 \cdot (2-1) (z-z_0)^{2-2} = 2c_2
\end{align}
A dx della \ref{serieode} non abbiamo alcun termine corrispondente, ergo
\begin{align}
	c_2 = 0
\end{align}
Procediamo e troviamo
\begin{align}
	n = 3 \to c_3 \cdot 3 \cdot (3-1) (z-z_0)^{3-2} = 6c_3 (z-z_0)
\end{align}
A dx troviamo che il termine corrispondente è
\begin{align}
	- q(z)v_0 (z-z_0)
\end{align}
E dunque per confronto
\begin{align}
	&6c_3 \cancel{(z-z_0)} = - q(z)v_0 \cancel{(z-z_0)}\firstpassage
	&c_3 = -\frac{v_0}{6}q(z) 
\end{align}
Proseguiamo imperterriti a mazzetta fino a trovare relazioni ricorrenti tra i coefficienti, e otteniamo così il  nostro sviluppo in serie.


\subsubsection{Punti singolari}

Per costruzione, le singolarità di una EDO della forma
\begin{align}
	\ddot{u}(z) + p(z) \dot{u}(z) + q(z)u(z) = 0
\end{align}
devono coincidere con quelle delle funzioni $p(z)$ e $q(z)$, esono divise in due categorie
\begin{enumerate}
	\item punti \textbf{singolari regolari} (anche detti \textbf{inessenziali} o \textbf{fuchsiani}).
	
	Sono particolarmente importanti quelli in cui le soluzioni non presentano singolarità essenziali.
	\item punti \textbf{singolari irregolari}
\end{enumerate}
Concentriamoci un attimo sul caso di un punto singolare regolare $z_0$. Le soluzioni che possono essere trovate intorno ad esso di base non saranno in generale monodrome, e qualora si applichi un prolungamento ci troveremo ad avere
\begin{align}
	(z-z_0) &\to (z-z_0)e^{2\pi i}\\
	u_1(z)  &\to U_1(z)\\
	u_2(z)  &\to U_2(z)
\end{align}
\newpage
Sappiamo però che $u_1(z)$ e $u_2(z)$ sono linearmente indipendenti, e formano quindi una base nello spazio delle soluzioni.

Di conseguenza possiamo scrivere
\begin{align}
	&\double{U_1(z) = a_{11} u_1(z) + a_{12} u_2(z)}{U_2(z) = a_{21} u_1(z) + a_{22} u_2(z)}\firstpassage
	&\left(
	 \begin{array}{c}
		 U_1(z)\\
		 U_2(z)
	 \end{array}\right) = 
	 \begin{pmatrix}
		 a_{11} & a_{12}\\
		 a_{21} & a_{22}
	 \end{pmatrix} 	\left(
     \begin{array}{c}
		 u_1(z)\\
		 u_2(z)
	\end{array}\right)
\end{align}
Si può dimostrare come
\begin{align}
	W(U_1,U_2) = \det(A) \cdot W(u_1, u_1)
\end{align}

La matrice $A$ è non singolare e dipendente dalla polidromia di $u_1(z)$ e $u_2(z)$. Come di consueto gli autovalori si ricavano dall'equazione caratteristica
\begin{align}
	\det(A -\lambda \mathbb{1}) = 0
\end{align}
e possono essere distinti o coincidenti.

A noi interessa trovare una base dove le polidromie in gioco siano le più semplici possibili. 

Questo si traduce nello scegliere una base che renda o diagonale o di Jordan (diagonale a blocchi) la matrice A. Andiamo a studiare i due possibili casi.

\subsubsubsection{Soluzioni per autovalori di $A$ non coincidenti}
Sia il caso $\lambda_1 \neq \lambda_2$, in questo caso possiamo porre
\begin{align}
	A = \begin{pmatrix}
		\lambda_1 & 0 \\
		0 & \lambda_1 
		\end{pmatrix} \to \double{U_1 = \lambda_1 u_1}{U_2 = \lambda_2 u_2}
\end{align}	
Definita la polidromia delle potenze
\begin{align}
	(z-z_0)^{\rho_i} \to (z-z_0)^{\rho_i} e^{2\pi i \rho_i} 
\end{align}
Possiamo estenderla alle soluzioni imponendo
\begin{align}
	\rho_i = \frac{\ln(\lambda_i)}{2\pi i}
\end{align}
questo rende infatti monodrome le funzioni
\begin{align}
	f_i(z) = \frac{u_i(z)}{(z-z_0)^{\rho_i}}
\end{align}
E quindi, ricordando che $z_0$ è una singolarità essenziale, possiamo espandere in esso le $f_i(z)$ in serie di Laurent, ottenendo
\begin{align}
	&f_i(z) = \sum_{k\in \Z} c_k (z-z_0)^k = \frac{u_i(z)}{(z-z_0)^{\rho_i}}\firstpassage
	&\double{u_1(z) = (z-z_0)^{\rho_1}\sum_{k\in \Z} c_k (z-z_0)^k}{u_2(z) = (z-z_0)^{\rho_2}\sum_{k\in \Z} c_k (z-z_0)^k}
\end{align}

\newpage

\subsubsubsection{Soluzioni per autovalori di $A$ coincidenti}
Se invece ci troviamo nel caso $\lambda_1 = \lambda_2$ non è detto che possiamo riscrivere la matrice in forma diagonale. Possiamo però ricondurci ad una matrice inferiore di Hessemberg del tipo
\begin{align}
	A = \begin{pmatrix}
		\lambda_1 & 0        \\
		b_{21}    & \lambda_1
	\end{pmatrix} \to \double{U_1(z) = \lambda_1 u_1(z) \hfill}{U_2(z) = b_{21} u_1(z) + \lambda_1 u_2(z)}
\end{align}
Riprendendo le funzioni definite prima, in questo caso vale ancora la monodromia per
\begin{align}
	f_1(z) =  \frac{u_1(z)}{(z-z_0)^{\rho_1}}
\end{align}
Invece per l'altra possiamo procedere nel seguente modo
\begin{align}
	\frac{U_2(z)}{U_1(z)} = \frac{b_{21} u_1(z) + \lambda_1 u_2(z)}{\lambda_1 u_1(z)} = \frac{b_{21}}{\lambda_1} + \frac{u_2(z)}{u_1(z)} 
\end{align}
Possiamo quindi affermare che il rapporto dei due vettori della base esibisce le stesse proprietà di polidromia del logaritmo (perché? cercare su appunti o scocciare paolo), e quindi possiamo definire la funzione monodroma
\begin{align}
	g(z) = \frac{u_2(z)}{u_1(z)} - \frac{b_{21}}{\lambda_1} \cdot \frac{\ln (z-z_0)}{2\pi i}
\end{align}
che possiamo espandere in serie di Laurent intorno a $z_0$
\begin{align}
	&g(z) = \sum_{k\in \Z}d_k (z-z_0)^k = \frac{u_2(z)}{u_1(z)} -A\ln (z-z_0) \spacer A = \frac{b_{21}}{\lambda_1} \cdot \frac{1}{2\pi i} \firstpassage
	&\frac{u_2(z)}{u_1(z)} = A\ln (z-z_0) + \sum_{k\in \Z}d_k (z-z_0)^k\nextpassage
	&u_2(z) = u_1(z) \cdot \left[ A\ln (z-z_0) + \sum_{k\in \Z}d_k (z-z_0)^k \right] \nextpassage
	&u_2(z) = A\ln(z-z_0)u_1(z) + u_1(z) \sum_{k\in \Z}d_k (z-z_0)^k\nextpassage
	&u_2(z) = A\ln(z-z_0)u_1(z) + (z-z_0)^{\rho_1}\sum_{k\in \Z} c_k (z-z_0)^k \sum_{k\in \Z}d_k (z-z_0)^k\nextpassagecomm{applico il prodotto di Cauchy?}
	&u_2(z) = A\ln(z-z_0)u_1(z) + (z-z_0)^{\rho_1}\sum_{k\in \Z}\sum_{i=0}^{k} c_k d_i (z-z_0)^{k+i}\nextpassage
	&u_2(z) = A\ln(z-z_0)u_1(z) + (z-z_0)^{\rho_1} \sum_{k\in \Z}b_k (z-z_0)^k
\end{align}

\newpage

\subsubsubsection{Punti singolari regolari e teorema di Fuchs}
Qualora $z_0$ sia un punto singolare regolare della funzione, e quindi non essenziale, le serie di Laurent devono troncarsi ad una potenza finita negativa. Dovremo quindi riscrivere, nel caso di autovalori non coincidenti
\begin{align}
	&\double{u_1(z) = (z-z_0)^{\tilde{\rho}_1}\sum_{n = -N}^{\infty} c_n (z-z_0)^n}{u_2(z) = (z-z_0)^{\tilde{\rho}_2}\sum_{n=-M}^{\infty} d_n (z-z_0)^n}
\end{align}
Sfruttando le proprietà di polidromia, sappiamo che possiamo porre
\begin{align}
	\tilde{\rho}_i = \rho_i + n_i \spacer \double{n_1 = N}{n_2 = M} \in \N
\end{align} 
in modo che
\begin{enumerate}
	\item essa sia contenuta tutta nei prefattori esterni alla somma
	\item la coda della serie di Laurent venga eliminata
\end{enumerate}
e possiamo quindi riscrivere
\begin{align}
	&\double{u_1(z) = (z-z_0)^{\rho_1}\sum_{k = 0}^{\infty} c_k (z-z_0)^k}{u_2(z) = (z-z_0)^{\rho_2}\sum_{k=0}^{\infty} d_k (z-z_0)^k} \firstpassage
	&\double{u_1(z) = \sum_{k = 0}^{\infty} c_k (z-z_0)^{k+\rho_1}}{u_2(z) = \sum_{k=0}^{\infty} d_k (z-z_0)^{k + \rho_2}}
\end{align}
Allargando il discorso al caso di autovalori coincidenti, possiamo scrivere
\begin{align}
	&\double{u_1(z) = \sum_{k = 0}^{\infty} c_k (z-z_0)^{k+\rho_1} \hfill }{u_2(z) = A \ln(z-z_0) \sum_{k = 0}^{\infty} c_k (z-z_0)^{k+\rho_1} + \sum_{k=0}^{\infty}b_k (z-z_0)^{k+\rho_1}} \firstpassage
	&\double{u_1(z) = \sum_{k = 0}^{\infty} c_k (z-z_0)^{k+\rho_1} \hfill }{u_2(z) = A \ln(z-z_0) \sum_{k = 0}^{\infty} c_k (z-z_0)^{k+\rho_1} + \sum_{k=0}^{\infty}b_k (z-z_0)^{k+\rho_1}}
\end{align}
Se andiamo a riprendere la forma generale della EDO
\begin{align}
	\ddot{u}(z) + p(z)\dot{u}(z) + q(z)u(z) = 0
\end{align}
a partire dalla forma delle soluzioni possiamo ricavare la forma dei coefficienti:
\begin{enumerate}
	\item Possiamo ricavare $q(z)$ direttamente dalla EDO
	\begin{align}
		q(z) = - \frac{\ddot{u}_1(z)}{u_1(z)} - p(z)\frac{\dot{u}_1(z)}{u_1(z)}
	\end{align}
	\item Invece $p(z)$ dall'equazione di Liouville che abbiamo introdotto in precedenza
	\begin{align}
		&\dot{W} + p(z) W = 0 \firstpassage
		& p(z) = - \frac{\dot{W}}{W} \nextpassage
		& p(z) = - \frac{u_1(z) \ddot{u}_2(z) - u_2(z) \ddot{u}_1(z)}{u_1(z) \dot{u}_2(z) - u_2(z) \dot{u}_1(z)} \label{pidizeta}
	\end{align}
\end{enumerate}
Detto questo, prendiamo in esame il caso degli autovalori non coincidenti.

Riscrivendo per comodità le soluzioni come
\begin{align}
	&\double{u_1(z) = (z-z_0)^{\rho_1} R_1(z)}{u_2(z) = (z-z_0)^{\rho_2}R_2(z) } \spacer \double{R_1(z) = \sum_{k = 0}^{\infty} c_k (z-z_0)^{k} }{ R_2(z) = \sum_{k = 0}^{\infty} d_k (z-z_0)^{k}}
\end{align}
E andiamo a calcolarne le derivate prime e seconde
\begin{align}
	\dot{u}_i(z) &= \rho_i(z-z_0)^{\rho_i-1} R_i(z) + (z-z_0)^{\rho_i} \dot{R}_i(z)\firstpassage
	\dot{u}_i(z) &= (z-z_0)^{\rho_i-1}\cdot [\rho_i R_i(z) +(z-z_0)\dot{R}_i(z) ] = \nonumber\\
				 &= (z-z_0)^{\rho_i-1}\cdot \tilde{R}_i(z) \firstpassagecomm{ripetendo il procedimento}
	\ddot{u}_i(z)&= (z-z_0)^{\rho_i-2}\cdot \tilde{\tilde{R}}_i(z)
\end{align}
Siamo così andati a definire le funzioni $R_i(z)$, $\tilde{R}_i(z)$ $\tilde{\tilde{R}}_i(z)$ che in $z_0$ sono
\begin{enumerate}
	\item analitiche 
	\item mai nulle
\end{enumerate}

Se quindi ad esempio prendiamo la \ref{pidizeta} possiamo scrivere
\begin{align}
	p(z) &= \frac{(z-z_0)^{\rho_1 + \rho_2 -2} (R_1(z)\tilde{\tilde{R}}_2(z) - R_2(z)\tilde{\tilde{R}}_1(z))}{(z-z_0)^{\rho_1 + \rho_2 -1} (R_1(z)\tilde{R}_2(z) - R_2(z)\tilde{R}_1(z))} = \nonumber \\
	 	 &=(z-z_0)^{-1} \frac{R_1(z)\tilde{\tilde{R}}_2(z) - R_2(z)\tilde{\tilde{R}}_1(z)}{R_1(z)\tilde{R}_2(z) - R_2(z)\tilde{R}_1(z)} \nextpassage
	p(z) &= \frac{P(z)}{z-z_0} \spacer P(z) = \frac{R_1(z)\tilde{\tilde{R}}_2(z) - R_2(z)\tilde{\tilde{R}}_1(z)}{R_1(z)\tilde{R}_2(z) - R_2(z)\tilde{R}_1(z)}
\end{align}
E in modo analogo troviamo
\begin{align}
	q(z) &= -\frac{(z-z_0)^{\rho_1-2}\cdot \tilde{\tilde{R}}_1(z)}{(z-z_0)^{\rho_1} R_1(z)} - p(z)\frac{(z-z_0)^{\rho_1-1}\cdot \tilde{R}_1(z)}{(z-z_0)^{\rho_1} R_1(z)} = \nonumber \\
		 &= -\frac{1}{(z-z_0)^{2}} \cdot \frac{\tilde{\tilde{R}}_1(z)}{R_1(z)} - \frac{P(z)}{z-z_0} \cdot \frac{1}{z-z_0} \cdot \frac{\tilde{R}_1(z)}{R_1(z)} = \nonumber \\
		 &=	-\frac{1}{(z-z_0)^{2}} \cdot \left[ \frac{\tilde{\tilde{R}}_1(z)}{R_1(z)} - P(z)\cdot \frac{\tilde{R}_1(z)}{R_1(z)} \right] = \nextpassage
	q(z) &= \frac{Q(z)}{(z-z_0)^2} \spacer Q(z) = P(z)\cdot \frac{\tilde{R}_1(z)}{R_1(z)} - \frac{\tilde{\tilde{R}}_1(z)}{R_1(z)} 
\end{align}

\newpage

Per costruzione, sia $P(z)$ che $Q(z)$ sono analitiche, e quindi
\begin{enumerate}
	\item $p(z)$ ha al massimo un polo singolo
	\item $q(z)$ ha al massimo un polo doppio
\end{enumerate}

Possiamo quindi enunciare il \textbf{Teorema di Fuch:} \textit{presa una EDO nella forma
\begin{align}
	\ddot{u}(z) + p(z)\dot{u}(z) + q(z)u(z) = 0
\end{align}
un punto $z_0$ si dice \textbf{singolare regolare} o \textbf{fuchsiano} se e solo se valgono le seguenti condizioni
\begin{align}
	\double{\limit{z}{z_0} p(z)(z-z_0) = p_0}{\limit{z}{z_0} q(z)(z-z_0)^2 = q_0} \spacer p_0, q_0 \; \text{costanti}
\end{align}
}
Questo significa che le EDO oggetto di studio di questo corso, in presenza di punti fuchsiani assumono la forma
\begin{align}
	\ddot{u}(z) + \frac{P(z-z_0)}{z-z_0} \cdot \dot{u}(z) + \frac{Q(z-z_0)}{(z-z_0)^2} \cdot u(z) = 0
\end{align}
Dobbiamo dimostrare che
\begin{enumerate}
	\item i coefficienti siano determinati a partire dalla EDO 
	\item le serie corrispondenti siano convergenti
\end{enumerate}

\subsubsubsubsection{Verifica dei coefficienti}
Poniamo $\xi = z-z_0$ ed espandiamo secondo Taylor in $\xi = 0$ le due funzioni analitiche
\begin{align}
	P(\xi) = \sum_{k = 0}^{\infty} p_k \xi^k \spacer Q(\xi) = \sum_{k = 0}^{\infty} q_k \xi^k
\end{align}
Prendiamo ora un Ansatz generico
\begin{align}
	&u(z) = \xi^\rho \sum_{k=0}^{\infty} c_k \xi^k \firstpassage
	&\double{\dot{u}(z) = \sum_{k=0}^{\infty} c_k (k+\rho)\xi^{k+\rho-1}\hfill}{\ddot{u}(z) = \sum_{k=0}^{\infty} c_k (k+\rho)(k+\rho-1)\xi^{k+\rho-2}}
\end{align}
Arriviamo così a riscrivere
\begin{align}
	&\ddot{u}(\xi) + \frac{P(\xi)}{\xi} \cdot \dot{u}(\xi) + \frac{Q(\xi)}{\xi^2} \cdot u(z) = 0 \firstpassage
	&\sum_{k=0}^{\infty} c_k (k+\rho)(k+\rho-1)\xi^{k+\rho-2} + \sum_{n = 0}^{\infty} p_n \xi^{n-1} \sum_{k=0}^{\infty} c_k (k+\rho)\xi^{k+\rho-1} + \sum_{n = 0}^{\infty} q_n \xi^{n-2}\sum_{k=0}^{\infty} c_k \xi^{k+\rho} = 0 \nextpassage
	&\sum_{k=0}^{\infty} c_k (k+\rho)(k+\rho-1)\xi^{k+\rho-2} + \sum_{n = 0}^{\infty} p_n \xi^n \sum_{k = 0}^{\infty} c_k (k+\rho)\xi^{k+\rho -2} + \sum_{n = 0}^{\infty} q_n \xi^n \sum_{k = 0}^{\infty} c_k \xi^{k+\rho -2} = 0
\end{align}
\newpage
Come anche in precedenza, ci appoggiamo al prodotto di Cauchy e definendo l'inidce $l = n+k$ otteniamo
\begin{align}
	&\sum_{n = 0}^{\infty} p_n \xi^n \sum_{k = 0}^{\infty} c_k (k+\rho)\xi^{k+\rho -2} = \sum_{l = 0}^{\infty} \sum_{n = 0}^{l} p_n c_k (l-n+\rho)\xi^{l+\rho -2}\\
	&\sum_{n = 0}^{\infty} q_n \xi^n \sum_{k = 0}^{\infty} c_k \xi^{k+\rho -2} = \sum_{n = 0}^{\infty}\sum_{n = 0}^{l} q_n c_l \xi^{l+\rho -2}
\end{align}
E sfruttando il fatto che $l$ e $k$ sono indici muti, possiamo raggruppare tutto sotto l'unica sommatoria
\begin{align}
	\sum_{k = 0}^\infty \left[ c_k (k+\rho)(k+\rho-1) + \sum_{n = 0}^{l} \left( p_n c_{k-n} (k-n+\rho) + q_n c_{k-n} \right) \right] \xi^{l+\rho -2} = 0
\end{align}
Da questo ricaviamo l'espressione ricorsiva dei coefficienti, data da
\begin{align}
	c_k (k+\rho)(k+\rho-1) + \sum_{n = 0}^{l} \left( p_n c_{k-n} (k-n+\rho) + q_n c_{k-n} \right) = 0 \label{ricors}
\end{align}
Ci si presentano due casi
\begin{enumerate}
	\item $c_0 = 0$
	
	In questo caso otteniamo la soluzione identicamente nulla
	\item $c_0 \neq 0$
	
	In questo caso per $k=0$ otteniamo la cosidetta \textbf{equazione indiciale}, indipendente dai $c_k$
	\begin{align}
		\rho^2 + (p_0 -1)\rho + q_0 = 0
	\end{align}
	grazie alla quale possiamo ricavare gli indici $\rho_1$ e $\rho_2$ dell'equazione. 
	
	Definiamo ora
	\begin{align}
		\double{F_0(\rho) = \rho(\rho -1) + \rho p_0 + q_0}{F_n(\rho) = \rho p_n + q_n \hfill}
	\end{align}
	questo ci permette di riscrivere la \ref{ricors} come
	\begin{align}
		&c_0 F_0(\rho) = 0\\
		&c_1 F_0(\rho + 1) + c_0 F_1(\rho) = 0\\
		&c_2 F_0(\rho + 2) + c_1 F_0(\rho + 1) + c_0 F_2(\rho) = 0\\
		&\dots \nonumber\\
		&c_n F_0(\rho + n) + c_{n-1} F_0(\rho + n-1) + \dots + c_1 F_n(\rho + 1) + c_0 F_1(\rho) = 0	 	
	\end{align}
 	Dobbiamo ora distinguere due casi
 	\begin{enumerate}
 		\item $\rho_i \neq \rho_j + n \spacer i,j,n \in N$
 		
 		In questo caso $F_0(\rho_1 + n)$ è sempre non nullo in quanto l'argomento non può essere soluzione dell'equazione indiciale. 
 		
 		Quindi il sistema determina completamente \begin{enumerate}
 			\item i coefficienti $c_k$ di $u_1$ con indice $\rho_1$
 			\item i coefficienti $d_k$ di $u_2$ con indice $\rho_2$
 		\end{enumerate}
 		\newpage
 		\item $\rho_i =    \rho_j + n \spacer i,j,n \in N$
 		In questo caso bisogna cambiare Ansatz per la soluzione, altrimenti da $\rho_2$ in poi i termini vanno ad annullarsi in cascata.
 		
 		Dobbiamo quindi passapre per il metodo del Wronskiano passando per l'Ansatz
 		\begin{align}
 			u_2(z) = A\ln(z)u_1(z) + z^{\rho_2} \sum_{k=0}^{\infty} b_k z^k
 		\end{align}
 		e ricavando le relazioni ricorsive per i coefficienti $b_k$ e A, quest'ultimo può essere anche nullo. 		
 	\end{enumerate}
\end{enumerate}


\subsubsubsubsection{Convergenza delle serie}
Consideriamo un intorno di $z_0$ di raggio $r$ compreso nel dominio di analiticità delle $P(\xi)$ e $Q(\xi)$.
Abbiamo visto come possono essere espanse secondo Taylor
\begin{align}
	P(\xi) = \sum_{k=0}^{\infty} p_k \xi^k \\
	Q(\xi) = \sum_{k=0}^{\infty} q_k \xi^k
\end{align}
Possiamo utilizzare la \textbf{formula di Cauchy} per riscrivere i coefficienti come
\begin{align}
	p_k = \frac{1}{k!} \frac{d^k P(\xi)}{d\xi^k} = \frac{1}{2\pi i} \oint d\xi \; \frac{P(\xi)}{\xi^{k+1}}\\
	q_k = \frac{1}{k!} \frac{d^k Q(\xi)}{d\xi^k} = \frac{1}{2\pi i} \oint d\xi \; \frac{Q(\xi)}{\xi^{k+1}}
\end{align}
Per il \textbf{teorema del massimo modulo} sappiamo che
\begin{align}
	|p_k| &\leq \frac{\max |P|}{r^k} = M_p\\
	|q_k| &\leq \frac{\max |Q|}{r^k} = M_q\\
	|F_k(\rho)| =  | \rho_1 p_k + q_k| &\leq \frac{\max |\rho_1 P + Q|}{r^k} = M_f
\end{align}
Definendo 
\begin{align}
	M = \max (M_p, M_q, M_f)
\end{align}
sappiamo per certo che per $r<r_0$ avremo
\begin{align}
	|p_k|,|q_k|,|F_k(\rho_1)| \leq \frac{M}{r^k}
\end{align}
e possiamo quindi enunciare il \textbf{teorema:} \textit{data una serie i cui coefficienti sono definibili tramite la \ref{ricors}
	\begin{align}
		\sum_{k=0}^{\infty} c_k \xi^k
\end{align}
i coefficienti soddisferanno la relazione
\begin{align}
	|c_k| \leq c_0 \left(\frac{M}{r}\right)^k
\end{align}
e di conseguenza la serie avrà raggio di convergenza finito
\begin{align}
	R = \frac{r}{M}
\end{align}
}

Iniziamo notando come per ipotesi
\begin{align}
	&\rho_1 \neq \rho_2 +n \firstpassage
	& \double{F_0(\rho) = (\rho - \rho_1)(\rho - \rho_2)\hfill }{F_0(\rho_1 +k) = (\rho_1 + k- \rho_1)(\rho_1 + k- \rho_2) = k (k+s)} \spacer s = \rho_1 - \rho_2
\end{align}
E dimostriamo il teorema per induzione
\begin{enumerate}
	\item $k=1$
	\begin{align}
		|c_1| = |c_0| \frac{|F_1(\rho_1)|}{|F_0(\rho_1 + 1)} \leq \frac{M}{r} \frac{|c_0|}{|s+1|} \leq |c_0| \frac{M}{r}
	\end{align}
	\item Supponiamo che il teorema sia valido fino a $n-1$
	\item $k=n$
	\begin{align}
		&c_n F_0(\rho + n) + \sum_{l=1}^{n} c_{n-l} F_l (\rho + n -l ) = 0\firstpassage
		&c_n  = - \frac{\sum_{l=1}^{n} c_{n-l} F_l (\rho + n -l )}{F_0(\rho + n)}\nextpassage
		&|c_n|  \leq - \frac{\sum_{l=1}^{n} |c_{n-l}| |F_l (\rho + n -l )|}{|n(n+s)|}
	\end{align}
	Siccome abbiamo che
	\begin{align}
		|F_l (n-l + \rho_l)| &= |(n-l+\rho_1)p_i + q_i| =\nonumber \\
							 &= |(n-l)p_l + (\rho_1p_i + q_i)| =\nonumber \\
							 &= |(n-l)p_l + F_1(\rho_1)| \leq \nonumber\\
							 &\leq (n-l)|p_l| + |F_1(\rho_1)| \leq \nonumber\\
							 &\leq (n-l) \frac{M^l}{r^l} + \frac{M^l}{r^l}\firstpassage
		|F_l (n-l + \rho_l)| &\leq (n-l+1) \cdot \frac{M^l}{r^l}	  
	\end{align}

	Utilizzando le relazioni
	\begin{align}
		&\sum_{l=1}^{n} (n-l+1) = (n+1)\sum_{l=1}^{n} 1 - \sum_{l=1}^{n}l = (n+l)n - \frac{n(n+1)}{2} = \frac{n(n+1)}{2}\\
		&n(n+s) = n^2 + ns > n^2
	\end{align}	
	Otteniamo così la tesi
	\begin{align}
		|c_n| &\leq -\sum_{l=1}^{n} \frac{1}{n(n+s)} \left[|c_{n-l}| \cdot |F_l (\rho + n -l )| \right] \leq \nonumber \\
			  &\leq - \frac{1}{n(n+s)} \cdot \sum_{l=1}^{n}  \left[|c_{n-l}| \cdot (n-l+1) \cdot \frac{M^l}{r^l}	 \right] \leq \nonumber \\
			  &\leq - \frac{1}{n(n+s)} \cdot \sum_{l=1}^{n}  \left[ |c_0|\frac{M^{n-\cancel{l}}}{r^{n-\cancel{l}}} \cdot (n-l+1) \cdot \frac{\cancel{M^l}}{\cancel{r^l}}	 \right] \firstpassage
		|c_n| &\leq - \frac{|c_0|}{n(n+s)}\cdot \frac{M^n}{r^n} \cdot \sum_{l=1}^{n}(n-l+1) \firstpassage
		|c_n| &\leq - \frac{|c_0|}{\cancel{n}(n+s)}\cdot \frac{M^n}{r^n}\cdot \frac{\cancel{n}(n+1)}{2}\firstpassage
		|c_n| &\leq - \frac{|c_0|}{2}\cdot \frac{n+1}{n+s}\cdot \frac{M^n}{r^n} \leq \nonumber\\
			  &\leq - \frac{|c_0|}{2}\cdot \frac{M^n}{r^n} \leq \nonumber\\
			  &\leq - |c_0|\cdot \frac{M^n}{r^n}
	\end{align}
\end{enumerate}

\textbf{Nota bene:} la dimostrazione vale anche per $\rho_i = \rho_j + n$, rispetto alla parte non contenente il logaritmo.

\subsubsubsection{Punti singolari non regolari}
Qualora il punto singolare non sia regolare la situazione si complica. Quello che succede di solito è che riesce a ricavare al massimo uno solo degli indici e quindi una soluzione sola, e non sempre si riesce a dimostrare la convergenza della serie ottenuta, che potrebbe quindi essere asintotica.

La situazione migliora in presenza del fenomeno di \textbf{confluenza}, ovvero quando la funzione oggetto di studio che presenta punti irreogolari è il limite di funzioni che invece presentano punti regolari. Vederemo un esempio più avanti.

\newpage

\subsubsection{Punto all'infinito}

Per studiare il comportamento della funzione intorno al punto all'infinito dobbiamo applicare il solito cambio di variaibli
\begin{align}
	\eta = \frac{1}{z}
\end{align}
Che matta il punto nell'origine, e che trasforma le derivate nel seguente modo
\begin{align}
	\frac{d}{dz} &= \frac{d}{d\eta} \frac{d\eta}{dz} = -\frac{1}{z^2} \frac{d}{d\eta} = -\eta^2 \frac{d}{d\eta}\\
	\frac{d^2}{dz^2} &= -\eta^2 \frac{d}{d\eta} \left(-\eta^2 \frac{d}{d\eta}\right) = \eta^4 \frac{d^2}{d\eta^2} + 2\eta^3 \frac{d}{d\eta}
\end{align}
E quindi la EDO si trasforma nel seguente modo
\begin{align}
	&\frac{d^2}{dz^2} u(z) + p(z)\frac{d}{dz} u(z) + q(z)u(z) = 0 \firstpassage
	&\left[\eta^4 \frac{d^2}{d\eta^2} + 2\eta^3 \frac{d}{d\eta}\right] u\left(\frac{1}{\eta}\right) + p\left(\frac{1}{\eta}\right)\left[-\eta^2 \frac{d}{d\eta}\right]u\left(\frac{1}{\eta}\right) + q\left(\frac{1}{\eta}\right)u\left(\frac{1}{\eta}\right) = 0 \nextpassage
	&\eta^4 \frac{d^2}{d\eta^2}u\left(\frac{1}{\eta}\right) + \left[2\eta^3 -\eta^2p\left(\frac{1}{\eta}\right) \right] \frac{d}{d\eta}u\left(\frac{1}{\eta}\right) + q\left(\frac{1}{\eta}\right)u\left(\frac{1}{\eta}\right) = 0 \nextpassagecomm{riscrivo in forma normale}
	&\frac{d^2}{d\eta^2}u\left(\frac{1}{\eta}\right) + \left[\frac{2}{\eta} -\frac{1}{\eta^2}p\left(\frac{1}{\eta}\right) \right] \frac{d}{d\eta}u\left(\frac{1}{\eta}\right) + \frac{1}{\eta^4} q\left(\frac{1}{\eta}\right)u\left(\frac{1}{\eta}\right) = 0 \label{pinf}
\end{align}
Definendo
\begin{align}
	\tilde{p}(\eta) &= \frac{2}{\eta} -\frac{1}{\eta^2}p\left(\frac{1}{\eta}\right) \\
	\tilde{q}(\eta) &= \frac{1}{\eta^4} q\left(\frac{1}{\eta}\right)
\end{align}
Possiamo riscrivere la \ref{pinf} nella forma a noi nota (ricontrollare la dipendenza di u, scocciare qualcuno)
\begin{align}
	\ddot{u}(\eta) + \tilde{p}(\eta) \dot{u}(\eta) + \tilde{q}(\eta) u(\eta) = 0
\end{align}
Ne segue che il punto all'infinito sarà regolare se
\begin{align}
	&\limit{\eta}{0} \tilde{p}(\eta) = cost.\\
	&\limit{\eta}{0} \tilde{q}(\eta) = cost.
\end{align}
Dalla prima ricaviamo che
\begin{align}
	&\limit{\eta}{0}\left[\frac{2}{\eta} -\frac{1}{\eta^2}p\left(\frac{1}{\eta}\right)\right] = cost \firstpassagecomm{espaniamo $p(\eta^{-1})$ secondo taylor in $\eta = 0$}
	&p\left(\frac{1}{\eta}\right) = 2\eta + \sum_{k=2}^{\infty} c_k \eta^k \nextpassage
	&\limit{\eta}{0}\frac{1}{\eta}p\left(\frac{1}{\eta}\right) = 2 \nextpassage
	&\limit{z}{\infty}zp\left(z\right) = 2	
\end{align}
Invece dalla seconda
\begin{align}
	&\limit{\eta}{0} \left[\frac{1}{\eta^4} q\left(\frac{1}{\eta}\right)\right] = cost.\firstpassagecomm{espaniamo $q(\eta^{-1})$ secondo taylor in $\eta = 0$}
	&q \left(\frac{1}{\eta}\right) = \sum_{k=4}^{\infty} c_k \eta^k = c_4 \eta^4 +  \sum_{k=5}^{\infty} c_k \eta^k \nextpassage
	&\limit{\eta}{0} \left[\frac{1}{\eta^4} q\left(\frac{1}{\eta}\right)\right] = cost. \nextpassage
	&\limit{z}{\infty} z^4 q\left(z\right) = cost.
\end{align}
Riasumendo, il punto all'infinito è \textbf{regolare} se
\begin{align}
	\double{\limit{z}{\infty}zp\left(z\right) &= 2\hfill}{\limit{z}{\infty} z^4 q\left(z\right) &= cost.}
\end{align}
Allo stesso modo si può arrivare a dire che il punto all'infinito è una \textbf{singolarità regolare} (ovvero è un \textbf{punto fuchsiano}) se
\begin{align}
	\double{\limit{z}{\infty}zp\left(z\right) &= cost. \hfill}{\limit{z}{\infty} z^2 q\left(z\right) &= cost.}
\end{align}
Se non rispetta nessuna di queste, è una \textbf{singolarità irregolare}.

All'infinito la serie di Taylor diventa una serie di Laurent, e le soluzioni possono scriversi come
\begin{align}
	&\double{u_1(z) = z^{-\rho_1} \sum_{n=0} c_n z_{-n}}{u_2(z) = z^{-\rho_2} \sum_{n=0} d_n z_{-n}} &\spacer &\rho_1 \neq \rho_2 + n\\
	&\double{u_1(z) = z^{-\rho_1} \sum_{n=0} c_n z_{-n}\hfill}{u_2(z) = z^{-\rho_2} \sum_{n=0} b_n z_{-n} + \alpha\ln(z) u_1(z) } &\spacer &\rho_1 = \rho_2 + n
\end{align}

\newpage

\subsubsection{Equazioni fuchsiane}

Se classificare le equazioni fuchsiane è improponibile, non lo è definirne la forma generale.

Presa una equazione
\begin{align}
	\ddot{u}(z) + p(z)\dot{u}(z) + q(z)u(z) = 0
\end{align}
con $N$ punti fuchsiani $\xi_i$ e il punto all'infinito anch'esso singolare regolare, possiamo definirne il P-sibolo come
\begin{align}
	P = \begin{pmatrix}
		\xi_1 & \xi_2 & \dots & \xi_N & \infty\\
		\downarrow & \downarrow & \downarrow & \downarrow & \downarrow\\
		\alpha_1 & \alpha_2 & \dots & \alpha_N & \mu_1\\
		\beta_1 & \beta_2 & \dots & \beta_N & \mu_2
	\end{pmatrix} \begin{array}{c}
	\leftarrow \text{punti singolari regolari} \hfill \\
	\\
	\leftarrow \text{primi indici relativi a ciascun punto}\hfill\\
	\leftarrow \text{secondi indici relativi a ciascun punto}
\end{array}
\end{align}
Devono quindi valere in contemporanea sia le condizioni di fuchsianità per gli $\xi_i$, ovvero
\begin{align}
	\double{\limit{z}{\xi_i} (z-\xi_i)p(z) = p_0}{\limit{z}{\xi_i} (z-\xi_i)^2 q(z)= q_0} \spacer p_0, q_0 \; \text{costanti} \label{omegakek}
\end{align}
che quelle per il punto all'infinito
\begin{align}
	\double{\limit{z}{\infty}zp\left(z\right) &= cost. \hfill}{\limit{z}{\infty} z^2 q\left(z\right) &= cost.}
\end{align}
Da queste ultime notiamo come, parametrizzando la funzione coefficiente $p(z)$ come 
\begin{align}
	p(z) = \frac{P(z)}{\prod_{i=1}^{N}(z-\xi_i)}
\end{align}
segue che $P(z)$ debba essere un polinomio di grado $N-1$ in $z$, e possiamo riscriverlo tramite il metodo delle frazioni parziali come
\begin{align}
	p(z) = \sum_{i=1}^{N} \frac{A_i}{z-\xi_i}
\end{align}
In modo analogo possiamo parametrizzare $q(z)$, tenendo a mente che il limite all'infinito deve avere un comportamento analogo a $z^{2N}$. Avremo quindi un poliniomio di grado $2N-2$ e possiamo quindi usare la rappresentazione
\begin{align}
	q(z) = \sum_{i=1}^{N} \frac{B_i}{(z-\xi_i)^2} +  \frac{\sum_{j=0}^{N-2} c_j z^{N-2-j} }{\prod_{i=1}^{N}(z-\xi_i)}
\end{align}
Con queste parametrizzazioni le condizioni \ref{omegakek} si riscrivono nella forma
\begin{align}
	\double{\limit{z}{\xi_i} (z-\xi_i)p(z) = A_i}{\limit{z}{\xi_i} (z-\xi_i)^2 q(z)= B_i}
\end{align}
L'equazione indiciale diventa
\begin{align}
	\rho^2 + (A_i -1)\rho + B_i = 0
\end{align}
In precedenza avevamo trovato
\begin{align}
	&\rho^2 + (p_0 -1)\rho + q_0 = 0 \firstpassage
	&(\rho- \alpha_i)(\rho- \beta_i) = 0 \nextpassage
	&\rho^2 - (\alpha_i + \beta_i) \rho +\alpha_i\beta_i = 0
\end{align}
Confrontando le due otteniamo
\begin{align}
	\double{ A_i =  1 - \alpha_i - \beta_i}{B_i = \alpha_i\beta_i \hfill}
\end{align}
Possiamo quindi riscrivere la nostra EDO come
\begin{align}
	&\ddot{u}(z) + p(z)\dot{u}(z) + q(z)u(z) = 0\firstpassage
	&\ddot{u}(z) + \left[ \sum_{i=1}^{N} \frac{A_i}{z-\xi_i} \right] \dot{u}(z) + \left[ \sum_{i=1}^{N} \frac{B_i}{(z-\xi_i)^2} +  \frac{\sum_{j=0}^{N-2} c_j z^{N-2-j} }{\prod_{i=1}^{N}(z-\xi_i)} \right] u(z) = 0\nextpassage
	&\ddot{u}(z) + \left[ \sum_{i=1}^{N} \frac{1 - \alpha_i - \beta_i}{z-\xi_i} \right] \dot{u}(z) + \left[ \sum_{i=1}^{N} \frac{\alpha_i\beta_i}{(z-\xi_i)^2} +  \frac{\sum_{j=0}^{N-2} c_j z^{N-2-j} }{\prod_{i=1}^{N}(z-\xi_i)} \right] u(z) = 0
\end{align}

Andiamo ora ad analizzare l'equazione indiciale intorno al punto all'infitito. Appoggiandoci all'Ansatz $u(z) = z^{-\mu}$ e tenendo solo 'andamento dominante otteniamo
\begin{align}
	\mu^2 + \left[ \sum_{i=1}^{N} (\alpha_i + \beta_i) + (1-N) \right]\mu + \left[ c_0 + \sum_{i=1}^{N}\alpha_i \beta_i \right] = 0
\end{align}

Noi ci concentreremo su i casi $N=1,2,3$. Dobbiamo però prima fare una deviazione e definire uno strumento fondamentale per lo studio delle soluzioni.

\subsubsubsection{Trasformazioni conformi}

Spesso fa comodo poter spostare le singolarità della EDO oggetto di studio. Uno strumento molto comodo è quello delle \textbf{trasformazioni conformi}, ovvero che mantengono gli angoli ma non le distanze. In particolare ci tornano molto utili le \textbf{trasformazioni lineari fratte} (o \textbf{di Möbius})
\begin{align}
	&z \to F(z) = \frac{az +b}{cz+d} \spacer \double{ad-bc= 1}{a,b,c,d\in \C} 
\end{align}
di cui un esempio è il \textbf{birapporto}, definito come
\begin{align}
	F = \frac{(z -z_1)(z_2 -z_3)}{(z -z_3)(z_2 -z_1)}
\end{align}
Queste trasformazioni formano un gruppo non-Abeliano isomorfo a $PSL(2,\C)$

Per l'isomorfismo associamo alla trasformazione fratta la matrice
\begin{align}
	\hat{F} = \begin{pmatrix}
		a & b \\
		c & d
	\end{pmatrix}
\end{align}
Le \textbf{trasformazioni di Möbius} godono delle seguenti proprietà:
\begin{enumerate}
	\item Composizione di due trasformazioni successive
	\begin{align}
		&z \to (F_2 \circ F_1)(z) = \frac{(a_1 a_2 + c_1b_2)z + (b_1a_2 + d_1 b_2)}{(c_2 a_1 + d_2c_1)z + (b_1c_2 + d_1 d_2)} = F_3(z) \firstpassage
		&\hat{F}_3 == \begin{pmatrix}
			a_1 a_2 + c_1b_2 & b_1a_2 + d_1 b_2 \\
			c_2 a_1 + d_2c_1 & b_1c_2 + d_1 d_2
		\end{pmatrix} = \begin{pmatrix}
			a_1 & b_1 \\
			c_1 & d_1
		\end{pmatrix}\begin{pmatrix}
			a_2 & b_2 \\
			c_2 & d_2
		\end{pmatrix} = \hat{F}_1 \hat{F}_2
	\end{align}
	\item LA trasformazione identica cosrrisponde alla matrice identià
	\begin{align}
		&z \to I(z) = z = \frac{1 \cdot z + 0 }{0 \cdot z + 1} \firstpassage
		&\hat{I} = \begin{pmatrix}
			1 & 0 \\
			0 & 1
		\end{pmatrix}
	\end{align}
	\item la trasfomrazione inversa si ha solo per $cz + d \neq 0$ e sarà data da
	\begin{align}
		&w = \frac{az + b}{cz +d} \to z =\frac{b - dw}{cw - a} \spacer \double{cz + d \neq 0}{cw -a \neq 0}\firstpassage
		&\hat{F}^{-1}  = \begin{pmatrix}
			-d & b \\
			c & -a
		\end{pmatrix}
	\end{align}
	\item Essendo le trasformazioni apparententi al gruppo $PSL(2,\C)$, avremo che 
	\begin{align}
		&\forall \lambda\neq 0 \in \C \taleche (e,f,g,h) = (\lambda a,\lambda a,\lambda b,\lambda c,\lambda d)\firstpassage
		&G=\frac{ex + f}{gx + h} = \frac{\lambda ax + \lambda b}{\lambda cx + \lambda d } =\frac{ax + b}{cx + d} = F
	\end{align}
 	Abbiamo quindi che gruppi di parametri multipli per un numero, le trasformazioni sono uguali.
\end{enumerate}

Notiamo dalla definizione come le trasformazioni lineari fratte abbiano una singolarità in
\begin{align}
	z = -\frac{d}{c}
\end{align}
E allo stesso modo le inverse in
\begin{align}
	w = +\frac{a}{c}
\end{align}
Possiamo quindi estendere le trasformazioni al punto all'infinito imponendo che
\begin{align}
	\double{F \left( -\frac{d}{c} \right) &=+\infty \hfill}{F(+\infty) &= \frac{a}{c} \hfill}
\end{align}
Ma con queste assunzioni in gioco, le trasformate diventano l'unica mappa biunivoca per la sfera di Riemann in se stessa. 

Possiamo quindi enunciare il seguente \textbf{teorema:} \textit{siano 
\begin{align}
	z_1,z_2,z_3\in \C
\end{align}
allora esiste una sola trasformazione lineare fratta tale che}
\begin{align}
	\triple{F(z_1) = w_1}{F(z_2) = w_2}{F(z_3) = w_3}
\end{align}

DIMOSTRAZIONE DA FARE

\newpage

\subsubsubsection{Equazioni con 1 punto fuchsiano}
Partendo dall'espressione
\begin{align}
	\ddot{u}(z) + \frac{P(z)}{z-z_0}\cdot \dot{u}(z) + \frac{Q(z)}{(z-z_0)^2} \cdot u(z)=0 
\end{align}
Consideriamo equazioni che per semplicità abbiano
\begin{enumerate}
	\item Un punto singolare regolare finito
	\item Il punto all'infinito regolare, il che ci impone
	\begin{align}
		\double{\limit{z}{\infty}zp\left(z\right) &= 2 \hfill}{\limit{z}{\infty} z^4 q\left(z\right) &= cost.} &\implies \double{P(z) = 2}{Q(z)= 0}   
	\end{align}
\end{enumerate}
In questo caso avremo EDO nella forma
\begin{align}
	\ddot{u}(z) + \frac{2}{z-z_0}\cdot \dot{u}(z) =0 
\end{align}
Per la soluzione abbiamo due strade
\begin{enumerate}
	\item procedere per separazione di variabili
	\item studiare l'equazione indiciale risultante dall'Ansatz $u= (z-z_0)^\rho + \xi^\rho$ dove otteniamo
	\begin{align}
		&\rho\cdot(\rho-1)\xi^{\rho -2} + \frac{2\rho}{\xi}\xi^{\rho-1} =0\firstpassage
		&\rho^2 + \rho =0 \to \rho(\rho +1) =0 \nextpassage
		&\rho_1 = 0 \spacer \rho_2 = -1\firstpassage
		&u(z) = c_1 + \frac{c_2}{z-z_0} 
	\end{align}
\end{enumerate}
\newpage
\subsubsubsection{Equazioni con 2 punti fuchsiani}
Consideriamo ora il caso con due punti singolari regolari al finito e il punto all'infinito regolare. Analogamente a prima avremo
\begin{align}
	\ddot{u}(z) + \left[ \sum_{i=1}^{2} \frac{A_i}{z-z_i} \right] \dot{u}(z) + \left[ \frac{Q(z)}{\prod_{i=1}^2 (z-z_i)^2} \right] u(z) =0
\end{align}
La regolarità del punto all'infinito impone
	\begin{align}
	\double{\limit{z}{\infty}zp\left(z\right) &= 2 \hfill}{\limit{z}{\infty} z^4 q\left(z\right) &= cost.} &\implies \double{A_1 + A_2 = 2\hfill}{Q(z)= cost.  = B}   \label{2punti}
\end{align}
Possiamo quindi scrivere le equazioni indiciali in forma generale come
\begin{align}
	&\rho^2 + (A_1 -1)\rho + \frac{B}{(z_1-z_2)^2} = 0 = (\rho - \alpha_1)(\rho - \beta_1) \firstpassage
	& \double{A_1 -1 &= - \alpha_1-\beta_1}{\frac{B}{(z_1-z_2)^2} &= \alpha_1 \beta_1\hfill} \\
	\nonumber \\
	&\rho^2 + (A_2 -1)\rho + \frac{B}{(z_2-z_1)^2} = 0 = (\rho - \alpha_2)(\rho - \beta_2) \firstpassage
	& \double{A_2 -1 &= - \alpha_2-\beta_2}{\frac{B}{(z_2-z_1)^2} &= \alpha_2 \beta_2\hfill}	
\end{align}
Dalla \ref{2punti} ricaviamo che
\begin{align}
	&A_1 + A_2 = 2 \firstpassage
	&1 - \alpha_1-\beta_1 + 1 - \alpha_2-\beta_2 =2\nextpassage
	&\alpha_1+ \beta_1 + \alpha_2 + \beta_2 =0 \nextpassagecomm{per confronto}
	&\double{\alpha_1 = - \alpha_2}{\beta_1 = - \beta_2}\\
	\nonumber \\
	&B + B = \alpha_1\beta_1 (z_1-z_2)^2 + \alpha_2\beta_2 (z_2-z_1)^2 \firstpassage
	& B = \frac{(\alpha_1\beta_1 + \alpha_2\beta_2)(z_1-z_2)^2 }{2}
\end{align}
E possiamo quindi riscrivere la EDO nella seguente forma
\begin{align}
	&\ddot{u}(z) + \left[ \sum_{i=1}^{2} \frac{1-( \alpha_i + \beta_i)}{z-z_i} \right] \dot{u}(z) + \left[ \frac{(\alpha_1\beta_1 + \alpha_2\beta_2)(z_1-z_2)^2}{2\prod_{i=1}^2 (z-z_i)^2} \right] u(z) =0 \firstpassagecomm{utilizzo le relazioni fra gli indici appena trovate}
	&\ddot{u}(z) + \left[ \sum_{i=1}^{2} \frac{1+ (-1)^i (\alpha_1 + \beta_1)}{z-z_i} \right] \dot{u}(z) + \left[ \frac{\alpha_1\beta_1(z_1-z_2)^2}{\prod_{i=1}^2 (z-z_i)^2} \right] u(z) =0 
\end{align}
Se adesso applichiamo la trasformazione conforme
\begin{align}
	z \to \frac{z-z_1}{z-z_2} \taleche \double{z_1 \to 0\hfill}{z_2 \to \infty}
\end{align}
otteniamo l'equazione in forma \textbf{di Eulero}
\begin{align}
	&\ddot{u}(z) + \left[ \frac{1+ \alpha_1 + \beta_1}{z} \right] \dot{u}(z) + \left[ \frac{\alpha_1\beta_1}{z^2} \right] u(z) =0 
\end{align}
Di cui già conosciamo le soluzioni, e tornando alle variabili originali avremo due casi
\begin{enumerate}
	\item $\alpha_1 \neq \beta_1$
		\begin{align}
			u(z) = c_1\left(\frac{z-z_1}{z-z_2}\right)^{\alpha_1} + c_2\left(\frac{z-z_1}{z-z_2}\right)^{\beta_1}
		\end{align}
	\item $\alpha_1 = \beta_1$
		\begin{align}
			u(z) = c_1\left(\frac{z-z_1}{z-z_2}\right)^{\alpha_1} + 	c_2\left(\frac{z-z_1}{z-z_2}\right)^{\alpha_1}\ln\left(\frac{z-z_1}{z-z_2}\right)
		\end{align}
\end{enumerate}


\newpage
\subsubsubsection{Equazioni con 3 punti fuchsiani: Papperitz-Riemann}
\subsubsubsection{Equazioni ipergeometriche}


\newpage
\subsection{Esempi}
N'aggio fatto in tempo, tanto Bufalini le ha scritte bene su Latex.