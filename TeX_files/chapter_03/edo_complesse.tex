\section{Equazioni Differenziali Ordinarie Complesse (Alessandro Marcelli)}

\subsection{Spunti di teoria}

\subsubsection{Concetti introduttivi}

Per iniziare il discorso, consideriamo il seguente problema di Cauchy del 1o ordine
\begin{align}
	\double{\dot{u}(x) = p(x)u(x)}{u(x) = u_0 \hfill}
\end{align}
Sappiamo che la soluzione sarà data da
\begin{align}
	u(x) = u_0 \cdot exp \left( - \int_{0}^{x} d\xi \; p(\xi) \right) \label{comp1}
\end{align}
Se prolunghiamo sui complessi e sviluppiamo in serie la $p(\xi)$ otteniamo
\begin{align}
	&p(\xi) = \sum_{k} c_k \xi^k \firstpassage
	&\int d\xi \; p(\xi) = \sum_{k \neq 1} c_k \frac{\xi ^{k+1}}{k+1} + c_{-1} \ln(\xi)
\end{align}
Avremo due casi
\begin{enumerate}
	\item $p(\xi)$ non possiede singolarità, e allora l'unica soluzione possibile è la \ref{comp1}
	\item $p(\xi)$ presenta dei poli, e allora abbiamo due scenari possibili
	\begin{enumerate}
		\item $c_{-1} \neq 0 \implies$ la soluzione è proporzionale a \ref{comp1} e ha una \textbf{singolarità algebrica} dell'origine
		\item $c_{-2} \neq 0 \implies$ la soluzione ha una \textbf{singolarità essenziale} nell'origine
	\end{enumerate}
	La presenza di poli rovina l'unicità della soluzione
\end{enumerate}

Ci concentriamo in questo corso nello studio di equazioni "standard" nella forma
\begin{align}
	\frac{d^2}{dz^2}u(z) + p(z) \frac{d}{dz}u(z) + q(z) u(z) = 0
\end{align}
Preso un punto $z_0 \in \C$, esso viene classificato come
\begin{enumerate}
	\item \textbf{punto regolare} dell'equazione se sia $p(z)$ che $q(z)$ sono analitiche in esso
	\item \textbf{punto singolare} dell'equazione se non lo sono
\end{enumerate}

\newpage

\subsubsection{Punti regolari e soluzione per serie}

Sia il seguente problema di Cauchy
\begin{align}
	\triple{\ddot{u}(z) + p(z) \dot{u}(z) + q(z) u(z) = 0}{\dot{u}(z_0) = c_1 \hfill }{u(z_0) = c_0 \hfill}
\end{align}
e sia $z_0$ un punto regolare dell'equazione. 

Vogliamo dimostrare come in un intorno $|z-z_0| < r_0$ la soluzione sia unica e come sia possibile determinarla con un procedimento iterativo.

Sappiamo come sia possibile trattare i problemi al 2o ordine come combinazione di problemi al 1o ordine. riscriviamo quindi
\begin{align}
	\left\{
	\begin{array}{c}
		\dot{v}(z) = -p(z) v(z) - q(z) u(z) \\
		\dot{u}(z) = v(z) \hfill\\
		v(z_0) = c_1 = \dot{u}(z_0) \hfill\\
		u(z_0) = c_0 \hfill
	\end{array}
	\right.
\end{align}
Questo è un caso particolare dove la dimostrazione passerebbe per la condizione di lipschitzianità invece che per quella di analiticità.

Integriamo formalmente le prime due equazioni, e otteniamo
\begin{align}
	v(z) &= v(z_0) + \int_{z_0}^{z} d\zeta\; [-p(\zeta) v(\zeta) - q(\zeta)u(\zeta)]\\
	u(z) &= u(z_0) + \int_{z_0}^{z} d\zeta \; v(\zeta) 
\end{align}
Andiamo ora a sostituire funzioni costanti pari ai valori di $v(z_0)$ e $u(z_0)$ negli integrali
\begin{align}
	v_1(z) &= v(z_0) + \int_{z_0}^{z} d\zeta\; [-p(\zeta) v_0 - q(\zeta)u_0]\\
	u_1(z) &= u(z_0) + \int_{z_0}^{z} d\zeta \; v_0 
\end{align}
Al passo successivo, poniamo $u_0 = u_1$ e $v_0=v_1$ e ripetiamo
\begin{align}
	v_2(z) &= v_1(z_0) + \int_{z_0}^{z} d\zeta\; [-p(\zeta) v_1(\zeta) - q(\zeta)u_1(\zeta)]\\
	u_2(z) &= u_1(z_0) + \int_{z_0}^{z} d\zeta \; v_1(\zeta) 
\end{align}
Siccome in $z=z_0$ i termini integrali spariscono, avremo in tutte le iterazioni che
\begin{align}
	v_n(z_0) &= v_0\\
	u_n(z_0) &= u_0
\end{align}
e possiamo quindi riscrivere
\begin{align}
	v_2(z) &= v_0 + \int_{z_0}^{z} d\zeta\; [-p(\zeta) v_1(\zeta) - q(\zeta)u_1(\zeta)]\\
	u_2(z) &= u_0 + \int_{z_0}^{z} d\zeta \; v_1(\zeta) \firstpassage
	&\dots \nextpassage
	v_n(z) &= v_0 + \int_{z_0}^{z} d\zeta\; [-p(\zeta) v_{n-1}(\zeta) - q(\zeta)u_{n-1}(\zeta)]\\ 
	u_n(z) &= u_0 + \int_{z_0}^{z} d\zeta \; v_{n-1}(\zeta)
\end{align}
Osserviamo come
\begin{align}
	u_n(z) = u_0 + (u_1 - u_0) + (u_2 - u_1) + \dots + (u_n - u_{n-1})
\end{align}
I termini n-simi possono essere quindi interpretati come somme parziali di ordine $n$
\begin{align}
	v_n(z) &= v_0 + \sum_{k=1}^{n} [v_{k}(z) - v_{k-1}(z)]\\ 
	u_n(z) &= u_0 + \sum_{k=1}^{n} [u_{k}(z) - u_{k-1}(z)]
\end{align}
Vogliamo quindi dimostrare che
\begin{align}
	&\limit{n}{\infty} u_n(z) = u(z)\\
	&\limit{n}{\infty} v_n(z) = v(z)\\
	&u(z), v(z) \quad \text{analitiche}
\end{align}
Definite le due grandezze
\begin{align}
	M &= max(|p|,|q|)\\
	m &> max(|u_0|,|v_0|)
\end{align}
Confrontandole con le espressioni di $u_1(z)$ e $v_1(z)$ otteniamo
\begin{align}
	|u_1(z) - u_0| &\leq m |z-z_0|\\
	|v_1(z) - v_0| &\leq 2Mm |z-z_0|	
\end{align}
Definendo
\begin{align}
	M_1 = max(m, 2Mm)
\end{align}
Possiamo maggiorare come
\begin{align}
	|u_1(z) - u_0| &\leq M_1 |z-z_0|\\
	|v_1(z) - v_0| &\leq M_1 |z-z_0|	
\end{align}
Passando all'iterazione successiva avremo
\begin{align}
	u_2(z) - u_1(z) = \left(\cancel{u_0} + \int_{z_0}^{z} d\zeta \; v_1(\zeta) \right) - \left(\cancel{u_0} + \int_{z_0}^{z} d\zeta \; v_0(\zeta) \right) = \int_{z_0}^{z} d\zeta \; [v_1(\zeta) - v_0(\zeta)]
\end{align}
E analogamente avremo
\begin{align}
	v_2(z) - v_1(z) = \int_{z_0}^{z} d\zeta\; [-p(\zeta) (v_1(\zeta)-v_0) - q(\zeta)(u_1(\zeta)-u_0)]
\end{align}
Anche qui possiamo quindi maggiorare come
\begin{align}
	u_2(z) - u_1(z) &= M_1\int_{z_0}^{z} d\zeta\; [\zeta - z_0)]\\
	v_2(z) - v_1(z) &= 2M_1m\int_{z_0}^{z} d\zeta\; [\zeta - z_0)]
\end{align}
Anche qui possiamo maggiorare con
\begin{align}
	&M_2 = max(M_1, 2M_1 m) \firstpassage
	&M_1\int_{z_0}^{z} d\zeta\; [\zeta - z_0)] \leq M_2 \frac{|z-z_0|^2}{2} = M_2 \frac{r^2}{2}
\end{align}
Da cui possiamo scrivere
\begin{align}
	|u_2(z) - u_1(z)| &\leq M_n \frac{r^n}{n!}\\
	|v_2(z) - v_1(z)| &\leq M_n \frac{r^n}{n!}	
\end{align}
Se adesso riprendiamo la
\begin{align}
	u_n(z) &= u_0 + \sum_{k=1}^{n} [u_{k}(z) - u_{k-1}(z)]
\end{align}
Possiamo riscrivere
\begin{align}
	|u_n(z)| &\leq \sum_{k=0}^{n} \frac{(Mr)^n}{n!}
\end{align}
Stesso discorso facciamo per $v(z)$.

Esiste quindi per forza un $r_0$ tale per cui la serie converge in modo assoluto e uniforme, e che quindi definisce una funzione analitica.

Possiamo quindi ridefinire la $u(z)$ tramite la sua espansione di taylor come
\begin{align}
	\triple{u(z) = \sum_{n=0}^{\infty} c_n(z-z_0)^n}{\dot{u}(z_0) = v_0 = c_1 \hfill}{u(z_0) = u_0 = c_0 \hfill}
\end{align}
Da qui non ci resta che ricavare i coefficienti. Esplicitiamo i termini e sostituiamo
\begin{align}
       u(z) &= c_0 + c_1 (z-z_0) + \sum_{n=2}^{\infty} c_n(z-z_0)^n = \nonumber \\
            &= u_0 + v_0 (z-z_0) + \sum_{n=2}^{\infty} c_n(z-z_0)^n \firstpassage
 \dot{u}(z) &= v_0 + \sum_{n=2}^{\infty} c_n n(z-z_0)^{n-1} \firstpassage
\ddot{u}(z) &= \sum_{n=2}^{\infty} c_n n(n-1)(z-z_0)^{n-2}
\end{align}
Sostituiamo nell'equazione generale e otteniamo
\begin{align}
	\ddot{u}(z) =& - p(z) \dot{u}(z) - q(z) u(z) \firstpassage
	\sum_{n=2}^{\infty} c_n n(n-1)(z-z_0)^{n-2} =& -p(z) \left(v_0 + \sum_{n=2}^{\infty} c_n n(z-z_0)^{n-1}\right) - q(z) \left( u_0 + v_0 (z-z_0) + \sum_{n=2}^{\infty} c_n(z-z_0)^n \right) \nextpassage
	\sum_{n=2}^{\infty} c_n n(n-1)(z-z_0)^{n-2} =& -p(z)v_0 - q(z)[u_0 + v_0 (z-z_0) ] + \nonumber\\ 
	&- \sum_{n=2}^{\infty} c_n [np(z)(z-z_0)^{n-1} + q(z)(z-z_0)^n] \label{serieode}
\end{align}
\newpage
Per comodità ripetiamo qui la \ref{serieode}
\begin{align}
	\sum_{n=2}^{\infty} c_n n(n-1)(z-z_0)^{n-2} =& -p(z)v_0 - q(z)[u_0 + v_0 (z-z_0) ] + \nonumber\\ 
	&- \sum_{n=2}^{\infty} c_n [np(z)(z-z_0)^{n-1} + q(z)(z-z_0)^n]
\end{align}
Si procede quindi per confronto, trovando
\begin{align}
	n = 2 \to c_2 \cdot 2 \cdot (2-1) (z-z_0)^{2-2} = 2c_2
\end{align}
A dx della \ref{serieode} non abbiamo alcun termine corrispondente, ergo
\begin{align}
	c_2 = 0
\end{align}
Procediamo e troviamo
\begin{align}
	n = 3 \to c_3 \cdot 3 \cdot (3-1) (z-z_0)^{3-2} = 6c_3 (z-z_0)
\end{align}
A dx troviamo che il termine corrispondente è
\begin{align}
	- q(z)v_0 (z-z_0)
\end{align}
E dunque per confronto
\begin{align}
	&6c_3 \cancel{(z-z_0)} = - q(z)v_0 \cancel{(z-z_0)}\firstpassage
	&c_3 = -\frac{v_0}{6}q(z) 
\end{align}
Proseguiamo imperterriti a mazzetta fino a trovare relazioni ricorrenti tra i coefficienti, e otteniamo così il  nostro sviluppo in serie.

\newpage

\subsubsection{Punti singolari}

Per costruzione, le singolarità di una EDO della forma
\begin{align}
	\ddot{u}(z) + p(z) \dot{u}(z) + q(z)u(z) = 0
\end{align}
devono coincidere con quelle delle funzioni $p(z)$ e $q(z)$, esono divise in due categorie
\begin{enumerate}
	\item punti \textbf{singolari regolari} (anche detti \textbf{inessenziali} o \textbf{fuchsiani}).
	
	Sono particolarmente importanti quelli in cui le soluzioni non presentano singolarità essenziali.
	\item punti \textbf{singolari irregolari}
\end{enumerate}
Concentriamoci un attimo sul caso di un punto singolare regolare $z_0$. Le soluzioni che possono essere trovate intorno ad esso di base non saranno in generale monodrome, e qualora si applichi un prolungamento ci troveremo ad avere
\begin{align}
	(z-z_0) &\to (z-z_0)e^{2\pi i}\\
	u_1(z)  &\to U_1(z)\\
	u_2(z)  &\to U_2(z)
\end{align}
Sappiamo però che $u_1(z)$ e $u_2(z)$ sono linearmente indipendenti, e formano quindi una base nello spazio delle soluzioni.

Di conseguenza possiamo scrivere
\begin{align}
	&\double{U_1(z) = a_{11} u_1(z) + a_{12} u_2(z)}{U_2(z) = a_{21} u_1(z) + a_{22} u_2(z)}\firstpassage
	&\left(
	 \begin{array}{c}
		 U_1(z)\\
		 U_2(z)
	 \end{array}\right) = 
	 \begin{pmatrix}
		 a_{11} & a_{12}\\
		 a_{21} & a_{22}
	 \end{pmatrix} 	\left(
     \begin{array}{c}
		 u_1(z)\\
		 u_2(z)
	\end{array}\right)
\end{align}
Si può dimostrare come
\begin{align}
	W(U_1,U_2) = \det(A) \cdot W(u_1, u_1)
\end{align}

\paragraph{Considerazioni sulla matrice}  

La matrice $A$ è non singolare e dipendente dalla polidromia di $u_1(z)$ e $u_2(z)$. Come di consueto gli autovalori si ricavano dall'equazione caratteristica
\begin{align}
	\det(A -\lambda \mathbb{1}) = 0
\end{align}
e possono essere distinti o coincidenti.

A noi interessa trovare una base dove le polidromie in gioco siano le più semplici possibili. 

Questo si traduce nello scegliere una base che renda o diagonale o di Jordan (diagonale a blocchi) la matrice A.
\newpage

\subsection{Esempi}
N'aggio fatto in tempo, tanto Bufalini le ha scritte bene su Latex.