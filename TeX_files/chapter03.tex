\chapter{Equazioni Differenziali Ordinarie}



\section{Spunti di teoria generale (Alessandro Marcelli)}

\subsection{Concetti generali}

Un'\textbf{equazione differenziale ordinaria} (EDO) di ordine $n$ viene genricamente definita come
\begin{align}
	\Phi(x, u, u', u'', \dots, u^{(n)}) = 0
\end{align}
In questo corso ci si occupa solo di equazioni che possono essere scritte in \textbf{forma normale}, ovvero dove si può esplicitare in funzione della derivata massima
\begin{align}
	u^{(n)}=f(x, u, u', u'', \dots, u^{(n-1)})
\end{align}
con la proprietà di poter essere riscritte come sistemi di $n$ EDO del 1o ordine
nel seguente modo
\begin{align}
	\left\{\begin{array}{c}
		v_1 \quad = u' \hfill \\
		v_2 \quad = v_1' = u'' \hfill \\
		\dots \hfill \\
		v_{n-1} = v_{n-2}' \hfill \\
		v_n \quad = f(x, u, v_1, v_2, \dots, v_{n-1}) \hfill 
	\end{array}
	\right.
\end{align}
Questo sistema definisce un insieme infinito di curve detto \textbf{integrale generale} dell'equazione.

Per individuare una curva specifica bisogna specificare almeno un punto  per cui deve passare, e i rispettivi valori che le varie derivate vi devono assumere fino alla $u^{(n-1)}$. Questo prende il nome di \textbf{problema di Cauchy}.

Nota: visto che come abbiamo visto i problemi di ordine superiore al primo possono essere visti come combinazione di EDO al primo ordine, nella teoria ci appoggeremo a problemi del 1o ordine del tipo
\begin{align}
	\left\{\begin{array}{c}
		u'(x) \; = F(x,u(x)) \hfill \\
		u(x_0) \;  = u_0 \hfill 
	\end{array}
	\right.
\end{align}

\subsection{Esistenza ed unicità delle soluzioni}

L'esistenza delle soluzioni di una EDO è dettata dai seguenti due teoremi:
\begin{enumerate}
	\item \textbf{Teorema di Cauchy:} \textit{sia assegnato un problema di Cauchy in un dominio
		\begin{align}
			\mathcal{D} = (a,b)x(c,d) \taleche (x_0,u_0) \in \mathcal{D}
		\end{align}
		Se rispetto ad $u$ si verifica la condizione di Lipschitzianità per $F(x,u)$
		\begin{align}
			|F(x,u_1) - F(x,u_2)| \leq L|u_1 - u_2|
		\end{align}
		Allora la soluzione del problema di Cauchy esiste ed è unica.	
	}
	\item \textbf{Teorema di Peano:} \textit{sia assegnato un problema di Cauchy in un dominio
		\begin{align}
			\mathcal{D} = (a,b)x(c,d) \taleche (x_0,u_0) \in \mathcal{D}
		\end{align}
		Se la $F(x,u)$ è continua, allora esiste un opportuno intorno $(x_0, x_0+\delta) \in \mathcal{D}$ tale per cui la soluzione del problema esiste.
	}
\end{enumerate}

Qualora ci si trovi nel secondo caso, non è più sufficiente un unico punto per determinare la soluzione, e bisogna ijnvece specificare, in numero pari all'ordine della EDO, delle \textbf{condizioni di bordo} (in inglese \textbf{Boundary Conditions}, BC), arriviamo così a definire due tipi di problemi:
\begin{enumerate}
	\item \textbf{Problemi di Dirichlet:} condizioni al bordo su $u(x)$. 
	
	Nel caso di equazioni lineari, questi si riconducono allo studio degli autovalori dei corrispondenti operatori lineari, e prendono il nome di \textbf{Problemi di Sturm-Liouville}.
	\item \textbf{Problemi di Neumann:} condizioni al bordo su $u'(x)$
\end{enumerate}

\subsection{Distribuzioni}

Le \textbf{distribuzioni} sono una famiglia di operatori lineari integrali definite come
\begin{align}
	(T,\phi) &= \int_{-\infty}^{+\infty} dx \; T(x) \phi(x)\\
	(T',\phi) &= \int_{-\infty}^{+\infty} dx \; T'(x) \phi(x) = -\int_{-\infty}^{+\infty} dx \; T(x) \phi'(x) = -(T,\phi')
\end{align}

Per quanto riguarda questo corso ci limiteremo allo spazio $S^{\infty}$ delle \textbf{funzioni di prova}, con derivate continue di ogni ordine che all'infinito tendono a zero più rapidamente di ogni potenza.

Esempi noti di distribuzioni sono la \textbf{delta} e la \textbf{theta}.

La continuità delle distribuzioni come funzionali consente di ottenere una loro definizione come limite di altre distribuzioni.

\subsection{Equazioni Differenziali Lineari}
Si parla di EDO lineari di ordine $n$ quando possono essere poste nella forma
\begin{align}
	&\mathcal{L}^{(n)}_x u(x) = f(x) \label{edolin}\\
	&\mathcal{L}^{(n)}_x = \sum_{k=0}^{n} c_k(x) \frac{d^k}{dx^k}
\end{align}
Lo studio passa quindi da $F$ ai vari $c_k$. Notiamo come possiamo avere singolarità della soluzione solo intorno a punti singolari dei coefficienti, che definiamo come \textbf{punti singolari fissi}.

Spesso le soluzioni sono note solo attraverso sviluppi in serie di Laurent intorno a tali punti.

Nel caso lineare, la continuità dei coefficienti ci garantisce esistenza ed unicità delle soluzioni.

Definiamo la \textbf{omogenea associata} dell'equazione \ref{edolin} come
\begin{align}
	&\mathcal{L}^{(n)}_x u(x) = \sum_{k=0}^{n} c_k(x) \frac{d^k}{dx^k}u(x) = 0
\end{align}
L'integrale generale di una EDO lineare può essere scritto come combinazione della soluzione dell'omogenea $u_{om}(x)$ e di una soluzione particolare $u_{p}(x)$
\begin{align}
	u(x) = u_{om}(x) + u_{p}(x)
\end{align}


\section{EDO lineari del 1o ordine [teoria](Alessandro Marcelli)}

Sia il problema di Cauchy
\begin{align}
	\left\{\begin{array}{c}
		c_1 (x) u'(x) + c_0(x) u(x) \; = f(x) \hfill \\
		u(x_0) \;  = u_0 \hfill 
	\end{array}
	\right. \label{ode1ord1}
\end{align}
Possiamo risolvere il problema in due modi:

\subsection{Separazione di variabili + variazione delle costanti}
	\begin{enumerate}
		\item La soluzione dell'omogenea si ricava per \textbf{separazione delle variabili}
		\begin{align}
			&c_1 (x) u'(x) + c_0(x) u(x) \; = 0 \firstpassage
			&c_1 (x) u'(x) = - c_0(x) u(x) \nextpassage
			&\frac{du}{dx} = -\frac{c_0(x)}{c_1(x)} u(x)\nextpassage
			&\int_{u_0}^u du \; \frac{1}{u}= - \int_{x_0}^x dt \;  \frac{c_0(t)}{c_1(t)}\nextpassage
			&\ln(u(x)) - C = - \int_{x_0}^x dt \;  \frac{c_0(t)}{c_1(t)} \spacer C = \ln (u_0)\nextpassage
			&u_{om}(x) = e^C e^{- \int_{x_0}^x dt \;  \frac{c_0(t)}{c_1(t)}}
		\end{align}
		\item La soluzione della particolare invece col metodo della \textbf{variazione delle costanti}, ovvero ponendo
		\begin{align}
			&u_p(x) = a(x) u_{om}(x)\firstpassage
			&u'_p(x)= a'(x) u_{om}(x) +  a(x) u'_{om}(x) \nextpassage
			&c_1(x) \cdot [a'(x) u_{om}(x) +  a(x) u'_{om}(x)] + c_0 a(x) u_{om}(x) =f(x) \nextpassage
			&c_1(x)u_{om}(x) \cdot a'(x) + \cancel{[c_1(x)u'_{om}(x) + c_0(x)u_{om}(x)]\cdot a(x)} = f(x)\nextpassage
			&c_1(x)u_{om}(x) \cdot a'(x) = f(x) \nextpassage
			&a'(x) =\frac{f(x)}{c_1(x)u_{om}(x)} \nextpassage
			&a(x) = \int_{x_0}^{x} dt\; \frac{f(t)}{c_1(t)u_{om}(t)} \nextpassage
			&u_p(x) = \left[ \int_{x_0}^{x} dt\; \frac{f(t)}{c_1(t)u_{om}(t)} \right] \cdot \left[ e^C e^{- \int_{x_0}^x dt \;  \frac{c_0(t)}{c_1(t)}} \right]
		\end{align}
		\item si sommano le due soluzioni, e si trova la curva che rispetta le condizioni iniziali del problema
	\end{enumerate}
	
\subsection{Metodo della funzione di Green} 
Il metodo della \textbf{funzione di Green} lo abbiamo già introdotto nel capitolo degli operatori. Andiamo a cercare una soluzione nel senso delle distribuzioni per
	\begin{align}
		\mathcal{L}^{(1)}_x G(x,y) = \delta(x-y) \label{green1}
	\end{align}
Abbiamo visto come l'integrale generale, la soluzione fondamentale è definita a meno di soluzioni dell'omogenea. 

Prendiamo in considerazione un problema a \textbf{condizioni omogenee} (o \textbf{nulle})
\begin{align}
	\left\{\begin{array}{c}
		c_1 (x) u'(x) + c_0(x) u(x) \; = g(x) \hfill \\
		v(x_0) \;  = 0 \hfill 
	\end{array}
	\right.
\end{align}

\textbf{Nota:} qualunque problema del tipo \ref{ode1ord1} può essere riscritto in questo modo col cambio di variabili
\begin{align}
	v(x) &= u(x) - u_0\\
	g(x) &= f(x) - c_0(x)u_0
\end{align}

Per il \textbf{teorema dell'alternativa}, se l'omogenea ammette solo la soluzione nulla, allora il problema ha soluzione unica.

In questo caso $\mathcal{L}^{(1)}_x$ risulta essere invertibile e la soluzione fondamentale con condizione iniziale omogenea $G(x_0,y) = 0$ della \ref{green1} prende il nome di \textbf{funzione di Green} dell'operatore ed è il nucleo integrale dell'operatore inverso. Grazie ad essa possiamo ricavare la soluzione come \textbf{convoluzione} col termine forzante $g(x)$
\begin{align}
	u(x) = \int dy \; G(x,y) g(y) = G * g
\end{align}
Andiamo a risolvere la \ref{green1}
\begin{align}
	&c_1 (x) \frac{d}{dx}G(x,y) + c_0(x) G(x,y)  = \delta(x-y) \firstpassage
	&\frac{d}{dx}G(x,y) + \frac{c_0(x)}{c_1(x)} G(x,y)  = \frac{\delta(x-y)}{c_1(x)}\firstpassage
	&\int_{y-\epsilon}^{y+\epsilon} dx \;\frac{d}{dx}G(x,y) + \int_{y-\epsilon}^{y+\epsilon} dx \;\frac{c_0(x)}{c_1(x)} G(x,y)  = \int_{y-\epsilon}^{y+\epsilon} dx \;\frac{\delta(x-y)}{c_1(x)}
\end{align}
Essendo i $c_k(x)$ regolari, l'integrando può al più fare un "salto" e quindi la sua primitiva $F$ è continua, e quindi nel limite $\epsilon \to 0$ il secondo integrale si annulla
\begin{align}
\limit{\epsilon}{0}\int_{y-\epsilon}^{y+\epsilon} dx \;\frac{c_0(x)}{c_1(x)} G(x,y)  = \int_{y}^{y} dx \;\frac{c_0(x)}{c_1(x)} G(x,y) = F(y,y) - F(y,y) = 0
\end{align}
Otteniamo quindi la condizione di salto
\begin{align}
&G(y+\epsilon,y) - G(y-\epsilon,y) = \frac{1}{c_1(y)}\firstpassage
&G(y_+,y) = G(y_-,y) + \frac{1}{c_1(y)}
\end{align}

\newpage
Andiamo ora a calcolare una soluzione dell'omogenea per \ref{green1}
\begin{align}
	&\frac{d}{dx}G(x,y) + \frac{c_0(x)}{c_1(x)} G(x,y)  = 0\firstpassage
	&G(x,y) = A_-(y) exp\left( - \int_{x}^{y} d\xi\; \frac{c_0(\xi)}{c_1(\xi)}\right) \; ; \;  x<y\\
	&G(x,y) = A_+(y) exp\left( - \int_{x}^{y} d\xi\; \frac{c_0(\xi)}{c_1(\xi)}\right) \; ; \; x>y
\end{align}
Applicando la condizione di salto
\begin{align}
	&A_+(y_) = A_-(y) + \frac{1}{c_1(y)} \firstpassage
	&G(x,y) = A_-(y) \cdot exp\left( - \int_{x}^{y} d\xi\; \frac{c_0(\xi)}{c_1(\xi)}\right) \; ; \;  x<y\\
	&G(x,y) = \left[  A_-(y) + \frac{1}{c_1(y)} \right] \cdot exp\left( - \int_{x}^{y} d\xi\; \frac{c_0(\xi)}{c_1(\xi)}\right) \; ; \;  x>y \firstpassage
	&G(x,y) = \left[  A_-(y) + \frac{\theta(x-y)}{c_1(y)} \right] \cdot exp\left( - \int_{x}^{y} d\xi\; \frac{c_0(\xi)}{c_1(\xi)}\right)	
\end{align}
Nel caso di problemi alle condizioni omogenee, possiamo fare l'ipotesi $x_0 < y$ , e la funzione di Green del problema deve soddisfare $G(x_0,y) = 0$, il che ci restituisce
\begin{align}
&G(x,y) = \frac{\theta(x-y)}{c_1(y)} \cdot exp\left( - \int_{y}^{x} d\xi\; \frac{c_0(\xi)}{c_1(\xi)}\right)	
\end{align}
Invece, nel caso di problemi generali del tipo \ref{ode1ord1}, il procedimento si snoda in due passaggi:
\begin{enumerate}
	\item Calcoliamo la convoluzione della generica soluzione fondamentale, tenendo a mente che $a<x_0<b$
	\begin{align}
		u(x) &= \int_a^b dy \; G(x,y) f(y) =	\nonumber\\
			 &= \int_{x_0}^b dy \;\left[  A(y) + \frac{\theta(x-y)}{c_1(y)} \right] \cdot exp\left( - \int_{y}^{x} d\xi\; \frac{c_0(\xi)}{c_1(\xi)}\right) \cdot f(y) =	\nonumber\\
			 &= \int_{x_0}^b dy \; A(y) f(y) \cdot exp\left( - \int_{y}^{x} d\xi\; \frac{c_0(\xi)}{c_1(\xi)}\right) + \nonumber\\
			 &+ \int_{x_0}^x dy \;  \frac{f(y)}{c_1(x)} \cdot exp\left( - \int_{y}^{x} d\xi\; \frac{c_0(\xi)}{c_1(\xi)}\right) 
	\end{align}
	\item Imponiamo ora le condizioni iniziali del problema, ottenendo così la definizione implicita di $A(y)$
	\begin{align}
		u_0 &= \int_{x_0}^b dy \; A(y) f(y) \cdot exp\left( - \int_{y}^{_0} d\xi\; \frac{c_0(\xi)}{c_1(\xi)}\right) 
	\end{align}
	Da cui otteniamo infine l'integrale generale
	\begin{align}
		u(x) &= u_0 \cdot  exp\left( - \int_{y}^{x} d\xi\; \frac{c_0(\xi)}{c_1(\xi)}\right) + \int_{x_0}^x dy \;  \frac{f(y)}{c_1(x)} \cdot exp\left( - \int_{y}^{x} d\xi\; \frac{c_0(\xi)}{c_1(\xi)}\right) 
	\end{align}
\end{enumerate}

\newpage

\section{EDO lineari del 2o ordine (Adriano Chialastri, Alessandro Marcelli)}


\subsection{Spunti di teoria (Alessandro Marcelli)}

Le EDO del 2o ordine sono rappresentabili come
\begin{align}
	c_2(x) u''(x) + c_1(x) u'(x) + c_0(x)u(x) = f(x)
\end{align}
Seguendo il procedimento illustrato in \ref{canon}, possono essere sempre ridotte in \textbf{forma canonica}
\begin{align}
	v''(x) + c(x)v(x) = g(x)
\end{align}
dove, ovviamente le condizioni al contorno vanno trasformate di conseguenza.

Rispetto alle EDO del 1o ordine, c'è la difficoltà aggiuntiva data dal fatto che non sempre si può trovare una soluzione dell'omogenea (e quindi l'integrale generale). Questo problema non si pone però per due famiglie di equazioni omogenee:
\begin{enumerate}
	\item \textbf{Omogenee a coefficienti costanti:}
	\begin{align}
		a u''(x) + b u'(x) + c u(x) = 0 \label{edo2ord1}
	\end{align}
	In questo caso si procede con l'\textbf{Ansatz} (o \textbf{ipotesi}) $u(x) = e^{\alpha x}$ e si va a cercare il parametro $\alpha$ tramite l'equazione caratteristica
	\begin{align}
		&a \alpha^2 + b \alpha + c = 0\firstpassage
		& \alpha_{1,2} = \frac{-b \pm \sqrt{b^2 - 4ac}}{2a} \firstpassage
		&u_1(x) = e^{\alpha_1 x} \spacer u_2(x) = e^{\alpha_2 x}
	\end{align}
	Se le soluzioni coincidono, e quindi si ha
	\begin{align}
		\alpha = -\frac{b}{2a}
	\end{align}
	si procede nel seguente modo
	\begin{align}
		&u_1(x) = e^{\alpha x} \spacer u_2(x) = \rho(x)e^{\alpha x} \firstpassage
		&\cancel{(a \alpha^2 + b \alpha + c)\rho(x)e^{\alpha x}} + (a \rho''(x) + b \rho'(x) + 2a\alpha \rho'(x))e^{\alpha x} = 0
	\end{align}
	Il primo termine si annulla, vista la \ref{edo2ord1}, otteniamo quindi
	\begin{align}
		&(a \rho''(x) + b \rho'(x) + 2a\alpha \rho'(x))e^{\alpha x} = 0\firstpassage
		&a \rho''(x) + b \rho'(x) + 2a\alpha \rho'(x)=0 \nextpassage
		&a \rho''(x) + b \rho'(x) - \cancel{2a}\frac{b}{\cancel{2a}} \rho'(x)=0 \to a \rho''(x) + \cancel{b \rho'(x)} - \cancel{b\rho'(x)} =0 \nextpassage
		&\rho''(x) = 0 \nextpassage
		&\rho(x) = c_1x + c_2
	\end{align}
	la soluzione più semplice si ha per $c_1 = 1$ e $c_2 = 0$, che ci restituisce
	\begin{align}
		u_2(x) = x e^{\alpha x}
	\end{align}

	\item \textbf{Equazioni di Eulero del 2o ordine:}
	\begin{align}
		u''(x) + \frac{a}{x}u'(x) + \frac{b}{x^2}u(x) = 0 \spacer a,b \;\;\ \text{costanti}
	\end{align}
	Stavolta le soluzioni le andiamo a cercare con l'Ansatz $u(x) = x^\alpha$, ottenendo la seguente equazione caratteristica
	\begin{align}
		&\alpha (\alpha - 1) + a\alpha + b = 0 \firstpassage
		&\alpha^2 + (a-1)\alpha + b = 0 \nextpassage
		&\alpha_{1,2} = \frac{(1-a) \pm \sqrt{(a-1)^2 - 4b}}{2} \firstpassage
		&u_1 (x) = x^{\alpha_1} \spacer u_2 (x) = x^{\alpha_2}
	\end{align}
	Nel caso di soluzioni coincidenti
	\begin{align}
		a = \frac{1-a}{2}
	\end{align}
	conviene trasformare l'equazione in una a coefficienti costanti col cambio di variabili
	\begin{align}
		&x(t) = e^t\firstpassage
		&\frac{d}{dx}u(x(t)) =e^{-t} \frac{du}{dt} \nextpassage
		&\frac{d^2}{dx^2}u(x(t)) =e^{-2t} \frac{d^2u}{dt^2} - e^{-2t} \frac{du}{dt} \nextpassage
		&e^{-2t} \left[ \frac{d^2u}{dt^2} + (a-1)\frac{du}{dt} + bu \right] = 0 \nextpassage
		&u_1(t) = e^{\alpha t} \spacer u_2(t) = te^{\alpha t} \nextpassage
		&u_1(x) = x^\alpha \spacer u_2(x) = x^\alpha \ln(x)
	\end{align}
\end{enumerate}

\newpage

\subsubsection{Metodo del Wronskiano}

Si definisce il \textbf{determinante di Wronsky} ( anche detto \textbf{Wronskiano}) di due funzioni $u_1$ e $u_2$ come
\begin{align}
	W(u_1(x),u_2(x)) = W(x) = \begin{vmatrix}
		u_1(x) & u_2(x) \\
		u'_1(x) &u'_2(x)
	\end{vmatrix} = u_1(x) u'_2(x) - u'_1(x) u_2(x) 
\end{align}
Se  $u_1(x)= a\cdot u_2(x)$, ovvero sono linearmente dipendenti, allora $W=0$.

Presa in considerazione una EDO del tipo
\begin{align}
	&a(x) u''(x) + b(x) u'(x) + c(x) u(x) = 0 \firstpassage
	&u''(x) + \frac{b(x)}{a(x)} u'(x) + \frac{c(x)}{a(x)} u(x) = 0\nextpassage
	&u''(x) + a_1(x) u'(x) + a_2(x) u(x) = 0 \\
	&a_1(x) =  \frac{b(x)}{a(x)} \spacer a_2(x)=\frac{c(x)}{a(x)}
\end{align}
e prese duesaoluzioni $u_1(x)$ e $u_2(x)$ possiamo ricavare
\begin{align}
	&u_1(x)u_2''(x) + a_1(x) u_1(x)u_2'(x) + a_2(x) u_1(x)u_2(x) = 0 \\
	&u_2(x)u_1''(x) + a_1(x) u_2(x)u_1'(x) + a_2(x) u_2(x)u_1(x) = 0\firstpassagecomm{sottraggo membro a membro}
	&u_1(x)u_2''(x) - u_2(x)u_1''(x) + a_1(x) u_1(x)u_2'(x) - a_1(x) u_2(x)u_1'(x) + \cancel{ a_2(x) u_1(x)u_2(x)} - \cancel{a_2(x) u_2(x)u_1(x)} = 0 \nextpassage
	&[u_1(x)u_2''(x) - u_2(x)u_1''(x)]+ a_1(x) [  u_1(x)u_2'(x) - u_2(x)u_1'(x)] = 0\nextpassage
	&W'(x) + a_1(x)W(x) = 0
\end{align}
Per separazione delle variabili, e definendo $W(x_0)=W_0$ otteniamo la \textbf{formula di Liouville}
\begin{align}
	W(x) = W_0\cdot exp\left( -\int_{x_0}^{x} d\xi a_1(\xi) \right)
\end{align}
Da cui notiamo che
\begin{enumerate}
	\item In formacanonica $W=0$
	\item Se $\exists x_1 \taleche W(x_1) = 0 \to W(x) = 0\quad\forall x$
	\item Se $u_1$ e $u_2$ sono lin ind allora $W(x) \neq 0 \quad \forall x$
\end{enumerate}
Nota una delle due soluzioni, dalla formula di Liouville possiamo ricavare una definizione implicita dell'altra nel seguente modo
\begin{align}
	&\frac{W}{u_1^2(x)} = \frac{u_1(x)u_2'(x) - u_2(x)u_1'(x)}{u_1^2(x)}\\
	&\frac{W}{u_1^2(x)} = \frac{W_0\cdot exp\left( -\int_{x_0}^{x} d\xi a_1(\xi) \right)}{u_1^2(x)} \firstpassage
	&\frac{u_1(x)u_2'(x) - u_2(x)u_1'(x)}{u_1^2(x)} = \frac{W_0\cdot exp\left( -\int_{x_0}^{x} d\xi a_1(\xi) \right)}{u_1^2(x)}
\end{align}
Il termnine a sinistra altro non è che 
\begin{align}
&\frac{u_1(x)u_2'(x) - u_2(x)u_1'(x)}{u_1^2(x)} = \frac{d}{dx} \left( \frac{u_2(x)}{u_1(x)} \right)
\end{align}
Da cui otteniamo, sostituendo $W_0$ con $A$ per liberare l'estremo inferior edi integrazione
\begin{align}
&\frac{d}{dx} \left( \frac{u_2(x)}{u_1(x)} \right) = \frac{W_0\cdot exp\left( -\int_{x_0}^{x} d\xi a_1(\xi) \right)}{u_1^2(x)} \firstpassage
& \int^x d\eta \; \frac{d}{d\eta} \left( \frac{u_2(\eta)}{u_1(\eta)} \right) =\int^x d\eta \; \frac{A \cdot exp\left( -\int_{x_0}^{\eta} d\xi a_1(\xi) \right)}{u_1^2(\eta)} + c_1 \nextpassage
&u_2(x) = Au_1(x) \int^x d\eta \; \frac{1}{u_1^2(\eta)}\cdot exp\left( -\int_{x_0}^{\eta} d\xi a_1(\xi) \right) + c_1u_1(x)
\end{align}


\subsubsection{Metodo della funzione di Green}
Consideriamo ora l'operatore differenziale in forma normale
\begin{align}
	\mathcal{L}_x^{(2)} =\frac{d^2}{dx^2} + a_1(x)\frac{d}{dx} + a_2(x)
\end{align}
Ne definiamo una \textbf{soluzione fondamentale} $G(x,y)$ come
\begin{align}
	&\mathcal{L}_x^{(2)} G(x,y)= \delta(x-y) \label{greenedo2}
\end{align}
Imponiamo la condzione di continuità 
\begin{align}
	&\frac{d^2}{dx^2}G(x,y) \propto \delta(x-y)
\end{align}
Da cui 
\begin{align}
	&\left.\frac{d}{dx}G(x,y) \right|_{y-\epsilon}^{y+\epsilon} + 0 = 1 \firstpassage
	&G'(y_+,y) - G'(y_-,y) = 1
\end{align}
Siano ora $u_1(x)$ e $u_2(x)$ due soluzioni della omogenea associata alla \ref{greenedo2}, possiamo scrivere
\begin{align}
	G(x,y) = A u_1(x) + B u_2(x) + G_P(x,y)
\end{align}
La soluzone particolare può essere scritta nella forma
\begin{align}
	G_P(x,y) = \double{c_1 u_1 (x) + c_2 u_2 (x) \quad x<y}{d_1 u_1 (x) + d_2 u_2 (x) \quad x>y} \spacer c_i,d_i = \double{costanti}{c_i(y), d_i(y)}
\end{align}
Per la condizione di continuità abbiamo che
\begin{align}
	&c_1 u_1 (y) + c_2 u_2 (y) = d_1 u_1 (y) + d_2 u_2 (y)\firstpassage
	&(c_1 - d_1) u_1 (y) +(c_2 - d_2)u_2 (y) =0
\end{align}
Mentre per il salto della derivata
\begin{align}
	&d_1 u'_1 (y) + d_2 u'_2 (y) - c_1 u_1 (y) - c_2 u_2 (y) = 1 \firstpassage
	&(d_1 - c_1)u'_1 (y) + (d_2 - c_2)u'_2 (y) = 1 
\end{align}
Otteniamo quindi il sistema
\begin{align}
	\double{(c_1 - d_1) u_1 (y) +(c_2 - d_2)u_2 (y) =0}{(d_1 - c_1)u'_1 (y) + (d_2 - c_2)u'_2 (y) = 1}
\end{align}
Se prendiamo come incognite le differenze tra i coefficienti tra parentesi, possiamo riscrivere il sistema come il prdotto matrice-vettore
\begin{align}
	\begin{pmatrix}
		u_1 (y) & u_2 (y)\\
		u'_1 (y) & u'_2 (y)
	\end{pmatrix} \left( \begin{array}{c}
	c_1 - d_1\\
	c_2 - d_2
\end{array} \right) = \left( \begin{array}{c}
0\\
1
\end{array} \right)
\end{align}
Notiamo come la matrice dei coefficienti altro non è che la matrice di Wronsky di $u_1(x)$ e $u_2(x)$, e che quindi non ha mai determinante nullo. 

Possiamo quindi scrivere le soluzioni come
\begin{align}
d_1 - c_1 &= \frac{1}{W(y)} \cdot \begin{vmatrix}
			0 & u_2(y) \\
			1 & u'_2(y)
	\end{vmatrix} = -\frac{u_2(y)}{W(y)} \to d_1 = c_1  -\frac{u_2(y)}{W(y)}\\
d_2 - c_2 &= \frac{1}{W(y)} \cdot \begin{vmatrix}
	u_1(y) & 0\\
	u'_1(y) & 1
\end{vmatrix} = +\frac{u_1(y)}{W(y)} \to d_2 = c_2 +\frac{u_1(y)}{W(y)}
\end{align}
Trovate queste possiamo riscrivere la soluzione particolare come
\begin{align}
	G_P(x,y) = \double{c_1 u_1 (x) + c_2 u_2 (x) \hfill x<y}{\left[  c_1  -\frac{u_2(y)}{W(y)} \right] u_1 (x) + \left[ c_2 +\frac{u_1(y)}{W(y)} \right] u_2 (x) \quad x>y} \spacer c_i,d_i = \double{costanti}{c_i(y), d_i(y)}
\end{align}
Possiamo riscrivere la seconda come
\begin{align}
&\left[  c_1  -\frac{u_2(y)}{W(y)} \right] u_1 (x) + \left[ c_2 +\frac{u_1(y)}{W(y)} \right] u_2 (x)\firstpassage
&c_1 u_1 (x) + c_2 u_2 (x) + \frac{u_1(y)u_2(x) - u_1(x)u_2(y)}{W(y)}
\end{align}
Conforntando con il caso $x<y$ e impiegando il funzionale a gradino $\theta(x-y)$ possiamo riscrivere la soluzuione particolare nella seguente forma compatta
\begin{align}
	G_P(x,y)= c_1 u_1 (x) + c_2 u_2 (x) + \theta(x-y) \frac{u_1(y)u_2(x) - u_1(x)u_2(y)}{W(y)}
\end{align}
Riportiamo due casi noti che ci saranno utili più avanti
\begin{enumerate}
	\item \textbf{Funzione di Green avanzata}
		\begin{align}
			&G(x,y) \equiv 0 \quad x<y 
		\end{align}
	\item \textbf{Funzione di Green ritardata}
		\begin{align}
			&G(x,y) \equiv 0 \quad x>y 
		\end{align}
\end{enumerate}
\newpage

\subsubsection{Teorema di Green}




\newpage

\subsection{Messaggio dall'autore}

Riporto qui, sotto richiesta dell'autore di questi esercizi, un suo messaggio:

\textit{Io sottoscritto Adriano Chialastri non mi riterrò responsabile di qualsiasi danno a cose o persone causato da questi esercizi, in quanto non svolti nel pieno esercizio delle mie facoltà mentali a causa nella prolungata permanenza presso il luogo correntemente noto come Sogene.
}

\subsection{Esempio 1: Problema di Cauchy (Adriano Chialastri) \label{canon}}
Risolvere il seguente problema di Cauchy
\begin{align}
	\triple{x^2 \ddot{y} - 2x \dot{y} + 2y = -x}{\dot{y}(1)=0}{y(1)=0}
\end{align}
Iniziamo passando in \textbf{forma canonica}
\begin{align}
	&y(x) = A(x)v(x)	\firstpassage
	&x^2 (\ddot{A}(x)v(x) + 2\dot{A}(x)\dot{v}(x) + A \ddot{v}(x)) - 2x (\dot{A}(x)v(x) + A \dot{v}(x)) + 2A(x)v(x) = -x \nextpassage
	&Ax^2 \ddot{v}(x) + 2(\dot{A}(x) x^2 - A(x)x) \dot{v}(x) + (\ddot{A}(x) x^2 - 2x\dot{A}(x) +2A)v(x) = -x	
\end{align}
Per poter fare il passaggio dobbiamo imporre che
\begin{align}
	&\dot{A}(x) x^2 - A(x)x = 0 \firstpassage
	&\dot{A}(x) x = A(x) \nextpassage
	&A(x) = x  
\end{align}
Da cui otteniamo che
\begin{align}
	&x^3 \ddot{v}(x) + (-\cancel{2x} +\cancel{2x})v(x) = -x	\firstpassage
	&\ddot{v}(x) = -\frac{1}{x^2} \spacer y(x) = x v(x)
\end{align}
Il nostro problema di Cauchy in forma canonica è quindi
\begin{align}
	\triple{\ddot{v}(x) = -\frac{1}{x^2}}{\dot{v}(1)=0}{v(1)=0}
\end{align}
Andiamo ora a risolvere l'\textbf{omogenea}
\begin{align}
	&\ddot{v}(x) = 0 \firstpassage
	&v_{om}(x) = Ax + B  \label{kekwdiff1}\firstpassage
	&\double{\dot{v}_{om}(1)=A=0}{v_{om}(1)=A+B=0} \to \double{A=0}{B=0} \to \text{l'operatore differenziale è invertibile} 
\end{align}
Dalla \ref{kekwdiff1} vediamo come
\begin{align}
	&v_{om}(x) = A v_1(x) + B v_2(x) \nextpassage
	&v_{om_1}(x) = x \spacer v_{om_2}(x) = 1
\end{align}
Calcoliamo il Wronskiano dell'equazione
\begin{align}
	W(v_1,v_2) = \begin{vmatrix}
		x && 1 \\
		1 && 0
	\end{vmatrix} = -1
\end{align}
E andiamo quindi a calcolare la \textbf{Funzione di Green}
\begin{align}
	G(x,y) = - \delta (x-y) (y-x) = (x-y) \delta(x-y)
\end{align}
Da cui otteniamo
\begin{align}
	v(x) &= \int_{1}^x dy \;G(x,y) \left(-\frac{1}{y^2}\right) = \nonumber \\
		 &= \int_{1}^x dy \; \delta (x-y) (y-x) \cdot \frac{1}{y^2}= \nonumber \\
		 &= \int_{1}^x dy \; \frac{y-x}{y^2}= \nonumber \\
		 &= \int_{1}^x dy \; \left(\frac{1}{y} - \frac{x}{y^2}\right)= \nonumber \\
		 &= \left[ \ln(y) + \frac{x}{y} \right]_1^x = \ln(x) +1 -x \nextpassage
	y(x) &= xv(x) = x\ln(x) - x^2 +x
\end{align}
Verifichiamo la validità della soluzione
\begin{align}
	&\dot{y}(x) = 2 - 2x + \ln(x)\\
	&\ddot{y}(x) = -2 + \frac{1}{x} \firstpassage
	&\double{\dot{y}(1) = 0}{y(1) = 0} \to \double{0 -\cancel{1} +\cancel{1} =0}{\cancel{2}-\cancel{2} + 0 =0}\\
	&x^2 \ddot{y} - 2x \dot{y} + 2y = -x \firstpassage
	&x^2 \left( -2 + \frac{1}{x} \right) -2x \left( 2 - 2x + \ln(x) \right) + 2 \left( x\ln(x) - x^2 +x \right) = -x \nextpassage
	& -\cancel{2x^2} + x - 4x + \cancel{4x^2} - 2x\ln(x) + 2x\ln(x) -\cancel{2x^2} +2x = -x \nextpassage
	& + x - 4x - \cancel{2x\ln(x)} + \cancel{2x\ln(x)} +2x = -x \nextpassage
	& -x = -x
\end{align}
E quindi l'esercizio è verificato.

\newpage

\subsection{Esempio 2: Omogenea di Eulero (Adriano Chialastri)}
Riprendiamo il precedente problema di Cauchy
\begin{align}
	\triple{x^2 \ddot{y} - 2x \dot{y} + 2y = -x}{\dot{y}(1)=0}{y(1)=0}
\end{align}
Partiamo dall'\textbf{omogenea}
\begin{align}
	&x^2 \ddot{y} - 2x\dot{y} + 2y = 0 \firstpassage
	&\ddot{y} - \frac{2}{x}\dot{y} + \frac{2}{x^2}y = 0
\end{align}
L'omogena scritta in questa forma risulta essre un'\textbf{equazione di Eulero}. Per risolverla poniamo quindi l'\textbf{Ansatz} $y=x^\alpha$ dove otteniamo
\begin{align}
	&\alpha(\alpha -1) x^{\alpha-2} - 2 \alpha x^{\alpha-2} + 2x^{\alpha-2} = 0 \firstpassage
	&\alpha^2 - \alpha - 2\alpha +2 =0 \nextpassage
	&\alpha^2 - 3\alpha +2 =0 \nextpassage
	&\alpha_{1,2} = \frac{3 \pm \sqrt{9-8}}{2} = \frac{3\pm 1}{2} = \double{2}{1} \nextpassage
	&y_{{om}_1}(x) = x \spacer y_{{om}_2}(x) = x^2 \nextpassage
	&y_{om}(x) = Ax + Bx^2   
\end{align}
Andiamo ora a cercare una \textbf{soluzione particolare} col \textbf{metodo delle costanti}. Poniamo
\begin{align}
	&y_p(x) = A(x)x + B(x)x^2 \firstpassage
	&\dot{y_p}(x)  = \dot{A}(x)x+ A(x) + \dot{B}(x) x^2 + 2B(x)x\\
	&\ddot{y_p}(x) = \ddot{A}(x)x+ 2\dot{A}(x) + \ddot{B}(x) x^2 + 4\dot{B}(x)x + 2B  \firstpassage
	&x^2 (\ddot{A}(x)x+ 2\dot{A}(x) + \ddot{B}(x) x^2 + 4\dot{B}(x)x + 2B) -2 (\dot{A}(x)x+ A(x) + \dot{B}(x) x^2 + 2B(x)x) + \nonumber \\
	&+2(A(x)x + B(x)x^2) = -x  \nextpassage
	&\dots \nextpassage
	&x^3 \ddot{A}(x) + x^4 \ddot{B}(x) + 2x^3 \dot{B}(x) = -x
\end{align}
Da cui otteniamo
\begin{align}
	&y(x) = y_p(x) + y_{om}(x) = y_p(x) + Ax + Bx^2 \firstpassage
	&\double{\dot{y}(1) = \dot{y}_p(1) + A + 2B = 0}{y(1) = y_p(1) + A + B=0}
\end{align}

\newpage

\subsection{Esempio 3: Problema di Sturm-Liouville (Adriano Chialastri)}

Risolvere il seguente problema di Sturm-Liouville
\begin{align}
	\triple{x \ddot{u}(x) + 2 \dot{u}(x) + 4xu(x) = 4}{u \left( \frac{\pi}{4}\right)=0}{u \left( \frac{\pi}{2}\right)=0}
\end{align}
Iniziamo passando il problema in \textbf{forma canonica}
\begin{align}
	&u(x) = A(x)v(x) \firstpassage
	&x(\ddot{A}(x)v(x) + 2\dot{A}(x) \dot{v}(x) + A(x) \ddot{v}(x)) + 2(\dot{A}(x)v(x) + A(x) \dot{v}(x)) + 4xA(x)v(x) = 4 \nextpassage
	&A(x)x \cdot \ddot{v}(x) + 2(\dot{A}(x)x + A(x)) \dot{v}(x) + (x \ddot{A}(x) + 2 \dot{A}(x) + 4xA(x)) v(x) = 4 \nextpassage
	&\dot{A}(x)x + A(x)=0 \to \ln(A(x)) = -\ln(x) \to A(x)= x^{-1}\nextpassage
	&\frac{1}{\cancel{x}} \cancel{x}\cdot \ddot{v}(x) + \left( \cancel{\cancel{x} \frac{2}{x^{\cancel{3}^2}}} - \cancel{\frac{2}{x^2}} + 4\cancel{x} \frac{1}{\cancel{x}} \right)v(x) = 4 \nextpassage
	&\triple{\ddot{v}(x) + 4v(x)= 4}{v \left( \frac{\pi}{4}\right)=0}{v \left( \frac{\pi}{2}\right)=0} \label{kekwdiff2}
\end{align}
Andiamo ora a risolvere l'\textbf{omogenea associata} dell'equazione in forma canonica
\begin{align}
	&\ddot{v}(x) + 4v(x) = 0 \firstpassage
	&\text{Ansatz } v(x)= e^{\alpha x} \nextpassage
	&\alpha^2 e^{\alpha} + 4 e^{\alpha} = 0 \nextpassage
	&\alpha^2 = -4 \to \alpha_{1,2} = \pm 2i \nextpassage
	&v_{{om}_1}(x) = e^{21x} \spacer v_{{om}_2}(x) = e^{-2ix} \nextpassage
	&v_{{om}}(x) = Ae^{2ix} + Be^{-2ix}
\end{align}
Dalla \ref{kekwdiff2} ricaviamo il nostro \textbf{operatore differenziale}
\begin{align}
	\mathcal{L}_x^c = \frac{d^2}{dx^2} + 4
\end{align}
Andiamo ora a verificare se è invertibile o meno, in base alle nostre Boundary Conditions:
\begin{align}
	&\double{v_{om}\left( \frac{\pi}{4}\right) =& Ae^{i\frac{\pi}{2}} + Be^{-i\frac{\pi}{2}}=iA-iB = 0}{v_{om}\left( \frac{\pi}{2}\right) =& Ae^{i\pi} + Be^{-i\pi} =-A -B = 0} \to \double{A-B=0}{A+B=0} \to \double{A=0}{B=0}\firstpassage
	&\mathcal{L}_x^c v=0 \leftrightarrow v=0 \quad \to \text{L'operatore è invertibile}
\end{align}
Andiamo ora a trovare due soluzioni dell'omogenea che soddisfino le BC
\begin{align}
	&v_A(x) \left( \frac{\pi}{4}\right) =0 \to iA - iB = 0 \to A=B \to v_A(x) = A(e^{2ix} + e^{-2ix}) \\
	&v_B(x) \left( \frac{\pi}{2}\right) =0 \to A+b = \to A=-B \to v_B(x) = A(e^{2ix} - e^{-2ix})
\end{align}
Siccome non abbiamo condizioni su A, la scegliamo comoda a seconda del caso, possiamo quindi scrivere
\begin{align}
	&v_A(x) = \frac{1}{2}(e^{2ix} + e^{-2ix}) = \cos(2x) \\
	&v_B(x) = \frac{1}{2i}(e^{2ix} - e^{-2ix}) = \sin(2x)\firstpassage
	& v_{om}(x) = \tilde{A} \cos(2x) + \tilde{B} \sin(2x) 
\end{align}

Andiamo ora a calcolare le \textbf{soluzioni fondamentali dell'equazione}
\begin{align}
	&W(v_A,v_B) = \begin{vmatrix}
		\cos(2x) && \sin(2x)\\
		-2\cos(2x) && 2 \cos(2x)
	\end{vmatrix} = \dots = 2\\
	&\tilde{G}(x,y) = \tilde{A} \cos(2x) + \tilde{B} \sin(2x) + \frac{\delta (x-y)}{2} \left[ \cos(2y) \sin(2x) - \cos(2x) \sin(2y) \right]\\
	&\mathcal{L}_x^c \tilde{G}(x,y) = \delta (x-y)
\end{align}

Per questo problema di SL, la \textbf{funzione di Green} sarà quindi data da
\begin{align}
	G(x,y) = \double{\frac{\cos(2x) \sin(2y)}{2} \quad x<y}{\frac{\cos(2y) \sin(2x)}{2} \quad x>y}
\end{align}
Da cui otteniamo
\begin{align}
	v(x) &= \int_{\frac{\pi}{4}}^{\frac{\pi}{2}} dy \; G(x,y) \cdot 4 = \nonumber \\
	&= 2\int_{\frac{\pi}{4}}^{x} dy \;\cos(2x) \sin(2y)+ 2\int_{x}^{\frac{\pi}{2}} dy \; \cos(2y) \sin(2x) = ... = \nonumber\\
	&=1 - \sin(2x) + \cos(2x) \firstpassage
	u(x) &= \frac{v(x)}{x} = \frac{1 - \sin(2x) + \cos(2x)}{x}
\end{align}

\bigskip

VERIFICA DA COPIARE QUANDO SONO MENO ABBOTTATO (pag 7 del pdf)

\newpage

\subsection{Esempio 4: Problema di Cauchy non lineare (Adriano Chialastri)}
Risolvere il seguente \textbf{problema di Cauchy}
\begin{align}
	\triple{x^2 \ddot{f}(x) + 3x \dot{f}(x) + f(x) = 0}{\dot{f}(1)=0}{f(1) =1}
\end{align}
Affinché l'operatore sia lineare dobbiamo avere BC omogenee, che non è il nostro caso. Dobbiamo quindi \textbf{linearizzare} il problema. Andiamo a definire
\begin{align}
	&u(x) = f(x) + mx + q\firstpassage
	&\dot{u}(x) = \dot{f}(x) + m
\end{align}	
Da cui otteniamo, imponendo che siamo omogenee le BC di $u(x)$
\begin{align}
	&\double{\dot{u}(1) =& \dot{f}(1) + m =0}{u(1) =& f(1) + m + q=0} \to \double{0 + m=0 }{1 + m + q=0} \to \double{m=& 0}{q=& -1} \firstpassage
	&u(x) = f(x) -1 \quad \leftrightarrow \quad f(x) = u(x)+1
\end{align}	
Il nostro problema diventa quindi
\begin{align}
	\triple{x^2 \ddot{u}(x) + 3x \dot{u}(x) + u(x) = -1}{\dot{u}(1)=0}{u(1) = 0}
\end{align}
Riscriviamo il problema in \textbf{forma canonica}
\begin{align}
	&u(x) = A(x)v(x)\firstpassage
	&\dot{u}(x) = \dot{A}(x) v(x) + A(x) \dot{v}(x)\nextpassage
	&\ddot{u}(x) = \ddot{A}(x) v(x) + 2\dot{A}(x) \dot{v}(x) + A(x) \ddot{v}(x)\nextpassage
	&[A(x) x^2] \ddot{v}(x) + [2\dot{A}(x) x^2 + 3A(x)x ]\dot{v}(x) + [\ddot{A}(x)x^2 + 3\dot{A}(x)x + A(x)]v(x) = -1 \nextpassage
	&2\dot{A}(x) x^2 + 3A(x)x=0 \to \dot{A}(x) = -\frac{3}{2}x^{-1}A(x) \to \ln(A(x)) = -\frac{3}{2} \ln(x)  \to A(x)= x^{-\frac{3}{2}} \nextpassage
	&\dot{A}(x) = -\frac{3}{2}x^{-\frac{5}{2}} \spacer \dot{A}(x) = -\frac{15}{2}x^{-\frac{7}{2}} \nextpassage
	&[x^{-\frac{3}{2}} x^2] \ddot{v}(x) + \left[x^2\left( \frac{15}{4}x^{-\frac{7}{2}} \right) + 3x \left( -\frac{3}{2} x^{-\frac{5}{2}} \right) + x^{-\frac{3}{2}} \right] v(x) = -1 \nextpassage
	&[x^{-\frac{3}{2}} x^2] \ddot{v}(x) + \left[\frac{15}{4} -\frac{9}{2} +1 \right] x^{-\frac{3}{2}} v(x) = -1 \nextpassage
	&x^{\frac{1}{2}} \ddot{v}(x) + \frac{1}{4}x^{-\frac{3}{2}} v(x) = -1 \nextpassage
	&\triple{\ddot{v}(x) + \frac{1}{4x^2} \dot{v}(x)  = -x ^{-\frac{1}{2}}}{\dot{v}(1)=0}{v(1) = 0}
\end{align}
Andiamo ora a risolvere l'\textbf{omogenea associata} dell'equazione in forma canonica
\begin{align}
	\ddot{v}(x) + \frac{1}{4x^2} \dot{v}(x)  = 0
\end{align}
Siccome è un'equazione di Eulero usiamo l'Ansatz $v(x) = x^\alpha$ e otteniamo
\begin{align}
	&\alpha(\alpha-1) + \frac{1}{4} = 0 \to \alpha_{1,2} = \frac{1\pm \sqrt{1-1}}{2} = \frac{1}{2} \firstpassage
	&v_{{om}_1}(x) = x^{\frac{1}{2}} \spacer v_{{om}_2}(x) = x^{\frac{1}{2}} \ln(x) \nextpassage
	&v_{om}(x) = Ax^{\frac{1}{2}} + Bx^{\frac{1}{2}}\ln(x) = x^{\frac{1}{2}}(A + B\ln(x)) \firstpassage
	&W(v_1,v_2) = \begin{vmatrix}
		x^{\frac{1}{2}} && x^{\frac{1}{2}}\ln(x)\\
		\frac{1}{2} x^{-\frac{1}{2}} && \frac{1}{2} x^{-\frac{1}{2}} \ln(x) + x^{-\frac{1}{2}}
	\end{vmatrix} = \dots = 1
\end{align}
La soluzione generica del PdC sarà data da
\begin{align}
	\tilde{G}(x,y) = \alpha v_1(x) + \beta v_2(x) + \theta(x-y)\left[ \frac{v_1(y) v_2(x) -v_1(x)v_2(y)}{W} \right] 
\end{align}
Siccome siamo in presenza di un PdC, a noi serve la \textbf{funzione di Green avanzata}, e quindi dobbiamo porre $\alpha=\beta=0$
\begin{align}
	G(x,y) &= \theta(x-y)\left[ \frac{v_1(y) v_2(x) -v_1(x)v_2(y)}{W} \right] = \nonumber \\
		   &= \theta(x-y)\left[ y^{\frac{1}{2}} x^{\frac{1}{2}} \ln(x) -x^{\frac{1}{2}}y^{\frac{1}{2}} \ln(y) \right] 
\end{align}
Possiamo quindi andare a calcolare la soluzione
\begin{align}
	v(x) &= \int_{1}^{x} dy \; G(x,y) \left( -y^{-\frac{1}{2}} \right) + \cancel{BC} = \nonumber\\
		 &= -\int_{1}^{x} dy \; \delta(x-y)\left[ x^{\frac{1}{2}} \ln(x) -x^{\frac{1}{2}} \ln(y) \right]= \nonumber\\
		 &= -x^{\frac{1}{2}}\int_{1}^{x} dy \; \left[ \ln(x) - \ln(y) \right] = \nonumber\\
		 &= -x^{\frac{1}{2}} \left[\ln(x)\int_{1}^{x} dy + \int_{1}^{x} dy\; \ln(y)\right] = \nonumber\\
		 &= -x^{\frac{1}{2}} \left[\ln(x)(x-1) - [y(\ln(y) -1)]_1^x \right] = \nonumber\\
		 &= -x^{\frac{1}{2}} \left[\ln(x)(x-1) - x(\ln(x) -1) -1 \right] = \nonumber\\
		 &= -x^{\frac{1}{2}} \left[\cancel{x\ln(x)} -\ln(x) - \cancel{x\ln(x)} +x -1 \right] \nonumber\\
		 &= x^{\frac{1}{2}} \left[1 +\ln(x)-x \right] \firstpassage
	u(x) &= x^{-\frac{3}{2}} v(x) = \frac{1}{x} \left[1 +\ln(x)-x \right] \firstpassage
	f(x) &= u(x) + 1 = \frac{1 + \ln(x)}{x}
\end{align}
Verifichiamo la bontà della soluzione trovata
\begin{align}
	&\dot{f}(x) = -\frac{\ln(x)}{x^2} \spacer \ddot{f}(x) = \frac{2\ln(x) -1}{x^3} \firstpassage
	&\double{\dot{f}(1) = -\frac{\ln(1)}{1}=0 }{f(1) = \frac{1 + \ln(1)}{1} = 1} \quad \leftarrow \quad \text{le BC sono rispettate}\\
	&\cancel{x^2}\frac{2\ln(x) -1}{x^{\cancel{3}}} - 3\cancel{x} \frac{\ln(x)}{x^{\cancel{2}}} + \frac{1+\ln(x)}{x} = 0 \to \frac{1}{x}[\cancel{2\ln(x)} -\cancel{3\ln(x)} + \cancel{\ln(x)} -\cancel{1} +\cancel{1}] = 0 \quad \leftarrow \quad \text{ok}
\end{align}
\newpage 
\subsection{Esempio 5: prova d'esame (Alessandro Marcelli)}
Sia il seguente problema di Cauchy
\begin{align}
	\triple{x^2 \ddot{f}(x) - x \dot{f}(x) + f(x) = x^2}{\dot{f}(1) = 0}{f(1)=0}
\end{align}
Si richede di calcolare
\begin{enumerate}
	\item la funzione di Green del problema
	\item la soluzione della non omogenea col metodo di Green per $x\geq 1$
\end{enumerate}
Iniziamo passando l'equazione in \textbf{forma omogenea} (omettiamo la dipendenza da x per snellire la notazione)
\begin{align}
	&f(x) = A(x)v(x) \to f=Av\firstpassage
	&\dot{f} = \dot{A}v + A \dot{v} \to \ddot{f} = \ddot{A}v + 2 \dot{A}\dot{v} + A \ddot{v} \nextpassage
	&x^2 \ddot{A}v + 2x^2 \dot{A}\dot{v} +x^2 A \ddot{v} -x\dot{A}v -x A \dot{v} + Av = x^2\nextpassage
	& x^2 \ddot{A}v +x(2x \dot{A} - A) \dot{v} + (x^2 \ddot{A} - x \dot{A} + A)v = x^2\nextpassage
	&2x \dot{A} - A = 0 \to \frac{dA}{dx} = \frac{A}{2x} \to \int dA \; \frac{1}{A} = \int dx \; \frac{1}{2x} = \ln(A) = \ln(\sqrt{x}) \nextpassage
	&A = \sqrt{x} = x^{\frac{1}{2}} \to \dot{A} = \frac{1}{2}x^{-\frac{1}{2}} \to \ddot{A} = -\frac{1}{4}x^{-\frac{3}{2}} \nextpassage
	&x^{\frac{5}{2}} \ddot{v} + \frac{1}{4} x^{\frac{1}{2}} = x^2\nextpassage
	&\ddot{v} +\frac{1}{4} x^{-2} v = x^{-\frac{1}{2}}
\end{align}
Cominciamo col risolvere l'omogenea:
\begin{align}
	\ddot{v} +\frac{1}{4} x^{-2} v = 0 
\end{align}
Siamo in presenza di un'equazione di eulero, visto che
\begin{align}
	&\ddot{v}  + \frac{a}{x} \dot{v} + \frac{b}{x^2} u = 0 \spacer a=0 \; , \; b= \frac{1}{4} \firstpassage
	&	\ddot{v} +\frac{1}{4x^2} v = 0 
\end{align}
Possiamo quindi fare l'Ansatz $v = x^\alpha$, ottenendo
\begin{align}
	\alpha (\alpha -1) x^{\alpha -2} +\frac{1}{4} x^{\alpha -2} = 0 \to \alpha^2 - \alpha + \frac{1}{4} = 0 
\end{align}
Siccome $\Delta = 0$ avremo una sola soluzione in $\alpha = \frac{1}{2}$. Ci conviene quindi trasformare l'equazione con il cambio di variabile $x=e^{t}$, dove otteniamo
\begin{align}
	&\frac{dv}{dx} = e^{-t}\frac{dv}{dt} \to \frac{d^2v}{dx^2} = e^{-2t}\frac{d^2 v}{dt^2} - e^{-2t}\frac{dv}{dt} \firstpassage
	&e^{-2t} \left[ \frac{d^2 v}{dt^2} - (a-1) \frac{dv}{dt} + bv\right] = 0
\end{align}
nel nostro caso $a=0$, quindi
\begin{align}
	\frac{d^2 v}{dt^2} + \frac{dv}{dt} + \frac{v}{4}= 0
\end{align}
Il nostro  Ansatz diventa $v = x^\alpha = (e^t)^\alpha  = e^{\alpha t}$ eotteniamo le due soluzioni
\begin{align}
	v_1 = e^{\alpha t} \spacer v_2 = t e^{\alpha t} 
\end{align}
Che, tornando nella variabile originale diventano
\begin{align}
	v_1 = x^\alpha \spacer v_2 = x^\alpha \ln(x) 
\end{align}
Nel nostro caso saranno quindi
\begin{align}
	&v_1 = x^\frac{1}{2} \spacer v_2 = x^\frac{1}{2} \ln(x) \firstpassage
	&v_{om} = x^\frac{1}{2}(A + B\ln(x)) \firstpassage
	&\dot{v}_{om} = \frac{1}{2}x^{-\frac{1}{2}}A + B \left( x^\frac{1}{2} \cdot \frac{1}{x} + \frac{1}{2}x^{-\frac{1}{2}} \right) = \frac{1}{2}x^{-\frac{1}{2}} A + \frac{3}{2}x^{-\frac{1}{2}} B = \frac{1}{2}x^{-\frac{1}{2}} (A + 3B) \nextpassage
	&\double{v_{om}(1) = 0}{\dot{v}_{om}(1) = 0} \to \double{1^\frac{1}{2}(A + B\ln(1)) = 0 \to& A = 0}{\frac{1}{2}\cdot 1^{-\frac{1}{2}} (A + 3B)=0 \to& B = 0}
\end{align}

La soluzione omogenea è nulla, e posso quindi usare il \textbf{teorema di Green}.
 
Iniziamo calcolando il \textbf{Wronskiano}
\begin{align}
	 W = \begin{vmatrix}
	x^\frac{1}{2} && x^\frac{1}{2} \ln(x) \\
	\frac{1}{2}x^{-\frac{1}{2}} && \frac{1}{2}x^{-\frac{1}{2}} \ln(x) + x^{-\frac{1}{2}}
\end{vmatrix} = \frac{1}{2} \ln(x) + 1 - \frac{1}{2}\ln(x) = 1
\end{align}
Andiamo ora calcolare la \textbf{funzione di Green avanzata} del problema
\begin{align}
	G(x,y) &= \theta(x-y)\left[ \frac{v_1(y) v_2(x) -v_1(x)v_2(y)}{W} \right] = \nonumber \\
	&= \theta(x-y)\left[y^\frac{1}{2} x^\frac{1}{2} \ln(x) - x^\frac{1}{2} y^\frac{1}{2}\ln(y) \right] 
\end{align}
Possiamo ora andare a calcolare la soluzione particolare
\begin{align}
	v(x) &= \int_{1}^{x} dy \; G(x,y) \cdot y^2 = \nonumber\\
		 &= \int_{1}^{x} dy \; \left[y^\frac{1}{2} x^\frac{1}{2} \ln(x) - x^\frac{1}{2} y^\frac{1}{2}\ln(y) \right] y^2 = \nonumber\\
		 &= x^\frac{1}{2} \ln(x)\int_{1}^{x} dy \;y^\frac{5}{2} - x^\frac{1}{2}\int_{1}^{x} dy \; y^\frac{5}{2}\ln(y)  = \nonumber\\
		 &= x^\frac{1}{2} \ln(x) \left.\frac{2}{7}y^{\frac{7}{2}} \right|_1^x + 
\end{align}

\newpage
\section{Equazioni Differenziali Complesse}

N'aggio fatto in tempo, tanto Bufalini le ha scritte bene su Latex.