\chapter{Equazioni Differenziali}

\section{Equazioni Differenziali Reali (Adriano Chialastri, Alessandro Marcelli)}

\subsection{Messaggio dall'autore}

Riporto qui, sotto richiesta dell'autore di questi esercizi, un suo messaggio:

\textit{Io sottoscritto Adriano Chialastri non mi riterrò responsabile di qualsiasi danno a cose o persone causato da questi esercizi, in quanto non svolti nel pieno esercizio delle mie facoltà mentali a causa nella prolungata permanenza presso il luogo correntemente noto come Sogene.
}
\subsection{Esempio 1: Problema di Cauchy (Adriano Chialastri)}
Risolvere il seguente problema di Cauchy
\begin{align}
	\triple{x^2 \ddot{y} - 2x \dot{y} + 2y = -x}{\dot{y}(1)=0}{y(1)=0}
\end{align}
Iniziamo passando in \textbf{forma canonica}
\begin{align}
	&y(x) = A(x)v(x)	\firstpassage
	&x^2 (\ddot{A}(x)v(x) + 2\dot{A}(x)\dot{v}(x) + A \ddot{v}(x)) - 2x (\dot{A}(x)v(x) + A \dot{v}(x)) + 2A(x)v(x) = -x \nextpassage
	&Ax^2 \ddot{v}(x) + 2(\dot{A}(x) x^2 - A(x)x) \dot{v}(x) + (\ddot{A}(x) x^2 - 2x\dot{A}(x) +2A)v(x) = -x	
\end{align}
Per poter fare il passaggio dobbiamo imporre che
\begin{align}
	&\dot{A}(x) x^2 - A(x)x = 0 \firstpassage
	&\dot{A}(x) x = A(x) \nextpassage
	&A(x) = x  
\end{align}
Da cui otteniamo che
\begin{align}
	&x^3 \ddot{v}(x) + (-\cancel{2x} +\cancel{2x})v(x) = -x	\firstpassage
	&\ddot{v}(x) = -\frac{1}{x^2} \spacer y(x) = x v(x)
\end{align}
Il nostro problema di Cauchy in forma canonica è quindi
\begin{align}
	\triple{\ddot{v}(x) = -\frac{1}{x^2}}{\dot{v}(1)=0}{v(1)=0}
\end{align}
Andiamo ora a risolvere l'\textbf{omogenea}
\begin{align}
	&\ddot{v}(x) = 0 \firstpassage
	&v_{om}(x) = Ax + B  \label{kekwdiff1}\firstpassage
	&\double{\dot{v}_{om}(1)=A=0}{v_{om}(1)=A+B=0} \to \double{A=0}{B=0} \to \text{l'operatore differenziale è invertibile} 
\end{align}
Dalla \ref{kekwdiff1} vediamo come
\begin{align}
	&v_{om}(x) = A v_1(x) + B v_2(x) \nextpassage
	&v_{om_1}(x) = x \spacer v_{om_2}(x) = 1
\end{align}
Calcoliamo il Wronskiano dell'equazione
\begin{align}
	W(v_1,v_2) = \begin{vmatrix}
		x && 1 \\
		1 && 0
	\end{vmatrix} = -1
\end{align}
E andiamo quindi a calcolare la \textbf{Funzione di Green}
\begin{align}
	G(x,y) = - \delta (x-y) (y-x) = (x-y) \delta(x-y)
\end{align}
Da cui otteniamo
\begin{align}
	v(x) &= \int_{1}^x dy \;G(x,y) \left(-\frac{1}{y^2}\right) = \nonumber \\
		 &= \int_{1}^x dy \; \delta (x-y) (y-x) \cdot \frac{1}{y^2}= \nonumber \\
		 &= \int_{1}^x dy \; \frac{y-x}{y^2}= \nonumber \\
		 &= \int_{1}^x dy \; \left(\frac{1}{y} - \frac{x}{y^2}\right)= \nonumber \\
		 &= \left[ \ln(y) + \frac{x}{y} \right]_1^x = \ln(x) +1 -x \nextpassage
	y(x) &= xv(x) = x\ln(x) - x^2 +x
\end{align}
Verifichiamo la validità della soluzione
\begin{align}
	&\dot{y}(x) = 2 - 2x + \ln(x)\\
	&\ddot{y}(x) = -2 + \frac{1}{x} \firstpassage
	&\double{\dot{y}(1) = 0}{y(1) = 0} \to \double{0 -\cancel{1} +\cancel{1} =0}{\cancel{2}-\cancel{2} + 0 =0}\\
	&x^2 \ddot{y} - 2x \dot{y} + 2y = -x \firstpassage
	&x^2 \left( -2 + \frac{1}{x} \right) -2x \left( 2 - 2x + \ln(x) \right) + 2 \left( x\ln(x) - x^2 +x \right) = -x \nextpassage
	& -\cancel{2x^2} + x - 4x + \cancel{4x^2} - 2x\ln(x) + 2x\ln(x) -\cancel{2x^2} +2x = -x \nextpassage
	& + x - 4x - \cancel{2x\ln(x)} + \cancel{2x\ln(x)} +2x = -x \nextpassage
	& -x = -x
\end{align}
E quindi l'esercizio è verificato.

\newpage

\subsection{Esempio 2: Omogenea di Eulero (Adriano Chialastri)}
Riprendiamo il precedente problema di Cauchy
\begin{align}
	\triple{x^2 \ddot{y} - 2x \dot{y} + 2y = -x}{\dot{y}(1)=0}{y(1)=0}
\end{align}
Partiamo dall'\textbf{omogenea}
\begin{align}
	&x^2 \ddot{y} - 2x\dot{y} + 2y = 0 \firstpassage
	&\ddot{y} - \frac{2}{x}\dot{y} + \frac{2}{x^2}y = 0
\end{align}
L'omogena scritta in questa forma risulta essre un'\textbf{equazione di Eulero}. Per risolverla poniamo quindi l'\textbf{Ansatz} $y=x^\alpha$ dove otteniamo
\begin{align}
	&\alpha(\alpha -1) x^{\alpha-2} - 2 \alpha x^{\alpha-2} + 2x^{\alpha-2} = 0 \firstpassage
	&\alpha^2 - \alpha - 2\alpha +2 =0 \nextpassage
	&\alpha^2 - 3\alpha +2 =0 \nextpassage
	&\alpha_{1,2} = \frac{3 \pm \sqrt{9-8}}{2} = \frac{3\pm 1}{2} = \double{2}{1} \nextpassage
	&y_{{om}_1}(x) = x \spacer y_{{om}_2}(x) = x^2 \nextpassage
	&y_{om}(x) = Ax + Bx^2   
\end{align}
Andiamo ora a cercare una \textbf{soluzione particolare} col \textbf{metodo delle costanti}. Poniamo
\begin{align}
	&y_p(x) = A(x)x + B(x)x^2 \firstpassage
	&\dot{y_p}(x)  = \dot{A}(x)x+ A(x) + \dot{B}(x) x^2 + 2B(x)x\\
	&\ddot{y_p}(x) = \ddot{A}(x)x+ 2\dot{A}(x) + \ddot{B}(x) x^2 + 4\dot{B}(x)x + 2B  \firstpassage
	&x^2 (\ddot{A}(x)x+ 2\dot{A}(x) + \ddot{B}(x) x^2 + 4\dot{B}(x)x + 2B) -2 (\dot{A}(x)x+ A(x) + \dot{B}(x) x^2 + 2B(x)x) + \nonumber \\
	&+2(A(x)x + B(x)x^2) = -x  \nextpassage
	&\dots \nextpassage
	&x^3 \ddot{A}(x) + x^4 \ddot{B}(x) + 2x^3 \dot{B}(x) = -x
\end{align}
Da cui otteniamo
\begin{align}
	&y(x) = y_p(x) + y_{om}(x) = y_p(x) + Ax + Bx^2 \firstpassage
	&\double{\dot{y}(1) = \dot{y}_p(1) + A + 2B = 0}{y(1) = y_p(1) + A + B=0}
\end{align}

\newpage

\subsection{Esempio 3: Problema di Sturm-Liouville (Adriano Chialastri)}

Risolvere il seguente problema di Sturm-Liouville
\begin{align}
	\triple{x \ddot{u}(x) + 2 \dot{u}(x) + 4xu(x) = 4}{u \left( \frac{\pi}{4}\right)=0}{u \left( \frac{\pi}{2}\right)=0}
\end{align}
Iniziamo passando il problema in \textbf{forma canonica}
\begin{align}
	&u(x) = A(x)v(x) \firstpassage
	&x(\ddot{A}(x)v(x) + 2\dot{A}(x) \dot{v}(x) + A(x) \ddot{v}(x)) + 2(\dot{A}(x)v(x) + A(x) \dot{v}(x)) + 4xA(x)v(x) = 4 \nextpassage
	&A(x)x \cdot \ddot{v}(x) + 2(\dot{A}(x)x + A(x)) \dot{v}(x) + (x \ddot{A}(x) + 2 \dot{A}(x) + 4xA(x)) v(x) = 4 \nextpassage
	&\dot{A}(x)x + A(x)=0 \to \ln(A(x)) = -\ln(x) \to A(x)= x^{-1}\nextpassage
	&\frac{1}{\cancel{x}} \cancel{x}\cdot \ddot{v}(x) + \left( \cancel{\cancel{x} \frac{2}{x^{\cancel{3}^2}}} - \cancel{\frac{2}{x^2}} + 4\cancel{x} \frac{1}{\cancel{x}} \right)v(x) = 4 \nextpassage
	&\triple{\ddot{v}(x) + 4v(x)= 4}{v \left( \frac{\pi}{4}\right)=0}{v \left( \frac{\pi}{2}\right)=0} \label{kekwdiff2}
\end{align}
Andiamo ora a risolvere l'\textbf{omogenea associata} dell'equazione in forma canonica
\begin{align}
	&\ddot{v}(x) + 4v(x) = 0 \firstpassage
	&\text{Ansatz } v(x)= e^{\alpha x} \nextpassage
	&\alpha^2 e^{\alpha} + 4 e^{\alpha} = 0 \nextpassage
	&\alpha^2 = -4 \to \alpha_{1,2} = \pm 2i \nextpassage
	&v_{{om}_1}(x) = e^{21x} \spacer v_{{om}_2}(x) = e^{-2ix} \nextpassage
	&v_{{om}}(x) = Ae^{2ix} + Be^{-2ix}
\end{align}
Dalla \ref{kekwdiff2} ricaviamo il nostro \textbf{operatore differenziale}
\begin{align}
	\mathcal{L}_x^c = \frac{d^2}{dx^2} + 4
\end{align}
Andiamo ora a verificare se è invertibile o meno, in base alle nostre Boundary Conditions:
\begin{align}
	&\double{v_{om}\left( \frac{\pi}{4}\right) =& Ae^{i\frac{\pi}{2}} + Be^{-i\frac{\pi}{2}}=iA-iB = 0}{v_{om}\left( \frac{\pi}{2}\right) =& Ae^{i\pi} + Be^{-i\pi} =-A -B = 0} \to \double{A-B=0}{A+B=0} \to \double{A=0}{B=0}\firstpassage
	&\mathcal{L}_x^c v=0 \leftrightarrow v=0 \quad \to \text{L'operatore è invertibile}
\end{align}
Andiamo ora a trovare due soluzioni dell'omogenea che soddisfino le BC
\begin{align}
	&v_A(x) \left( \frac{\pi}{4}\right) =0 \to iA - iB = 0 \to A=B \to v_A(x) = A(e^{2ix} + e^{-2ix}) \\
	&v_B(x) \left( \frac{\pi}{2}\right) =0 \to A+b = \to A=-B \to v_B(x) = A(e^{2ix} - e^{-2ix})
\end{align}
Siccome non abbiamo condizioni su A, la scegliamo comoda a seconda del caso, possiamo quindi scrivere
\begin{align}
	&v_A(x) = \frac{1}{2}(e^{2ix} + e^{-2ix}) = \cos(2x) \\
	&v_B(x) = \frac{1}{2i}(e^{2ix} - e^{-2ix}) = \sin(2x)\firstpassage
	& v_{om}(x) = \tilde{A} \cos(2x) + \tilde{B} \sin(2x) 
\end{align}

Andiamo ora a calcolare le \textbf{soluzioni fondamentali dell'equazione}
\begin{align}
	&W(v_A,v_B) = \begin{vmatrix}
		\cos(2x) && \sin(2x)\\
		-2\cos(2x) && 2 \cos(2x)
	\end{vmatrix} = \dots = 2\\
	&\tilde{G}(x,y) = \tilde{A} \cos(2x) + \tilde{B} \sin(2x) + \frac{\delta (x-y)}{2} \left[ \cos(2y) \sin(2x) - \cos(2x) \sin(2y) \right]\\
	&\mathcal{L}_x^c \tilde{G}(x,y) = \delta (x-y)
\end{align}

Per questo problema di SL, la \textbf{funzione di Green} sarà quindi data da
\begin{align}
	G(x,y) = \double{\frac{\cos(2x) \sin(2y)}{2} \quad x<y}{\frac{\cos(2y) \sin(2x)}{2} \quad x>y}
\end{align}
Da cui otteniamo
\begin{align}
	v(x) &= \int_{\frac{\pi}{4}}^{\frac{\pi}{2}} dy \; G(x,y) \cdot 4 = \nonumber \\
	&= 2\int_{\frac{\pi}{4}}^{x} dy \;\cos(2x) \sin(2y)+ 2\int_{x}^{\frac{\pi}{2}} dy \; \cos(2y) \sin(2x) = ... = \nonumber\\
	&=1 - \sin(2x) + \cos(2x) \firstpassage
	u(x) &= \frac{v(x)}{x} = \frac{1 - \sin(2x) + \cos(2x)}{x}
\end{align}

\bigskip

VERIFICA DA COPIARE QUANDO SONO MENO ABBOTTATO (pag 7 del pdf)

\newpage

\subsection{Esempio 4: Problema di Cauchy non lineare (Adriano Chialastri)}
Risolvere il seguente \textbf{problema di Cauchy}
\begin{align}
	\triple{x^2 \ddot{f}(x) + 3x \dot{f}(x) + f(x) = 0}{\dot{f}(1)=0}{f(1) =1}
\end{align}
Affinché l'operatore sia lineare dobbiamo avere BC omogenee, che non è il nostro caso. Dobbiamo quindi \textbf{linearizzare} il problema. Andiamo a definire
\begin{align}
	&u(x) = f(x) + mx + q\firstpassage
	&\dot{u}(x) = \dot{f}(x) + m
\end{align}	
Da cui otteniamo, imponendo che siamo omogenee le BC di $u(x)$
\begin{align}
	&\double{\dot{u}(1) =& \dot{f}(1) + m =0}{u(1) =& f(1) + m + q=0} \to \double{0 + m=0 }{1 + m + q=0} \to \double{m=& 0}{q=& -1} \firstpassage
	&u(x) = f(x) -1 \quad \leftrightarrow \quad f(x) = u(x)+1
\end{align}	
Il nostro problema diventa quindi
\begin{align}
	\triple{x^2 \ddot{u}(x) + 3x \dot{u}(x) + u(x) = -1}{\dot{u}(1)=0}{u(1) = 0}
\end{align}
Riscriviamo il problema in \textbf{forma canonica}
\begin{align}
	&u(x) = A(x)v(x)\firstpassage
	&\dot{u}(x) = \dot{A}(x) v(x) + A(x) \dot{v}(x)\nextpassage
	&\ddot{u}(x) = \ddot{A}(x) v(x) + 2\dot{A}(x) \dot{v}(x) + A(x) \ddot{v}(x)\nextpassage
	&[A(x) x^2] \ddot{v}(x) + [2\dot{A}(x) x^2 + 3A(x)x ]\dot{v}(x) + [\ddot{A}(x)x^2 + 3\dot{A}(x)x + A(x)]v(x) = -1 \nextpassage
	&2\dot{A}(x) x^2 + 3A(x)x=0 \to \dot{A}(x) = -\frac{3}{2}x^{-1}A(x) \to \ln(A(x)) = -\frac{3}{2} \ln(x)  \to A(x)= x^{-\frac{3}{2}} \nextpassage
	&\dot{A}(x) = -\frac{3}{2}x^{-\frac{5}{2}} \spacer \dot{A}(x) = -\frac{15}{2}x^{-\frac{7}{2}} \nextpassage
	&[x^{-\frac{3}{2}} x^2] \ddot{v}(x) + \left[x^2\left( \frac{15}{4}x^{-\frac{7}{2}} \right) + 3x \left( -\frac{3}{2} x^{-\frac{5}{2}} \right) + x^{-\frac{3}{2}} \right] v(x) = -1 \nextpassage
	&[x^{-\frac{3}{2}} x^2] \ddot{v}(x) + \left[\frac{15}{4} -\frac{9}{2} +1 \right] x^{-\frac{3}{2}} v(x) = -1 \nextpassage
	&x^{\frac{1}{2}} \ddot{v}(x) + \frac{1}{4}x^{-\frac{3}{2}} v(x) = -1 \nextpassage
	&\triple{\ddot{v}(x) + \frac{1}{4x^2} \dot{v}(x)  = -x ^{-\frac{1}{2}}}{\dot{v}(1)=0}{v(1) = 0}
\end{align}
Andiamo ora a risolvere l'\textbf{omogenea associata} dell'equazione in forma canonica
\begin{align}
	\ddot{v}(x) + \frac{1}{4x^2} \dot{v}(x)  = 0
\end{align}
Siccome è un'equazione di Eulero usiamo l'Ansatz $v(x) = x^\alpha$ e otteniamo
\begin{align}
	&\alpha(\alpha-1) + \frac{1}{4} = 0 \to \alpha_{1,2} = \frac{1\pm \sqrt{1-1}}{2} = \frac{1}{2} \firstpassage
	&v_{{om}_1}(x) = x^{\frac{1}{2}} \spacer v_{{om}_2}(x) = x^{\frac{1}{2}} \ln(x) \nextpassage
	&v_{om}(x) = Ax^{\frac{1}{2}} + Bx^{\frac{1}{2}}\ln(x) = x^{\frac{1}{2}}(A + B\ln(x)) \firstpassage
	&W(v_1,v_2) = \begin{vmatrix}
		x^{\frac{1}{2}} && x^{\frac{1}{2}}\ln(x)\\
		\frac{1}{2} x^{-\frac{1}{2}} && \frac{1}{2} x^{-\frac{1}{2}} \ln(x) + x^{-\frac{1}{2}}
	\end{vmatrix} = \dots = 1
\end{align}
La soluzione generica del PdC sarà data da
\begin{align}
	\tilde{G}(x,y) = \alpha v_1(x) + \beta v_2(x) + \theta(x-y)\left[ \frac{v_1(y) v_2(x) -v_1(x)v_2(y)}{W} \right] 
\end{align}
Siccome siamo in presenza di un PdC, a noi serve la \textbf{funzione di Green avanzata}, e quindi dobbiamo porre $\alpha=\beta=0$
\begin{align}
	G(x,y) &= \theta(x-y)\left[ \frac{v_1(y) v_2(x) -v_1(x)v_2(y)}{W} \right] = \nonumber \\
		   &= \theta(x-y)\left[ y^{\frac{1}{2}} x^{\frac{1}{2}} \ln(x) -x^{\frac{1}{2}}y^{\frac{1}{2}} \ln(y) \right] 
\end{align}
Possiamo quindi andare a calcolare la soluzione
\begin{align}
	v(x) &= \int_{1}^{x} dy \; G(x,y) \left( -y^{-\frac{1}{2}} \right) + \cancel{BC} = \nonumber\\
		 &= -\int_{1}^{x} dy \; \delta(x-y)\left[ x^{\frac{1}{2}} \ln(x) -x^{\frac{1}{2}} \ln(y) \right]= \nonumber\\
		 &= -x^{\frac{1}{2}}\int_{1}^{x} dy \; \left[ \ln(x) - \ln(y) \right] = \nonumber\\
		 &= -x^{\frac{1}{2}} \left[\ln(x)\int_{1}^{x} dy + \int_{1}^{x} dy\; \ln(y)\right] = \nonumber\\
		 &= -x^{\frac{1}{2}} \left[\ln(x)(x-1) - [y(\ln(y) -1)]_1^x \right] = \nonumber\\
		 &= -x^{\frac{1}{2}} \left[\ln(x)(x-1) - x(\ln(x) -1) -1 \right] = \nonumber\\
		 &= -x^{\frac{1}{2}} \left[\cancel{x\ln(x)} -\ln(x) - \cancel{x\ln(x)} +x -1 \right] \nonumber\\
		 &= x^{\frac{1}{2}} \left[1 +\ln(x)-x \right] \firstpassage
	u(x) &= x^{-\frac{3}{2}} v(x) = \frac{1}{x} \left[1 +\ln(x)-x \right] \firstpassage
	f(x) &= u(x) + 1 = \frac{1 + \ln(x)}{x}
\end{align}
Verifichiamo la bontà della soluzione trovata
\begin{align}
	&\dot{f}(x) = -\frac{\ln(x)}{x^2} \spacer \ddot{f}(x) = \frac{2\ln(x) -1}{x^3} \firstpassage
	&\double{\dot{f}(1) = -\frac{\ln(1)}{1}=0 }{f(1) = \frac{1 + \ln(1)}{1} = 1} \quad \leftarrow \quad \text{le BC sono rispettate}\\
	&\cancel{x^2}\frac{2\ln(x) -1}{x^{\cancel{3}}} - 3\cancel{x} \frac{\ln(x)}{x^{\cancel{2}}} + \frac{1+\ln(x)}{x} = 0 \to \frac{1}{x}[\cancel{2\ln(x)} -\cancel{3\ln(x)} + \cancel{\ln(x)} -\cancel{1} +\cancel{1}] = 0 \quad \leftarrow \quad \text{ok}
\end{align}
\newpage 
\subsection{Esempio 5: prova d'esame (Alessandro Marcelli)}
Sia il seguente problema di Cauchy
\begin{align}
	\triple{x^2 \ddot{f}(x) - x \dot{f}(x) + f(x) = x^2}{\dot{f}(1) = 0}{f(1)=0}
\end{align}
Si richede di calcolare
\begin{enumerate}
	\item la funzione di Green del problema
	\item la soluzione della non omogenea col metodo di Green per $x\geq 1$
\end{enumerate}
Iniziamo passando l'equazione in \textbf{forma omogenea} (omettiamo la dipendenza da x per snellire la notazione)
\begin{align}
	&f(x) = A(x)v(x) \to f=Av\firstpassage
	&\dot{f} = \dot{A}v + A \dot{v} \to \ddot{f} = \ddot{A}v + 2 \dot{A}\dot{v} + A \ddot{v} \nextpassage
	&x^2 \ddot{A}v + 2x^2 \dot{A}\dot{v} +x^2 A \ddot{v} -x\dot{A}v -x A \dot{v} + Av = x^2\nextpassage
	& x^2 \ddot{A}v +x(2x \dot{A} - A) \dot{v} + (x^2 \ddot{A} - x \dot{A} + A)v = x^2\nextpassage
	&2x \dot{A} - A = 0 \to \frac{dA}{dx} = \frac{A}{2x} \to \int dA \; \frac{1}{A} = \int dx \; \frac{1}{2x} = \ln(A) = \ln(\sqrt{x}) \nextpassage
	&A = \sqrt{x} = x^{\frac{1}{2}} \to \dot{A} = \frac{1}{2}x^{-\frac{1}{2}} \to \ddot{A} = -\frac{1}{4}x^{-\frac{3}{2}} \nextpassage
	&x^{\frac{5}{2}} \ddot{v} + \frac{1}{4} x^{\frac{1}{2}} = x^2\nextpassage
	&\ddot{v} +\frac{1}{4} x^{-2} v = x^{-\frac{1}{2}}
\end{align}
Cominciamo col risolvere l'omogenea:
\begin{align}
	\ddot{v} +\frac{1}{4} x^{-2} v = 0 
\end{align}
Siamo in presenza di un'equazione di eulero, visto che
\begin{align}
	&\ddot{v}  + \frac{a}{x} \dot{v} + \frac{b}{x^2} u = 0 \spacer a=0 \; , \; b= \frac{1}{4} \firstpassage
	&	\ddot{v} +\frac{1}{4x^2} v = 0 
\end{align}
Possiamo quindi fare l'Ansatz $v = x^\alpha$, ottenendo
\begin{align}
	\alpha (\alpha -1) x^{\alpha -2} +\frac{1}{4} x^{\alpha -2} = 0 \to \alpha^2 - \alpha + \frac{1}{4} = 0 
\end{align}
Siccome $\Delta = 0$ avremo una sola soluzione in $\alpha = \frac{1}{2}$. Ci conviene quindi trasformare l'equazione con il cambio di variabile $x=e^{t}$, dove otteniamo
\begin{align}
	&\frac{dv}{dx} = e^{-t}\frac{dv}{dt} \to \frac{d^2v}{dx^2} = e^{-2t}\frac{d^2 v}{dt^2} - e^{-2t}\frac{dv}{dt} \firstpassage
	&e^{-2t} \left[ \frac{d^2 v}{dt^2} - (a-1) \frac{dv}{dt} + bv\right] = 0
\end{align}
nel nostro caso $a=0$, quindi
\begin{align}
	\frac{d^2 v}{dt^2} + \frac{dv}{dt} + \frac{v}{4}= 0
\end{align}
Il nostro  Ansatz diventa $v = x^\alpha = (e^t)^\alpha  = e^{\alpha t}$ eotteniamo le due soluzioni
\begin{align}
	v_1 = e^{\alpha t} \spacer v_2 = t e^{\alpha t} 
\end{align}
Che, tornando nella variabile originale diventano
\begin{align}
	v_1 = x^\alpha \spacer v_2 = x^\alpha \ln(x) 
\end{align}
Nel nostro caso saranno quindi
\begin{align}
	&v_1 = x^\frac{1}{2} \spacer v_2 = x^\frac{1}{2} \ln(x) \firstpassage
	&v_{om} = x^\frac{1}{2}(A + B\ln(x)) \firstpassage
	&\dot{v}_{om} = \frac{1}{2}x^{-\frac{1}{2}}A + B \left( x^\frac{1}{2} \cdot \frac{1}{x} + \frac{1}{2}x^{-\frac{1}{2}} \right) = \frac{1}{2}x^{-\frac{1}{2}} A + \frac{3}{2}x^{-\frac{1}{2}} B = \frac{1}{2}x^{-\frac{1}{2}} (A + 3B) \nextpassage
	&\double{v_{om}(1) = 0}{\dot{v}_{om}(1) = 0} \to \double{1^\frac{1}{2}(A + B\ln(1)) = 0 \to& A = 0}{\frac{1}{2}\cdot 1^{-\frac{1}{2}} (A + 3B)=0 \to& B = 0}
\end{align}

La soluzione omogenea è nulla, e posso quindi usare il \textbf{teorema di Green}.
 
Iniziamo calcolando il \textbf{Wronskiano}
\begin{align}
	 W = \begin{vmatrix}
	x^\frac{1}{2} && x^\frac{1}{2} \ln(x) \\
	\frac{1}{2}x^{-\frac{1}{2}} && \frac{1}{2}x^{-\frac{1}{2}} \ln(x) + x^{-\frac{1}{2}}
\end{vmatrix} = \frac{1}{2} \ln(x) + 1 - \frac{1}{2}\ln(x) = 1
\end{align}
Andiamo ora calcolare la \textbf{funzione di Green avanzata} del problema
\begin{align}
	G(x,y) &= \theta(x-y)\left[ \frac{v_1(y) v_2(x) -v_1(x)v_2(y)}{W} \right] = \nonumber \\
	&= \theta(x-y)\left[y^\frac{1}{2} x^\frac{1}{2} \ln(x) - x^\frac{1}{2} y^\frac{1}{2}\ln(y) \right] 
\end{align}
Possiamo ora andare a calcolare la soluzione particolare
\begin{align}
	v(x) &= \int_{1}^{x} dy \; G(x,y) \cdot y^2 = \nonumber\\
		 &= \int_{1}^{x} dy \; \left[y^\frac{1}{2} x^\frac{1}{2} \ln(x) - x^\frac{1}{2} y^\frac{1}{2}\ln(y) \right] y^2 = \nonumber\\
		 &= x^\frac{1}{2} \ln(x)\int_{1}^{x} dy \;y^\frac{5}{2} - x^\frac{1}{2}\int_{1}^{x} dy \; y^\frac{5}{2}\ln(y)  = \nonumber\\
		 &= x^\frac{1}{2} \ln(x) \left.\frac{2}{7}y^{\frac{7}{2}} \right|_1^x + 
\end{align}

\newpage
\section{Equazioni Differenziali Complesse}

N'aggio fatto in tempo, tanto Bufalini le ha scritte bene su Latex.