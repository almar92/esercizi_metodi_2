\chapter{Note introduttive}

\section{Ciancio alle bande}

Questa ""dispensa"" è una raccolta di materiale utile (almeno al momento della scrittura di questa introduzione) alla prova scritta dell'esame  di Metodi e Modelli Matematici per la Fisica 2 del corso di laurea magistrale in Fisica dell'Univeristà Tor Vergata. 

Sono esercizi svolti da gente brava che, come esercizio personale ai fini di comprenderli al meglio, ho raccolto insieme, modificandoli quel tanto che bastava per passare dal "calcolo su foglio" a roba che sembra effettivamente una dispensa. Sul successo dell'operazione lascio il giudizio ai posteri, ammesso che ce ne saranno.

Quando possibile ho cercato di aggiungere dei minimi richiami di teoria qualora fossero assenti dagli appunti che nel corso del tempo ho \st{sciacallato} raccolto durante la preparazione di questa prova scritta.

Ben lungi dall'essere un lavoro perfetto, ma sinceramente spero di non fare in tempo a renderlo tale (RIP Equazioni Differenziali in campo complesso). 

\section{Crediti}

Nei titoli degli argomenti sono creditati i vari autori dei materiali raccolti. Qualora una sezione fosse composta da materiale di più persone, queste sono prima creditate nel titolo della sezione e poi nelle sottosezioni rilevanti. Se invece una sezione è farina del sacco di un solo autore, questo viene creditato solo nel titolo della sezione.

Detto in parole povere, ste dispense sono l'equivalente di un mixtape, e a parte che con il buon Paolo, non ho collaborato con gli altri due autori, mi sono limitato a fare l'amanuense con il loro lavoro. Ergo se avete cacciato soldi per avere sto documento vi hanno truffato.

\section{Contatti}

Qualora eventuali futuri \st{masochisti} utenti volessero contribuire a quest'accrocco mi si può raggiunere all'indirizzo \textbf{marcelli.alessandro.92@gmail.com}

