\section{Espansione di Mittag-Leffler (Davide Bufalini)}

Presa una $f(z)$ analitica nell'origine e con infiniti poli semplici che si accumulano all'infinito, vale la seguente approssimazione
\begin{align}
	f(z) &= f(0) + \sum_{n=1}^{+\infty} R_n \cdot \left( \frac{1}{z-z_n} + \frac{1}{z_n} \right)\\
	R_n &= Res(f, z_n)
\end{align}


\subsection{Esempio 1: $\cot (z)$}

\subsubsection{Calcolo dell'espansione}

La funzione  
\begin{align}
	f(z) = \cot(z)
\end{align}
ha poli in $z_n = n\pi$ con $n\in \Z$, ma \textbf{non} è analitica in $z=0$.

Per eliminare la singolarità in 0 considero allora la funzione
\begin{align}
	f(z) = \cot(z) - \frac{1}{z}
\end{align}
I cui residui saranno
\begin{align}
	R_n = \limit{z}{n\pi} \left[ (z-n\pi) \cdot \frac{z\cot(z) -1}{z} \right] = \dots = 1  
\end{align}
Applicando lo sviluppo ottengo
\begin{align}
	\cot(z) - \frac{1}{z} &= \sum_{n\neq 0} \left( \frac{1}{z-n\pi} + \frac{1}{n\pi}\right) \firstpassage
	\cot(z) &= \frac{1}{z} + \sum_{n\neq 0} \left( \frac{1}{z-n\pi} + \frac{1}{n\pi}\right) = \nonumber\\
	&= \frac{1}{z} + \sum_{n>0} \left[ \frac{1}{z-n\pi} + \frac{1}{n\pi} - \left( \frac{1}{-z-n\pi} + \frac{1}{n\pi} \right) \right] = \nonumber\\
	&= \dots  = \nonumber\\
	&= \frac{1}{z} + \sum_{n>0} \frac{2z}{z^2 - (n\pi)^2}
\end{align}

\newpage

\subsubsection{Verifica dell'espansione}

Dimostrare la validità della relazione
\begin{align}
	\cot(z) = \frac{1}{z} + \sum_{n>0} \frac{2z}{z^2 - (n\pi)^2}
\end{align}

Iniziamo dimostrando l'analiticità. Come abbiamo visto prima, dobbiamo prendere
\begin{align}
	f(z) = \cot(z) - \frac{1}{z}
\end{align}
Andiamo quindi a calcolarne il limite in 0
\begin{align}
	\limit{z}{0} \cot(z) \frac{1}{z} &= \limit{z}{0} \frac{\frac{z\cos(z)}{\sin(z)} -1}{z} = \nonumber\\
	&= \limit{z}{0} \frac{z\cos(z) - \sin(z)}{z\sin (z)} = \nonumber\\
	&= \limit{z}{0}\frac{\cos(z) \cdot [z - \tan (z)]}{z - \sin(z)} = \limit{z}{0} \frac{\cot(z)}{z} \cdot(z-\tan(z)) = 1\cdot 0 = 0
\end{align}
Verificata l'analiticità, procediamo a studiare poli e residui della funzione. 

Avremo dei poli semplici quando $\sin(z) = 0$, e qundi per $z_n=n\pi$, con $n=0$ \textbf{escluso}, in quanto vi si annulla la funzione. Ne segue che i residui saranno
\begin{align}
	Res(f,z_n) = \limit{z}{n\pi}  (z-n\pi) \cdot \frac{z\cot(z) -1}{z}
\end{align} 
Facendo il cambio di variaible $u=z-n\pi$ otteniamo
\begin{align}
	Res(f,z_n) = \limit{u}{0} u \cdot \frac{(u+n\pi)\cot(u+n\pi) -1}{u+n\pi} \label{eq3}
\end{align} 
Se però osserviamo un attimo il termine $\cot(u+n\pi)$ notiamo come lo si possa manipolare nel seguente modo
\begin{align}
	\cot(u+n\pi) &= \frac{\cos(u+n\pi)}{\sin(u+n\pi)} = \frac{\cos(u)\cdot(-1)^n}{\sin(u)\cdot(-1)^n} = \frac{\cos(u)}{\sin(u)} = \cot(u)
\end{align}
Possiamo quindi riscrivere la \ref{eq3} come
\begin{align}
	Res(f,z_n) &= \limit{u}{0} u \cdot \frac{(u+n\pi)\cot(u) -1}{u+n\pi} \simeq \limit{u}{0} u\cdot \frac{(u+n\pi)\cdot \frac{1}{u} -1}{u+n\pi} = \nonumber\\
	&= \left. \frac{u+n\pi -u}{u+n\pi} \right|_{u=0} = 1 \quad \forall n\in \N
\end{align} 

Possiamo quindi applicare lo sviluppo
\begin{align}
	f(z) &= \sum_{n\neq 0} \left( \frac{1}{n-n\pi} + \frac{1}{n\pi} \right) = \nonumber\\
	&= \sum_{n=-\infty}^{-1} \left( \frac{1}{n-n\pi} + \frac{1}{n\pi} \right) + \sum_{n=1}^{+\infty} \left( \frac{1}{n-n\pi} + \frac{1}{n\pi} \right) = \nonumber \\
	&= \sum_{n>0} \left[ \frac{1}{z-n\pi} + \frac{1}{n\pi} - \left( \frac{1}{-z-n\pi} + \frac{1}{n\pi} \right) \right] = \nonumber\\
	&= 2z\sum_{n=1}^{+\infty} \frac{1}{z^2 - (n\pi)^2}
\end{align}
E la tesi è così dimostrata.

\newpage

\subsection{Esempio 2: $\sin^{-2}(z)$}

Utilizziamo l'integrale (di Cauchy? Chiedere)
\begin{align}
	I_n = \oint_{Q_n} \frac{d\zeta}{(\zeta -z) \sin^2(\zeta)}
\end{align}
Studiando i poli otteniamo
\begin{align}
	\zeta_0 &= z \quad &\text{polo semplice}\\
	\zeta_n &= n\pi \; n\in \N  \quad &\text{poli doppi}
\end{align}
Siccome $I_n \arrowlim{n}{+\infty} 0$ abbiamo che
\begin{align}
	0 &= \sum_{n\in \N} Res(f(\zeta), n\pi) + Res(f(\zeta), z) =\nonumber\\
	&= \sum_{n\in \N} Res(f(\zeta), n\pi) + \frac{1}{\sin^2(z)}  \label{eq4}
\end{align}
Da cui otteniamo
\begin{align}
	\frac{1}{\sin^2(\zeta)} &= -\sum_{n\in \N} Res(f(\zeta), n\pi)
\end{align}
Dato che
\begin{align}
	Res(f(\zeta), n\pi) &= \limit{\zeta}{n\pi} \frac{d}{d\zeta} \frac{(\zeta - n\pi)^2}{(\zeta - z) \sin^2(\zeta)} \firstpassage
	\double{u=\zeta - n\pi}{\zeta = u + n\pi} \nextpassage
	Res(f(\zeta), n\pi) &= \limit{u}{0} \frac{d}{du} \frac{u^2}{(u + n\pi - z) \sin^2(u + n\pi)} =  \limit{u}{0} \frac{d}{du} \frac{u^2}{(u + n\pi - z) (-1)^{2n} \sin^2(u)}= \nonumber\\
	&= \limit{u}{0} \frac{d}{du} \frac{u^2}{(u + n\pi - z) \left(u -\frac{u^3}{3!} + \dots\right)^2} = \left. \frac{d}{du} \frac{u^2}{(u + n\pi - z) \left(u -\frac{u^3}{3!} + \dots\right)^2} \right|_{u=0} = \nonumber \\
	&= \left. \frac{d}{du} \frac{1}{(u + n\pi - z) \left(1 -\frac{u^2}{3!} + \dots\right)^2} \right|_{u=0} = \left. \frac{d}{du} \frac{1}{(u + n\pi - z) \left(1 -\frac{u^2}{3!} + \dots\right)^2} \right|_{u=0} = \nonumber \\
	&= \left. \frac{d}{du} \frac{\left(1 +\frac{u^2}{3!} - \dots\right)^2}{(u + n\pi - z) } \right|_{u=0} = \left. \frac{d}{du} \frac{\left(1 +\frac{u^2}{3} + \dots\right)}{(u + n\pi - z) } \right|_{u=0} = \nonumber \\
	&= \left. \frac{\left( \frac{2}{3} u + \dots \right) (u + n\pi-z ) - \left(1 + \frac{u^2}{3}\right)}{(u+ n\pi -z)^2} \right|_{u=0} = -\frac{1}{(n\pi -z)^2}
\end{align}
La \ref{eq4} diventa
\begin{align}
	\frac{1}{\sin^2(z)} = \sum_{n\in \N} \frac{1}{(n\pi -z)^2}
\end{align}

\newpage
