\section{Operatori differenziali (Davide Bufalini, Alessandro Marcelli)}

\subsection{Spunti di teoria (Alessandro Marcelli)}

Gli operatori differenziali $\mathcal{L}_x^{(n)}$di ordine $n$ sono definiti come
\begin{align}
	&\double{\mathcal{L}_x^{(n)} u(x) = f(x)}{\text{Boundary conditions}}\\
	&\mathcal{L}_x^{(n)} = \sum_{k=0}^{n} a_k(x) \frac{d^k}{dx^k}
\end{align}


I problemi di ricerca di soluzioni particolari si dividono in due famiglie
\begin{align}
	&\left\{
	\begin{array}{ccc}
		\mathcal{L}_x^{(n)} u(x) &= &f(x)\\
		\frac{d^n}{dx^n}u(x_0) &= &u^{(n)}_0\\
		\frac{d^{n-1}}{dx^{n-1}}u(x_0) &= &u^{(n-1)}_0\\
		\vdots\\
		\frac{d}{dx}u(x_0) &= &u'_0\\
		u(x_0) &= &u_0\\
	\end{array}
	\right. \quad &\text{Problemi di Cauchy}\\
	&\left\{
	\begin{array}{ccc}
		\mathcal{L}_x^{(n)} u(x) &= &f(x)\\
		u(x_n) &= &u_n\\
		u(x_{n-1}) &= &u_{n-1}\\
		\vdots\\
		u(x_1) &= &u_1\\
		u(x_0) &= &u_0\\
	\end{array}
	\right. \quad &\text{Problemi di Sturm-Liouvlille}
\end{align}

La ricerca delle soluzioni può passare attraverso il calcolo del nucleo $G(x,y)$ dell'operatore inverso, denominato \textbf{funzione di Green}, per il quale vale
\begin{align}
	\mathcal{L}_x^{(n)} G(x,y) = \delta(x-y)
\end{align}
E che quindi si può definire
\begin{align}
	G(x,y) = c_1 u_1(x) + c_2 u_2(x) + \theta(x-y) \left[ \frac{u_1(y)u_2(x) - u_1(x)u_2(y)}{W(y)} \right]
\end{align}
Dove al denominatore abbiamo il \textbf{denominatore della matrice di Wronsky}, definito come
\begin{align}
	W = \det \begin{pmatrix}
		u_1^0 && u_2^0\\
		\dot{u}_1^0 && \dot{u}_2^0
	\end{pmatrix}
\end{align}
Due famiglie importanti di funzioni di Green sono le
\begin{enumerate}
	\item \textbf{Funzioni di Green avanzate}, utili nello studio dei problemi di Cauchy
	\begin{align}
		G(x,y) \equiv 0 \quad x>y
	\end{align}
	\item \textbf{Funzioni di Green ritardate}
	\begin{align}
		G(x,y) \equiv 0 \quad x<y
	\end{align}
\end{enumerate}

\newpage


\subsection{Esempio 1: Operatore Differenziale del IIo Ordine (Davide Bufalini)}

Sia l'operatore
\begin{align}
	\mathcal{L}_x^\lambda = -\frac{d^2}{dx^2} - \lambda \spacer \lambda= \text{cost.}
\end{align}

Si richiede di
\begin{enumerate}
	\item calcolare, al variare di $\lambda$, la soluzione del problema omogeneo
	\begin{align}
		\triple{\mathcal{L}_x^\lambda f(x) &= 0}{f(0) &=0}{\dot{f}(\pi) &=0}
	\end{align}
	\item ricavare la Funzione di Green (l'operatore risolvente) per i valori in cui $\mathcal{L}_x^\lambda$ è invertibile del corrispontente problema di Sturm-Liouville
\end{enumerate}

L'operatore è in forma canonica con coefficienti costanti nella forma
\begin{align}
	a \ddot{f} + b \dot{f} + cf=0 \spacer \text{nel nostro caso } b=0
\end{align}
Prendendo un Ansatz del tipo $f(x) \sim e^{\alpha x}$ otteniamo
\begin{align}
	-\alpha^2 - \lambda =0 \to \alpha= \pm i \sqrt{\lambda}
\end{align}
Da cui segue
\begin{align}
	f(x) = A e^{i \sqrt{\lambda}} + B e^{-i \sqrt{\lambda}}
\end{align}
Imponendo l condizioni di bordo otteniamo
\begin{align}
	&\double{f(0) =0}{\dot{f}(\pi) =0} \to \double{A+B = &0}{i\sqrt{\lambda} \left(A e^{i \sqrt{\lambda}} - B e^{-i \sqrt{\lambda}}\right) = &0} \firstpassage
	&\double{B = &-A}{Ai \sqrt{\lambda} 2 \cos(\sqrt{\lambda} \pi) =&0}
\end{align}
Escludendo le soluzioni banali $A=0=B$ e $\lambda=0$ abbiamo $n\in\Z$ autovalori nella forma
\begin{align}
	\lambda = \frac{(2k+1)^2}{4}
\end{align}
Per trovare gli autovettori corrispondenti ci affidiamo alla condizione $A = -B$ dalla quale segue
\begin{align}
	f_n(x)=-2i B \sin\left(\frac{2k+1}{2} x\right)
\end{align}
Per trovare B sfruttiamo l'ortogonalità
\begin{align}
	\delta_{n,m} = <f_n,f_m> &= \int_{0}^{\pi} dx \; \overline{f_n}(x) f_m(x) = \nonumber \\
	&= \int_0^\pi dx\; (2i B^*) \sin\left(\frac{2k+1}{2} x\right) \cdot (-2i B) \sin\left(\frac{2k+1}{2} x\right)
\end{align}
Iniziamo studiando il caso $n=m$
\begin{align}
	\delta_{m,m} = <f_n,f_m> &= 4|B|^2 \int_0^\pi dx\; \sin^2\left(\frac{2k+1}{2} x\right) = 4|B|^2 \frac{\pi}{2}
\end{align}
Per ortonormalità imponiamo
\begin{align}
	4|B|^2 \frac{\pi}{2} = 1 \to |B| = \frac{1}{\sqrt{2\pi}} 
\end{align}
Che, nel caso $n\neq m$, diventa 	
\begin{align}
	B = \frac{e^{i\rho}}{\sqrt{2\pi}} \spacer \rho \in \R	
\end{align}
Da cui segue che
\begin{align}
	f_n(x) = -i \sqrt{\frac{2}{\pi}}e^{i\rho}\sin \left( \frac{2n+1}{2}x \right) \spacer \double{\rho \in \R}{n\in\Z}
\end{align}

Siccome $\lambda=0$ non è autovalore non abbiamo modi nulli.

Andiamo ora a calcolare l'\textbf{operatore risolvente}.
\begin{align}
	&(\mathcal{L}_{SL} -\lambda \mathbb{1}) G_\lambda (x,y) = \delta (x-y) \spacer \lambda \neq \lambda_n \; \text{per evitare poli semplici} \firstpassage
	&-\frac{d^2}{dx^2} G_x(x,y) - \lambda G_\lambda = \delta (x-y)
\end{align}

Utilizziamo un trucco. Siccome il problema è della tipologia
\begin{align}
	-\frac{d^2f}{dx^2} - \lambda f = 0
\end{align}
Andiamo a cercare due soluzioni $f^0_{1,2}$ tali che ciascua soddisfi una delle due condizioni al bordo. Una volta trovate andiamo a calcolare il Wronskiano, necessario per la definizione della funzione di Green
\begin{align}
	G_\lambda (x,y) = \double{\frac{f_1^0(x)f_2^0(x)}{W} \spacer x<y}{\frac{f_1^0(y)f_2^0(y)}{W} \spacer x>y}
\end{align}
In base alle condizioni di bordo proviamo con
\begin{align}
	f_1^0(x) &= \sin(\sqrt{\lambda} x) \to f_1^0(0) =0\\
	f_2^0(x) &= \cos(\sqrt{\lambda} (\pi-x)) \to f_2^0(\pi) =0
\end{align}
Andiamo quindi a calcolarne il Wronskiano
\begin{align}
	W = \begin{vmatrix}
		f_1^0 && f_2^0\\
		\dot{f_1^0} && \dot{f_2^0}
	\end{vmatrix} = \begin{vmatrix}
		\sin(\sqrt{\lambda} x) && \cos(\sqrt{\lambda} (\pi-x))\\
		\sqrt{\lambda}\cos(\sqrt{\lambda}x) && \sqrt{\lambda} \cos (\sqrt{\lambda} (\pi-x))
	\end{vmatrix} = \cdots = -\sqrt{\lambda} \cos( \sqrt{\lambda} \pi)
\end{align}
Da cui otteniamo
\begin{align}
	G_\lambda (x,y) = \double{-\frac{ \sin(\sqrt{\lambda} x)\cos(\sqrt{\lambda} (\pi-x))}{\sqrt{\lambda} \cos( \sqrt{\lambda} \pi)} \spacer x<y}{-\frac{ \sin(\sqrt{\lambda} y)\cos(\sqrt{\lambda} (\pi-y))}{\sqrt{\lambda} \cos( \sqrt{\lambda} \pi)} \spacer x>y}
\end{align}

Vediamo come per i $\lambda_n$ che abbiamo trovato in precedenza ci siano solo poli semplici.

\newpage

\subsection{Esempio 2 (Davide Bufalini)}

Dato l'operatore differenziale
\begin{align}
	\mathcal{L}_x = -i \frac{d}{dx} + \frac{\beta}{x} \spacer \beta \in \R
\end{align}
Il cui nucleo è dato da
\begin{align}
	\mathcal{D}_{\mathcal{L}_x} = \left\{ f \in L^2 \left( \left[\frac{1}{2},1\right] \right) \taleche f\left(\frac{1}{2}\right) = f(1) \right\}
\end{align}

Si richiede di
\begin{enumerate}
	\item Costruire $(\mathcal{L}_x^\dagger, \mathcal{D}_{\mathcal{L}_x^\dagger})$
	\item Discutere autovalori e autovettori di $(\mathcal{L}_x, \mathcal{D}_{\mathcal{L}_x})$
	\item Trovare i valori di $\beta$ per cui $(\mathcal{L}_x, \mathcal{D}_{\mathcal{L}_x})$ è autoaggiunto e/o invertibile
\end{enumerate}

Partiamo dall'\textbf{aggiunto}:
\begin{align}
	<g,\mathcal{L}_x f> &= \int_{\frac{1}{2}}^{1} dx \; \overline{g} \left(-i \frac{d}{dx} + \frac{\beta}{x}\right) f = \nonumber\\
	&= \int_{\frac{1}{2}}^{1} dx \; \overline{g} \left(-i \frac{d}{dx}f\right) + \int_{\frac{1}{2}}^{1} dx \; \overline{g}\left(\frac{\beta}{x} f\right) = \nonumber\\
	&=-i \int_{\frac{1}{2}}^{1} dx \; \overline{g} \dot{f} + \int_{\frac{1}{2}}^{1} dx \; \overline{g}\frac{\beta}{x}f = \nonumber\\
	&= \left. -i f \overline{g} \right|_{\frac{1}{2}}^{1} +i \int_{\frac{1}{2}}^{1} dx \;  \dot{\overline{g}}f + \int_{\frac{1}{2}}^{1} dx \; \overline{\left(\frac{\beta}{x} g\right)} f
\end{align}
Imponendo che $\mathcal{L}_x^\dagger$ sia autoagiunto otteniamo che
\begin{align}
	&-i f\left( \frac{1}{2} \right) \cdot\left[ \overline{g}\left(\frac{1}{2}\right) - \overline{g}(1) \right] = 0 \firstpassage
	&\overline{g}\left(\frac{1}{2}\right) = \overline{g}(1) \nextpassage
	&\mathcal{L}_x^\dagger = -i \frac{d}{dx} + \frac{\beta}{x} \spacer \mathcal{D}_{\mathcal{L}_x^\dagger} = \left\{ g \in L^2 \left( \left[\frac{1}{2},1\right] \right) \taleche g\left(\frac{1}{2}\right) = g(1) \right\}
\end{align}
Andiamo ora a calcolare gli autovalori e gli autovettori:
\begin{align}
	&\mathcal{L}_x f= \lambda f \firstpassage
	&\left( -i \frac{d}{dx} + \frac{\beta}{x} \right) f = \lambda f\nextpassage
	&\frac{df}{dx} = i\left( \lambda - \frac{\beta}{x} \right) f \nextpassage
	&\int_{\frac{1}{2}}^{x} dt \; \frac{\dot{f}(t)}{f(t)} ) = \int_{\frac{1}{2}}^{x} dt \; \left( \lambda - \frac{\beta}{x} \right) \nextpassage
	&\vdots\nextpassage
	&f(x) = f\left(\frac{1}{2}\right) e^{i\left[ \lambda \left( x - \frac{1}{2} \right)  - \beta \ln(2x)\right]}
\end{align}
Applicando le condizioni al contorno otteniamo
\begin{align}
	&f(1) =  f\left(\frac{1}{2}\right) \firstpassage
	&f\left(\frac{1}{2}\right) e^{i\frac{\lambda}{2}  - \beta \ln(2)} = f\left(\frac{1}{2}\right) e^{i\cdot 0}\nextpassage
	&e^{i\frac{\lambda}{2}  - \beta \ln(2)} = 1 \nextpassage
	&e^{i\frac{\lambda_n}{2}  - \beta \ln(2)} = e^{i2n\pi} \nextpassage
	&\frac{\lambda_n}{2}  = \beta \ln(2) + 2n\pi \nextpassage
	&\lambda_n = 2\beta \ln(2) + 4n\pi \to f_n(x) = f\left(\frac{1}{2}\right) e^{i\left[ (2\beta \ln(2) + 4n\pi) \left( x - \frac{1}{2} \right) - \beta \ln(2x)\right]}
\end{align}

\newpage