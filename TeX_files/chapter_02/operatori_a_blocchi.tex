\section{Operatori a blocchi (Davide Bufalini, Alessandro Marcelli, Paolo Proia)}
\subsection{Esempio 1 con metodo esplicito e analitico (Davide Bufalini, Paolo Proia)}
Dato il seguente operatore a blocchi
\begin{align}
	\triple{(Tx)_1 \quad\;\; &= &0 \qquad \qquad \qquad}{(Tx)_{2n} \quad &= &\alpha x_{2n+1} \quad n> 1}{(Tx)_{2n+1} &= &\alpha x_{2n} \quad\quad n> 1} \spacer \alpha \in \C
\end{align}
Si richiede di
\begin{enumerate}
	\item Costruire $T^\dagger$ e calcolare $||T||$ e  $||T^\dagger||$
	\item Discutere gli spettri puntuali e relative molteplicità
	\item Discutere gli spettri residui
	\item Discutere per quali valori di $\alpha \in \C$ (se esistono) l'operatore $T$ è una combinazione delle seguenti
	\begin{enumerate}
		\item Autoaggiunto
		\item Unitario
		\item Invertibile
	\end{enumerate}
\end{enumerate}

Un possibile modo per il calcolo dell'aggiunto è iniziare esplicitando l'azione dell'operatore
\begin{align}
	\begin{matrix}
		Tx_1 &= &0\\
		Tx_2 &= &\alpha x_3\\
		Tx_3 &= &\alpha x_2\\
		Tx_4 &= &\alpha x_5\\
		Tx_5 &= &\alpha x_4
	\end{matrix} \to T= 
	\begin{bmatrix}
		0 && 0 && 0 && 0 && 0 && 0 && \dots \\
		0 && 0 && \alpha && 0 && 0 && 0 && \dots \\
		0 && \alpha && 0 && 0 && 0 && 0 && \dots \\
		0 && 0 && 0 && 0 && \alpha && 0 && \dots \\
		0 && 0 && 0 && \alpha && 0 && 0 && \dots \\
		\vdots && \vdots && \vdots && \vdots && \vdots && \vdots && \ddots
	\end{bmatrix}
\end{align}
Da cui segue che
\begin{align}
	T^\dagger= 
	\begin{bmatrix}
		0 && 0 && 0 && 0 && 0 && 0 && \dots \\
		0 && 0 && \alpha^* && 0 && 0 && 0 && \dots \\
		0 && \alpha^* && 0 && 0 && 0 && 0 && \dots \\
		0 && 0 && 0 && 0 && \alpha^* && 0 && \dots \\
		0 && 0 && 0 && \alpha^* && 0 && 0 && \dots \\
		\vdots && \vdots && \vdots && \vdots && \vdots && \vdots && \ddots
	\end{bmatrix} \to 	\triple{(T^\dagger x)_1 \quad\;\; &= &0 \;\qquad \qquad \qquad}{(T^\dagger x)_{2n} \quad &= &\alpha^* x_{2n+1} \quad n> 1}{(T^\dagger x)_{2n+1} &= &\alpha^* x_{2n} \quad\quad n> 1}
\end{align}

Un altro modo per il calcolo dell'aggiunto passa dalla sua definizione, che ci dice che
\begin{align}
	&(y,Tx) = (T^\dagger y,x) \firstpassage
	&\sum_{i=1}^{N} y^*_i (Tx)_i = \sum_{i=1}^{N} (T^\dagger y)_i^* x_i
\end{align}
Esplicitando le somme si ottengono i seguenti confronti
\begin{align}
	&0 + y_2^* \cdot \alpha x_3 + y_3^* \cdot \alpha x_2 + y_4^* \cdot \alpha x_5 + y_5^* \cdot \alpha x_4 + \dots =  \nonumber \\
	=&(T^\dagger y)_1^* \cdot x_1 + (T^\dagger y)_2^* \cdot x_2 +(T^\dagger y)_3^* \cdot x_3 +(T^\dagger y)_4^* \cdot x_4 + (T^\dagger y)_5^* \cdot x_5 + \dots\firstpassage
	&y_1 \cdot 0 = 0  = (T^\dagger y)_1^* \cdot x_1 \to (T^\dagger y)_1 = 0\\
	&\double{y_2^* \cdot (Tx)_2 = y_2^* \cdot \alpha x_3 = (T^\dagger y)_3^* \cdot x_3}{y_3^* \cdot (Tx)_3 = y_3^* \cdot \alpha x_2 = (T^\dagger y)_2^*\cdot x_2} \to  \double{(T^\dagger y)_2^* = \alpha y_3^*}{(T^\dagger y)_3^* = \alpha y_2^*} \to  \double{(T^\dagger y)_2
		= \alpha^* y_3}{(T^\dagger y)_3 = \alpha^* y_2}\\
	&\double{y_4^* \cdot (Tx)_4 = y_4^* \cdot \alpha x_5 = (T^\dagger y)_5^*\cdot x_5}{y_5^* \cdot (Tx)_5 = y_5^* \cdot \alpha x_4 = (T^\dagger y)_4^*\cdot x_4} \to \double{(T^\dagger y)_4^* = \alpha y_5^*}{(T^\dagger y)_5^* = \alpha y_4^*}\to \double{(T^\dagger y)_4 = \alpha^* y_5}{(T^\dagger y)_5 = \alpha^* y_4}\\
	&\vdots \nonumber
\end{align}
Da cui otteniamo (per fortuna) lo stesso risultato
\begin{align}
	\triple{(T^\dagger x)_1 \quad\;\; &= &0 \;\qquad \qquad \qquad}{(T^\dagger x)_{2n} \quad &= &\alpha^* x_{2n+1} \quad n> 1}{(T^\dagger x)_{2n+1} &= &\alpha^* x_{2n} \quad\quad n> 1}
\end{align}




Andiamo quindi a calcolare la norma dell'operatore
\begin{align}
	||Tx||^2 &= <Tx,Tx> = \sum_{n=1}^{\infty} \overline{(Tx)}_n (Tx)_n = \nonumber\\
	&=\cancel{\overline{(Tx)}_1 (Tx)_1} + \sum_{n=2}^{\infty} (|\alpha|^2 |x_{2n+1}|^2 + |\alpha|^2 |x_{2n}|^2) = \nonumber\\
	&=\sum_{n=2}^{\infty} |\alpha|^2( |x_{2n+1}|^2 + |x_{2n}|^2) = \nonumber\\
	&=|\alpha|^2 \sum_{n=2}^{\infty} |x_{n}|^2 =|\alpha|^2( ||x||^2 - |x_1|^2) \leq |\alpha|^2 \sum_{n=1}^{\infty} |x_{n}|^2 = |\alpha|^2 ||x||^2 \firstpassage
	||T|| &\leq |\alpha|	
\end{align}
Se $|x_1|=0$ allora la norma è raggiunta(?).
Siccome per operatori limitati e densamente definiti allora $||T|| = ||T^\dagger||$ e quindi 
\begin{align}
	||T^\dagger|| &\leq |\alpha|	
\end{align}
Andiamo ora a calcolare lo \textbf{spettro puntuale}. Dobbiamo risolvere
\begin{align}
	T\vec{x} &= \lambda \vec{x}	
\end{align}
Da cui otteniamo
\begin{align}
	\begin{matrix}
		Tx_1 =& 0 &= \lambda x_1 \\
		Tx_2 =& \alpha x_3 &= \lambda x_2\\
		Tx_3 =& \alpha x_2 &= \lambda x_3\\
		Tx_4 =& \alpha x_5 &= \lambda x_4\\
		Tx_5 =& \alpha x_4 &= \lambda x_5\\
		&\dots
	\end{matrix} 
\end{align}
\newpage
Distinguiamo ora i vari casi
\begin{enumerate}
	\item $\lambda = 0 \spacer \alpha =0$
	\begin{align}
		0 &= 0\cdot x_1 \to x_1 = \text{qualsiasi} \\
		0 &= 0\cdot x_2 \to x_2=\text{qualsiasi}\\
		0 &= 0\cdot x_3 \to x_3=\text{qualsiasi} \\
		&\dots \firstpassage
		\vec{x} &= (x_1,x_2, x_3, \dots)
	\end{align}
	\item $\lambda = 0 \spacer \alpha \neq 0$
	\begin{align}
		0 &= 0\cdot x_1 \to x_1 = \text{qualsiasi} \\
		\alpha x_3 &= 0\cdot x_2 \to x_3=0\\
		\alpha x_2 &= 0\cdot x_3 \to x_2=0 \\
		&\dots \firstpassage
		\vec{x} &= (x_1,0, 0, \dots)
	\end{align}
	\item $\lambda \neq 0 \spacer \alpha =0$
	\begin{align}
		0 &= \lambda x_1 \to x_1=0 \\
		0 &= \lambda x_2 \to x_3=0\\
		0 &= \lambda x_3 \to x_2=0 \\
		&\dots \firstpassage
		\vec{x} &= (0,0, 0, \dots)
	\end{align}
	\item $\lambda \neq 0 \spacer \alpha \neq 0$
	\begin{align}
		0 &= \lambda x_1 \to x_1=0 \\
		\alpha x_2 &= \lambda x_3 \to x_2=\frac{\lambda}{\alpha} x_3\\
		\alpha x_3 &= \lambda x_2 \to x_3=\frac{\lambda^2}{\alpha^2} x_3 \to \left(1 - \frac{\lambda^2}{\alpha^2}\right) x_3 = 0 \\
		\alpha x_4 &= \lambda x_5 \to x_4=\frac{\lambda}{\alpha} x_5\\
		\alpha x_5 &= \lambda x_4 \to x_5=\frac{\lambda^2}{\alpha^2} x_5 \to \left(1 - \frac{\lambda^2}{\alpha^2}\right) x_5 = 0\\
		&\dots \firstpassage
		x_{2n} &=\frac{\lambda}{\alpha} x_{2n+1} \to \left(1 - \frac{\lambda^2}{\alpha^2}\right) x_{2n+1} = 0 
	\end{align}
	
	\newpage
	Dobbiamo distinguere due casi
	\begin{enumerate}
		\item $|\lambda| \neq |\alpha|$
		\begin{align}
			&\left(1 - \frac{\lambda}{\alpha^2}\right) x_{2n+1} = 0 \to x_{2n+1} = 0 \;\; \forall n \in \N \to x_n = 0 \forall n\in \N \firstpassage
			&\vec{x} = \vec{0}
		\end{align}
		\item $|\lambda| = |\alpha| \to \lambda = e^{i\phi} |\alpha|$
		\begin{align}
			&x_{2n} =\frac{e^{i\phi} |\alpha|}{|\alpha|} x_{2n+1} \to x_{2n} = e^{i\phi}x_{2n+1}\firstpassage
			&\vec{x} = (0, e^{i\phi}x_{3}, x_3, e^{i\phi}x_{5}, x_5,\dots)\\
			&\sigma_p(T) = \{ \lambda \in \C \quad|\quad \lambda = e^{i\phi} |\alpha| \spacer \alpha \in \C \}
		\end{align}
	\end{enumerate}
\end{enumerate}

Si ottiene un risultato analogo per $\sigma(T^\dagger)$.

Andiamo ora a calcolare lo \textbf{spettro residuo} dell'operatore e del suo aggiunto.

Andiamo quindi a cercare 
\begin{align}
	&z\neq \lambda \taleche \exists \vec{\eta} \neq \vec{0} \taleche <\vec{\eta} , \vec{v}> =0 \\
	&\vec{v} = (z\mathbb{1} - T)\vec{x} \in \text{Range}(z\mathbb{1} - T) \quad \forall \vec{x}\in \mathcal{D}_T
\end{align}
Nel nostro caso abbiamo
\begin{align}
	z\mathbb{1} - T= 
	\begin{bmatrix}
		z      && 0       && 0       && 0       && 0       && 0      && \dots\\
		0      && z       && -\alpha && 0       && 0       && 0      && \dots\\
		0 	   && -\alpha && z       && 0       && 0       && 0      && \dots\\
		0      && 0       && 0       && z       && -\alpha && 0      && \dots\\
		0 	   && 0       && 0       && -\alpha && z       && 0      && \dots\\
		\vdots && \vdots  && \vdots  && \vdots  && \vdots  && \vdots && \ddots
	\end{bmatrix}
\end{align}
Da cui segue che
\begin{align}
	(z\mathbb{1} - T)\vec{x} = \left( \begin{matrix}
		zx_1\\
		zx_2 - \alpha x_3\\
		zx_3 - \alpha x_2\\
		zx_4 - \alpha x_5\\
		zx_5 - \alpha x_4\\
		\vdots
	\end{matrix} \right)
\end{align}
E arriviamo quindi a scrivere
\begin{align}
	0 &= <\vec{\eta}, \vec{v} > = \sum_{n} \overline{\eta}_n v_n = \nonumber\\
	&= \eta_1^*zx_1 + \eta_2^*(zx_2 - \alpha x_3) + \eta_3^*(zx_3 - \alpha x_2) + \dots = \nonumber \\
	&= x_1z\eta_1^* + x_2(z\eta_2^* - \alpha \eta_3^*) + x_3(z\eta_3^* - \alpha \eta_2^*) + \dots
\end{align}
Siccome la somma è nulla, questo ci porta a scrivere
\begin{align}
	\eta^*_1 =& 0 \\
	\eta^*_2 =& \frac{\alpha}{\lambda} \eta^*_3\\
	\eta^*_3 =& \frac{\alpha}{\lambda} \eta^*_2 = \frac{\alpha^2}{\lambda^2} \eta^*_3 \to  \left(1-\frac{\alpha^2}{\lambda^2}\right) \eta^*_3 = 0\\
	&\vdots \nonumber
\end{align}
Abbiamo però un "problema". Siccome $z \neq \lambda = e^{i\phi} |\alpha|$, abbiamo che
\begin{align}
	1-\frac{\alpha^2}{\lambda^2} \neq 0 \to \eta_3^* = 0 \to \vec{\eta} = \vec{0}
\end{align}
Ma questo non è possibile per definizione, quindi abbiamo
\begin{align}
	\sigma_\rho(T) = \emptyset
\end{align}

In conclusione rispondiamo alle domande del quarto punto:
\begin{enumerate}
	\item L'operatore è \textbf{autoaggiunto} per $\alpha = \alpha^*$, ovvero per $\alpha \in \R$, dato che in tale caso segue $T= T^\dagger$ e $\mathcal{D}_T=\mathcal{D}_{T\dagger}$
	\item Non può mai essere \textbf{unitario}, dato che $TT^\dagger \neq \mathbb{1} \neq T^\dagger T$
	\item Non è \textbf{invertibile} dato che $z\in \sigma_p(T)$
\end{enumerate}


\newpage

\subsection{Esempio 2 con metodo esplicito e analitico (Davide Bufalini, Alessandro Marcelli)}

Sia l'operatore in $l^2(\C)$ definito da
\begin{align}
	\double{(Tx)_1 =& 2^a x_2\quad\quad\quad\quad\quad\quad\quad\quad\;\;\;\;\;}{(Tx)_n =& x_n + (n+1)^ax_{n+1} \quad n \geq 2}
\end{align}

Si richiede di
\begin{enumerate}
	\item Costruire l'aggiunto
	\item Discuttere gli spettripuntuali
	\item Discutere se, al variare di $a\in\R$, $z_0=0$ appartiene allo spettro dell'operatore o dell'aggiunto
\end{enumerate}


Iniziamo dal primo punto.

Un possiible metodo è quello di \textbf{esplicitare} l'azione dell'operatore
\begin{align}
	\begin{matrix}
		Tx_1 &= &2^a x_2\\
		Tx_2 &= &x_2 +3^a x_3\\
		Tx_3 &= &x_3 +4^a x_4\\
		\vdots
	\end{matrix} \to T= 
	\begin{bmatrix}
		0 	   && 2^a    && 0 	   && 0 	 && 0 	   && 0 	 && \dots \\
		0 	   && 1 	 && 3^a    && 0 	 && 0 	   && 0 	 && \dots \\
		0	   && 0      && 1 	   && 4^a 	 && 0 	   && 0 	 && \dots \\
		0 	   && 0 	 && 0	   && 1	     && 5^a    && 0	     && \dots \\
		0 	   && 0 	 && 0 	   && 0		 && 1 	   && 6^a 	 && \dots \\
		\vdots && \vdots && \vdots && \vdots && \vdots && \vdots && \ddots
	\end{bmatrix}
\end{align}
Da cui otteniamo quindi
\begin{align}
	T^\dagger= 
	\begin{bmatrix}
		0 	   && 0      && 0 	   && 0 	 && 0 	   && 0 	 && \dots \\
		2^a    && 1 	 && 0      && 0 	 && 0 	   && 0 	 && \dots \\
		0	   && 3^a    && 1 	   && 0 	 && 0 	   && 0 	 && \dots \\
		0 	   && 0 	 && 4^a	   && 1	     && 0      && 0	     && \dots \\
		0 	   && 0 	 && 0 	   && 5^a	 && 1 	   && 0 	 && \dots \\
		\vdots && \vdots && \vdots && \vdots && \vdots && \vdots && \ddots
	\end{bmatrix} \to T^\dagger = \double{(T^\dagger x)_1 =& 0}{(T^\dagger x)_n =& x_n + n^a x_{n-1} \quad n \geq 2}
\end{align}
Altrimenti possiamo calcolarlo \textbf{analiticamente}, studiando il prodotto
\begin{align}
	&(y,Tx) = (T^\dagger y, x)\firstpassage
	&\sum_{i=1}^{n} y_i^* (Tx)_i =  \sum_{i=1}^{n} (T^\dagger y)_i^* x_i
\end{align}
Da cui otteniamo che
\begin{align}
	y_1^* (Tx)_1 &= y_1^* 2^a x_2 = (T^\dagger y)_1^* x_1\\
	y_2^* (Tx)_2 &=	y_2^*( x_2 + 3^a x_3) = (T^\dagger y)_2^* x_2\\
	y_3^* (Tx)_3 &=	y_3^*( x_3 + 4^a x_4) = (T^\dagger y)_3^* x_3\\
	y_4^* (Tx)_4 &=	y_4^*( x_4 + 4^a x_5) = (T^\dagger y)_4^* x_4\\
	y_5^* (Tx)_5 &=	y_5^*( x_5 + 6^a x_6) = (T^\dagger y)_5^* x_5\\
	&\vdots \nonumber
\end{align}
Possiamo quindi scrivere
\begin{align}
	&y_1^* 2^a x_2 + y_2^* x_2 + y_2^* 3^a x_3 + y_3^* x_3 + y_3^* 4^a x_4  + y_4^* x_4 + y_4^* 5^a x_5 + \dots = \nonumber \\
	=& (y_1^* 2^a + y_2^*)x_2 + (y_2^* 3^a + y_3^*) x_3 + (y_3^* 4^a + y_4^*) x_4 =  \\
	=&(T^\dagger y)_1^* x_1 + (T^\dagger y)_2^* x_2 + (T^\dagger y)_3^* x_3 + (T^\dagger y)_4^* x_4 + \dots
\end{align}
Confrontando i termini otteniamo quindi
\begin{align}
	(T^\dagger y)_1^* &= 0\\
	(T^\dagger y)_2^* &= y_1^* 2^a + y_2^*\\
	(T^\dagger y)_3^* &= y_2^* 3^a + y_3^*\\
	(T^\dagger y)_4^* &=y_3^* 4^a + y_4^*\\
	\vdots \nonumber
\end{align}
Da cui (per fortuna) otteniamo di nuovo
\begin{align}
	T^\dagger = \double{(T^\dagger x)_1 =& 0}{(T^\dagger x)_n =& x_n + n^a x_{n-1} \quad n \geq 2}
\end{align}

Andiamo ora a calcolare gli spettri puntuali dell'operatore e del suo aggiunto.
\begin{align}
	&Tx = \lambda x \to (Tx)_n = \lambda x_n\firstpassage
	&(Tx)_1 = 2^a x_2 = \lambda x_1\\
	&(Tx)_2 = x_2 + 3^a x_3 = \lambda x_3\\
	&(Tx)_3 = x_3 + 4^a x_4 = \lambda x_4\\
	&(Tx)_4 = x_4 + 5^a x_5 = \lambda x_5\\
	\vdots \nonumber
\end{align}

Dobbiamo ora distinguere i diversi casi: 

\begin{enumerate}
	
	\item $\lambda = 0 \spacer a=0$
	\begin{align}
		&x_1 = \text{indeterminato}\\
		&x_2 = 0\\
		&x_2 + x_3 = 0 \to x_3 =0\\
		&x_3 + x_4 = 0 \to x_4 =0\\
		&x_4 + x_5 = 0 \to x_5=0\\
		&\vdots \nextpassage
		&\vec{x} = (x_1, 0, 0, 0, \dots) \spacer \nu = 1
	\end{align}
	
	\item $\lambda = 0 \spacer a \neq 0$
	\begin{align}
		&x_1 = \text{indeterminato}\\
		&2^a x_2 = 0 \to x_2 = 0\\
		&x_2 + 3^a x_3 = 0 \to x_3 =0\\
		&x_3 + 4^a x_4 = 0 \to x_4 =0\\
		&x_4 + 5^a x_5 = 0 \to x_5=0\\
		&\vdots \nextpassage
		&\vec{x} = (x_1, 0, 0, 0, \dots) \spacer \nu = 1
	\end{align}
	
	\item $\lambda \neq 0 \spacer a \neq 0$
	\begin{align}
		&2^ax_2 = \lambda x_1 \to \double{x_1 =& \text{indeterminato}}{x_2 =& \frac{\lambda}{2^a} x_1}\\
		&x_2 + 3^a x_3 = \lambda x_2 \to x_3 = \frac{\lambda -1}{3^a} x_2 \\
		&x_3 + 4^a x_4 = \lambda x_3 \to x_4 = \frac{\lambda -1}{4^a} x_3 = \frac{(\lambda -1)^2}{(3\cdot4)^a} x_2  = \frac{\lambda(\lambda -1)^2 }{(2\cdot 3\cdot4)^a} x_1 =  \frac{\lambda(\lambda -1)^2 }{(4!)^a} x_1\\
		&x_4 + 5^a x_5 = \lambda x_4 \to x_5 = \frac{\lambda -1}{5^a} x_4 = \dots =  \frac{\lambda(\lambda -1)^3 }{(5!)^a} x_1\\
		&\vdots \nextpassage
		&\triple{x_1 =& \text{indeterminato}}{x_2 =& \frac{\lambda}{2^a} x_1}{x_n =& \frac{\lambda(\lambda -1)^{n-2} }{(n!)^a} x_1 \spacer n>2}
	\end{align}
	
	La situazione si complica. Dobbiamo trovare per quali $\lambda$ e $a$ abbiamo $\vec{x}\in l^2(\C)$. Andiamo a calcolare la norma
	\begin{align}
		||x||^2 &= \sum_{n=1}^{+\infty} |x_n|^2 = |x_1|^2 + \left| \frac{\lambda}{2^a} x_1 \right|^2 + \sum_{n=3}^{+\infty} \left| \frac{\lambda(\lambda -1)^{n-2} }{(n!)^a} x_1 \right|^2 = \nonumber \\
		&= \left( 1 + \frac{|\lambda|^2}{2^{2a}} \right) |x_1|^2 + \sum_{n=3}^{+\infty}  \frac{|\lambda|^2|\lambda -1|^{2n-4} }{(n!)^{2a}} \left| x_1 \right|^2
	\end{align}
	Applichiamo ora il criterio della radice, ovvero
	\begin{align}
		n| \sim \frac{n^n}{e^n} \sqrt{2\pi n}
	\end{align}
	E otteniamo
	\begin{align}
		\frac{|\lambda|^2|\lambda -1|^{2n-4} }{(n!)^{2a}} \sim \frac{|\lambda|^\frac{2}{n}|\lambda -1|^\frac{{2n-4}}{n} }{\left(\frac{n^n}{e^n} \sqrt{2\pi n}\right)^\frac{2a}{n}} \sim \left| \frac{(\lambda -1)^2 e^{2a}}{n^{2a}} \right| \to \triple{0 \quad &a>0}{|\lambda -1|^2 \quad &a=0}{\infty \quad &a<0}
	\end{align}
	Il caso $a<0$ diverge, e quindi lo escludiamo a priori. Studiamo ora gli altri due
	\begin{enumerate}
		\item $a = 0 \to |\lambda -1|^2<1 \to |\lambda -1| < 1$ 
		
		Abbiamo quindi una regione di convergenza circolare centrata in 1 con raggio di convergenza pari a 1.
		
		\item $a > 0 \to 0$
		
		In questo caso la serie converge $\forall \lambda \in \C$, e abbiamo un raggio di convergenza infinito.
	\end{enumerate}
	
\end{enumerate}

\newpage

Andiamo ora a studiare lo spettro puntuale dell'aggiunto.
\begin{align}
	&T^\dagger v = \lambda v \to (T^\dagger  v)_n = \lambda v_n\firstpassage
	&(T^\dagger  v)_1 = 0 = \lambda v_1\\
	&(T^\dagger  v)_2 = v_2 + 2^a v_1 = \lambda v_2\\
	&(T^\dagger  v)_3 = v_3 + 3^a v_2 = \lambda v_3\\
	&(T^\dagger  v)_4 = v_4 + 4^a v_3 = \lambda v_4\\
	\vdots \nonumber
\end{align}

Di nuovo, andiamo a distinguere i vari casi
\begin{enumerate}
	\item $\lambda = 0 \spacer a=0$
	\begin{align}
		&v_1 = \text{qualsiasi}\\
		&v_2 + v_1 = 0 \to v_2 = -v_1\\
		&v_3 + v_2 = 0 \to v_3 = -v_2 = v_1\\
		&v_4 + v_3 = 0 \to v_4 = -v_3 = -v_1\\
		\vdots \nextpassage
		&\vec{v} = (v_1, -v_1 , v_1, \dots, (-1)^{n+1}v_1, \dots) \spacer \nu = 1
	\end{align}
	
	
	
	\item $\lambda = 0 \spacer a \neq 0$
	\begin{align}
		&v_1 = \text{qualsiasi}\\
		&v_2 + 2^a v_1 = 0 \to v_2 = -2^a v_1 \\
		&v_3 + 3^a v_2 = 0 \to v_3 = -3^a v_2 = 6^a v_1 \\
		&v_4 + 4^a v_3 = 0 \to v_2 = -4^a v_3 = -24^a v_1  \\
		\vdots \nextpassage
		&\vec{v} = (v_1, -2^a v_1 ,6^a v_1, \dots, (-1)^{n+1} (n!)^a v_1, \dots) \spacer \nu = 1
	\end{align}
	
	
	
	\item $\lambda \neq 0 \spacer a \neq 0$
	\begin{align}
		&v_1 = 0\\
		&v_2 = \lambda v_2 \to (1 - \lambda)v_2 = 0 \to \double{v_2 = 0}{\lambda = 1}\\
		&(1 - \lambda)v_3 = -3^a v_2\\ 
		\vdots \nonumber
	\end{align}
	Distinguiamo i due casi
	\begin{enumerate}
		\item $\lambda \neq 1 \to v_2 = 0 \to v_3 = 0 \to \dots \to \vec{v} = \vec{0}$
		\item $\lambda = 1 \to v_2 = \text{qualsiasi} \to v_3 = v_3 -3^a v_2 \to v_2 = 0 \to \vec{v} = \vec{0}$
	\end{enumerate}
\end{enumerate}

\newpage

Andiamo ora a studiare lo \textbf{spettro residuo}. Vale il teorema per cui
\begin{align}
	&\sigma_\rho (A) \subset \overline{\sigma_\rho (A^\dagger)} \leftrightarrow \sigma_\rho (A^\dagger) \subset \overline{\sigma_\rho (A)}\\
	&0 \in\sigma_\rho (A) \to 0 \in \overline{\sigma_\rho (A^\dagger)}
\end{align}

CHIEDERE A PAOLO
