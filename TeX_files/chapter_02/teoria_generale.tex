\section{Spunti di teoria generale (Alessandro Marcelli)}

\subsection{Nozioni generali sugli spazi}

\subsubsection{Spazi vettoriali}

Si definisce uno \textbf{spazio vettoriale} $V$ come un insieme su un campo $K$ (di solito $\R$ o $\C$) dove sono definite le operazioni di
\begin{enumerate}
	\item \textbf{Somma vettoriale}
	\begin{align}
		V \times V &\rightarrow V\\
		(v_1,v_2)&\rightarrow v_1 + v_2
	\end{align}
	\item \textbf{Prodotto per uno scalare}
	\begin{align}
		K \times V &\rightarrow V\\
		(\lambda,v)&\rightarrow \lambda v
	\end{align}
\end{enumerate}

Preso $A$ sottoinsieme di $V$, ne definiamo i \textbf{punti di aderenza} come quei punti tali che, preso un loro intorno, contengono almeno un punto di $A$. \newline
In particolare, preso un punto di aderenza $z_0$, se nel suo intorno contiene almeno un punto $z_1 \neq z_0$ appartenente ad $A$, allora $z_0$ si dice \textbf{punto di accumulazione}.

Possiamo quindi dividere i punti di aderenza in due categorie
\begin{enumerate}
	\item tutti i punti di A
	\item  i punti di accumulazione esterni ad $A$
\end{enumerate}

Possiamo ora definire la \textbf{chiusura} $\overline{A}$ come l'insieme dei punti di aderenza di $A$. 

Un insieme $A$ si dice \textbf{denso} in $B$ se $B \subseteq \overline{A}$. In particolare se $\overline{A} = V$ l'insieme $A$ si dice \textbf{ovunque denso}.



\newpage
\subsubsection{Spazi metrici, normati ed euclidei}

Dato uno spazio  vettoriale V e, presi due elementi dello spazio $x$ e $y$, definita la funzione \textbf{distanza} $d(x,y)$ tale che
\begin{align}
	d(x,y) &\geq 0 \spacer d(x,y) = 0 \leftrightarrow x=y\\
	d(x,y) & = d(y,x)\\
	d(x,y) &\leq d(x,z) + d(y,z \spacer \forall x,y,z \in M) 
\end{align}

Si definisce \textbf{Spazio Metrico} la coppia $(V,d)$.

Preso di nuovo lo spazio vettoriale $V$ e presi $v,w\in V$ e $\alpha \in K$ si può definire la \textbf{norma} $||\cdot|| \taleche V \rightarrow \R$ come
\begin{align}
	||v|| &\geq 0 \spacer ||v|| = 0 \leftrightarrow v= 0\\
	||\alpha v || &= |\alpha|\cdot ||v||\\
	||v + w|| &\leq ||v|| + ||w||
\end{align}

La coppia $(V, ||\cdot ||)$ si chiama \textbf{spazio normato}. 


\textbf{Importante:} tutti gli spazi normati sono anche spazi metrici, con $d(x,y) = ||x-y||$, ma non tutti gli spazi metrici sono normati.

In uno spazio metrico abbiamo che un punto $x_a\in M$ sarà
\begin{enumerate}
	\item di \textbf{aderenza} se e solo se $\exists \{x_n\} \in M \taleche \limit{n}{+\infty} x_n = x_a$
	\item di \textbf{accumulazione} se e solo se $\exists \{x_n\neq x_a\} \in M \taleche \limit{n}{+\infty} x_n = x_a$
\end{enumerate}

Definiamo ora una \textbf{successione di Cauchy} come tale se
\begin{align}
	\forall \epsilon > 0 \; \exists n_\epsilon \taleche d(x_m,x_n) <\epsilon \quad \forall m,n > n_\epsilon
\end{align}
Possiamo quindi dire che uno spazio metrico $V$ si dice \textbf{completo} se ogni successione di Cauchy in esso definita è convergente.


Uno spazio \textbf{euclideo} è uno spazio normato la cui norma è definita attraverso un \textbf{prodotto scalare}, ovvero un'applicazione $ (\cdot , \cdot) \taleche V \times V \rightarrow \C $ che, $\forall x,y,z \in V$ e $\forall \lambda, \mu \in \C$, soddisfa
\begin{align}
	&(x,y) = \overline{(y,x)} \quad &\text{Hermitianità}\\
	&(x, \lambda y + \mu z) = \lambda(x,y) + \mu (x,z) \quad &\text{Linearità}\\
	&(x,y) \geq 0 \spacer (x,x) = 0 \leftrightarrow x=0
\end{align}

Uno spazio euclideo completo si dice \textbf{spazio di Hilbert}.




\newpage

\subsubsubsection{Teorema del completamento}
Uno spazio metrico non completo può essere "completato", combinandolo con l'insieme contentente tutti i limiti delle sue succesioni di Cauchy, al prezzo di considerare equivalenti le successioni con identico limite. 

Possiamo fare questo appoggiandoci al \textbf{teorema del completamento:} \textit{dato uno spazio metrico $(X,d)$, esistono
\begin{enumerate}
	\item uno spazio metrico completo $(\tilde{X}, \tilde{d})$
	\item una applicazione isometrica di $X$ su un insieme denso di $(\tilde{X}, \tilde{d})$
\end{enumerate}
}

Per dimostrare il teorema, iniziamo definendo
\begin{align}
	Z = \left\{ \{z_n\} \taleche z_n \in X \; ,\; \{z_n\} \; \text{successione di Cauchy in } X \right\}
\end{align}
E definiamo la seguente relazione di equivalentza tra elementi di $Z$
\begin{align}
	\limit{n}{\infty} d(x_n,y_n) = 0 \implies \{x_n\} \sim \{y_n\} 
\end{align}
PAOLO


\newpage


\subsubsubsection{Completezza di $l_2$}
Si può dimostrare la completezza dello spazio metrico $l_2$. Sia una successione di elementi di $l_2$ (successione di successioni)
\begin{align}
	\mathbf{x}_n = \{ x_n^1,  x_n^2, \dots, x_n^i, \dots\}\spacer \sum_{i=1}^{\infty} |x_n^i|^n < \infty
\end{align}
se la successione è di Cauchy, preso $\epsilon>0$ avremo che
\begin{align}
	 \exists n_\epsilon \taleche ||x_n-x_m||<\epsilon \quad \forall n,m\in n_\epsilon
\end{align}
Dobbiamo dimostrare come tale successione converga sempre ad un elemento di $l_2$.

Se consideriamo la comonente i-esima di ogni elemento della successione avremo
\begin{align}
	|(\mathbf{x}_n-\mathbf{x}_m)^i| < ||\mathbf{x}_n-\mathbf{x}_m|| = \sum_{i=1}^{\infty} |(\mathbf{x}_n-\mathbf{x}_m)^i|<\epsilon
\end{align}
Questo implica che la successione delle i-esime componenti è di Cauchy in $\R$, che essendo completo garantisce la convergenza delle successioni.
Se ora definiamo
\begin{align}
	x^i = \limit{n}{\infty}x_n^i
\end{align}
Possiamo quindi dire che la successione  $\{\mathbf{x}_n\}$ ha come limite la $\{\mathbf{x}\} = \{ x^1,x^2 ,\dots, x^i,\dots \}$ e ci basta dimostrare come questa dia un elemento di $l_2$, ovvero che
\begin{align}
	\sum_{i=1}^{\infty} |x^i|^2 < \infty
\end{align}
Se consideriamo la somma finita $\sum_{i=1}^{N} |x_n^i - x_m^i|$ possiamo procedere nel seguente modo
\begin{align}
	&\limit{m}{\infty}\left(\sum_{i=1}^{N} |x_n^i - x_m^i|\right) = \sum_{i=1}^{N} |x_n^i - x^i|\firstpassagecomm{per $n$ abbastanza grande}
	&\sum_{i=1}^{N} |x_n^i - x^i| < \epsilon
\end{align}
Per $N\to \infty$ possiamo quindi scrivere 
\begin{align}
	\sum_{i=1}^{\infty} |x^i|^2 &= \sum_{i=1}^{\infty} |-x^i|^2 = \sum_{i=1}^{\infty} |-x^i + x_n^i - x_n^i|^2 = \nonumber\\
	&= \sum_{i=1}^{\infty} | x_n^i -x^i - x_n^i|^2 \leq  \sum_{i=1}^{\infty} | x_n^i -x^i|^2 + \sum_{i=1}^{\infty}|- x_n^i|^2<\infty
\end{align}
	
\newpage

\subsubsubsection{Teorema di ortonormalizzazione}

\textit{Sia $\{f_n\}$ un sistema di vettori linearmente indipendenti in uno spazio di Hilbert separabile. Allora esiste un sistema ortonormale $\{\phi_n\}$ equivalente al precedente}
\begin{align}
	\phi_n = \sum_{k} a_{nk} f_k \quad\leftrightarrow\quad f_n = \sum_{k} a_{nk} \phi_k
\end{align}

Dimostrazione: lul dopo

\newpage


\subsection{Operatori}

Presi due spazi di Hilbert $H_1$ e $H_2$, definiamo un \textbf{operatore lineare} $A$ come un'applicazione del tipo
\begin{align}
	&A \taleche \mathcal{D}_A \subseteq H_1 \rightarrow R_A \subseteq H_2 \\
	&A(\alpha v + \beta w) = \alpha A(v) + \beta A(w) \spacer \forall v,w \in \mathcal{D}_A \quad \forall \alpha, \beta \in K
\end{align}
Un operatore si dice
\begin{enumerate}
	\item \textbf{Limitato} se $\exists N\in \R \taleche ||A(x)|| \leq N||x|| \quad \forall x \in \mathcal{D}_A$
	\item \textbf{Continuo} se $\forall \{ x_n\} \in \mathcal{D}_A \taleche x_n \rightarrow x$ segue che $\limit{n}{infty} A(x_n) = A(x)$
	\textbf{Nota bene:} un operatore lineare $(A,\mathcal{D}_A)$ è continuo se e solo se è limitato.
\end{enumerate}

Iniziamo definendo il seguente funzionale lineare
\begin{align}
	&x \rightarrow \phi_\nu(x)\\
	&\phi_\mu(x) = (\mu, Ax) \quad \forall x \in \mathcal{D}_A
\end{align}
Dato quindi un operatore $A$, si definisce il suo \textbf{aggiunto} $A^\dagger$ come l'operatore per cui, grazie anche al teorema di Riesz
\begin{align}
	(\mu, Ax) = (A^\dagger \mu, x)
\end{align}
Se $A$ è lineare, limitato e densamente definito ne segue che $||A|| = ||A^\dagger||$.

Un'operatore $A$ si dice \textbf{unitario} quando
\begin{enumerate}
	\item è \textbf{Isometrico}, ovvero 
	\begin{align}
		&(Ax,Ay) = (x,y) \quad \forall x,y \in \mathcal{D}_A\\
		&||A|| = 1
	\end{align}
	\item il suo range è denso in H
\end{enumerate} 
Per gli operatori unitari vale che
\begin{align}
	U^\dagger U = \mathbb{1}
\end{align}

Un operatore si dice \textbf{simmetrico} quando
\begin{align}
	(x,Ay) = (Ax,y) \spacer \forall x,y \in \mathcal{D}_A
\end{align}
Se un operatore è simmetrico e $\mathcal{D}_A = \mathcal{D}_{A^\dagger}$ allora si dice che è \textbf{autoaggiunto}.

\newpage

\subsubsection{Spettri per operatori Infinito-Dimensionali}

Sia $H$ uno spazio di Hilbert. Dato un \textbf{operatore lineare densamente definito}
\begin{align}
	A : \mathcal{D}_A \subseteq H \rightarrow R_A \subseteq H
\end{align} 
e definita la famiglia di operatori
\begin{align}
	&T_z(A) \taleche \mathcal{D}_A \rightarrow H\\
	&T_z(A) = z\mathbb{1} - A \spacer z \in \C
\end{align}


Si diche che $z$ appartiene all'\textbf{insieme risolvente} $Res(A)$ dell'operatore $A$ se $T_z (A)$ è \textbf{biunivoco} con \textbf{inverso limitato}.

Tale inverso prende il nome di \textbf{Operatore risolvente} di $A$ e viene definito come
\begin{align}
	&R_z(A) \taleche \mathcal{R}_{T_z(A)} \rightarrow H\\
	&R_z(A) = (z\mathbb{1} - A)^{-1}
\end{align}

Se invece $z \notin Res(A)$, allora si dice che esso appartiene allo spettro di $A$.
Rispetto agli operatori finito-dimensionali questo non implica che $z$ sia autovalore di $A$. Ci troviamo quindi a dividere gli elementi dello spettro in tre categorie:
\begin{enumerate}
	\item Lo \textbf{spettro puntuale} $\sigma_p(A)$, ovvero l'insieme degli autovalori di $A$
	\begin{align}
		 \sigma_p= \{ z \in \C \taleche  z \mathbb{1}\vec{x} = A\vec{x} \; , \vec{x} \neq \vec{0} \} 
	\end{align}
	
	\item Lo \textbf{spettro residuo} $\sigma_\rho(A)$, ovvero l'insieme degli $z\notin \sigma_p(A)$ per i quali $R_z(A)$ non è densamente definito, il che si traduce in
	\begin{align}
		 \sigma_\rho= \{ z \in \C \taleche \exists \vec{\eta} \neq \vec{0} \taleche  (\vec{\eta}, (z\mathbb{1} -A)\vec{x})=\vec{0} \} 
	\end{align}
	\item \textbf{Spettro continuo} $\sigma_c(A)$, ovvero l'insieme degli $z\notin \sigma_p(A) \cup \sigma_\rho (A)$  per i quali $R_z(A)$, per quanto densamente definito, non è limitato, ovvero
	\begin{align}
		\sigma_c= \{ z \in \C \taleche \exists \{\vec{x}^{(n)}\} \taleche  ||(z\mathbb{1} -A)^{-1}||> M \; \forall M\in \N \} 
	\end{align}
	
\end{enumerate}




\newpage
